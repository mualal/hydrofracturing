\documentclass[main.tex]{subfiles}

\begin{document}

% \textcolor{red}{Вводная лекция}

\section{Лекция 16.03.2021 (Донцов Е.В.)}

\subsection{Математическая модель радиальной трещины ГРП}

\includegraphics[width=\textwidth, page=49]{HF_slides_2021_corrected.pdf}

Напомню: в случае радиальной геометрии рассматриваем планарную трещину (растёт, например, в плоскости $Oxy$), есть точечный источник и объёмная закачка жидкости $Q_0$; предполагается, что слоёв нет (или эффекты слоёв малы);
радиальная симметрия и задача всё равно одномерная; единственная разница с плоской трещиной в том, что дивергенция потока будет записана в цилиндрических координатах и уравнение упругости будет другим (а именно другое ядро $M$ -- довольно сложная функция -- её можно найти, но важно знать, что она другая).
\\

Уравнения для радиальной и плоской трещин очень похожи: также есть утечки, есть связь потока и градиента давления, есть критерий распространения.

Аналогично, в случае радиальной трещины можем взять уравнение и оценить значения в нём по порядку величины (ввести безразмерные величины).

Но в чём разница с плоской трещиной?

Появился квадрат в первом уравнении перед характерным масштабом длины, т.к. в исходном уравнении изменились расход (он был двухмерный с размерностью м$^2$/с; теперь он трёхмерный с размерностью м$^3$/с) и дельта-функция (она была одномерной с размерностью 1/м; теперь у неё размерность 1/м$^2$).

И такая разница в один характерный масштаб длины даёт совершенно другой ответ для решения.
Такой, так сказать, геометрический фактор.

\includegraphics[width=\textwidth, page=50]{HF_slides_2021.pdf}

Аналогично, можем оценить решение в пределе viscosity-storage (т.е. когда нет трещиностойкости и нет утечек): решаем алгебраическую систему уравнений аналитически и получаем зависимость радиуса, открытия и давления от времени.
Здесь радиус зависит от времени в степени 4/9; в плоской трещине степень была равна 2/3 (т.е. в случае радиальной трещины радиус от времени растёт не так быстро -- что вполне логично: наш блин растекается, площадь становится больше и поэтому скорость роста со временем уменьшается).

Если сравним с более полным (но тоже приближённым) решением, увидим, что добавились численные множители и зависимость от безразмерного радиуса, которые мы не могли найти с помощью обезразмеривания.

\includegraphics[width=\textwidth, page=51]{HF_slides_2021.pdf}

Абсолютно аналогично можно найти приближённое решение; я не буду рассказывать про детали: берём асимптотику, закон сохранения объёма, решаем и получаем приближённое решение (единственная разница будет в интегрировании по объёму за счёт появившегося дополнительного радиуса).

Но сильно изменяется параметрическое пространство и то, как решение себя ведёт.
Здесь сильно другие безразмерные параметры: по оси $Ox$ безразмерное время (оно теперь не зависит от утечек); по очи $Oy$ безразмерные утечки.

Аналогично есть 4 концептуальных предела.

Но важно, что теперь во времени решение может перескакивать от одного предела к другому.
\\

Можно понять на пальцах: когда трещина очень маленькая, то реализуется режим $M$ (скорость распространения относительно большая), далее при продолжении закачки трещина расширяется (скорость распространения становится меньше) и вязкостная диссипация падает, т.е. переходим в режим $K$; если продолжается дальнейший рост трещины, то её площадь ещё больше возрастает и утечки через эту большую площадь начинают доминировать.
\\

Ещё раз: качественно поведение совсем другое, но то, как вы можете использовать параметрическое пространство, идентично тому, что было в плоской трещине, т.е. мы можем посчитать два безразмерных параметра, они дадут нам точку на параметрическом пространстве, из которой вы узнаете текущий режим.

\includegraphics[width=\textwidth, page=52]{HF_slides_2021.pdf}

Для каждого из режимов есть аналитические решения на открытие, радиус и давление.

Из этих всех формул важно знать, как зависит радиус от времени (в степени примерно 1/2, если утечки маленькие, и в степени 1/4, если утечки большие).

Радиусы в режимах $\tilde{M}$ и $\tilde{K}$ одинаковы (определяется только утечками).

\includegraphics[width=\textwidth, page=53]{HF_slides_2021.pdf}

Здесь всё примерно то же самое, что и для плоской трещины: решение переходит от одного к другому (со временем).
То же самое с формой: в режиме трещиностойкости эллиптическое решение.
\\

На самом деле все решения (что для плоской трещины, что для радиальной) -- открытие как функция радиуса -- в самом первом приближении есть эллипсы; при изменении режима этот эллипс немного приплющивается возле кончика.

\includegraphics[width=\textwidth, page=54]{HF_slides_2021.pdf}

Цель графиков, представленных на слайде, показать, что решение плавно переходит от одного к другому.
И мы знаем аналитические решения для пунктирных линий (предельные решения), мы можем их использовать, чтобы оценить полное решение без использования численного расчёта.

Т.е. всё то же самое, что и для плоской трещины.

\includegraphics[width=\textwidth, page=55]{HF_slides_2021.pdf}

Здесь показаны точки в параметрическом пространстве для разных значений входных параметров.

Если возьмём различные полевые условия, различные лабораторные условия, то на параметрическом пространстве реально можем быть везде.

Правда вряд ли будем слишком глубоко в режиме $\tilde{M}$ или $\tilde{K}$ (обычно всё таки возле границы), но влево/вправо между трещиностойкостью и вязкостью очень легко прыгать.
\\

Эффективность = доля жидкости, которая осталась в трещине после закачки и утечек.

И без расчётов мы знаем, что оба параметра (и трещиностойкость, и вязкость) важны: оба эти параметра влияют на рост трещины.
\\

Slick Water = практически подобна воде, чуть-чуть более вязкая (эффективность около 1/2).
\\

Linear Gel = вязкий гель.
\\

Чтобы результаты лабораторных экспериментов соответствовали результатам, полученным в полях, необходимо сохранять положение в параметрическом пространстве.

В лабораторных экспериментах для этого обычно увеличивается вязкость.
\\

Замечание: в модели радиальной трещины понятие безразмерная трещиностойкость не актуально, т.к. трещиностойкость $K'$ есть и в $\phi$, и в $\tau$.

\includegraphics[width=\textwidth, page=56]{HF_slides_2021.pdf}

Хочу привести вам интересные результаты численного моделирования распространения множественных трещин ГРП.

Здесь нет сильной теории, подтверждающей результаты; можно считать, что это просто численный эксперимент.
\\

На слайде представлено параметрическое пространство с предельными решениями.

Теперь вместо одной радиальной трещины будем распространять несколько радиальных трещин (10 трещин), которые будут расположены близко друг к другу (т.е. они друг с другом взаимодействуют -- есть уравнение упругости и если мы открываем одну трещину, то меняем напряжённое состояние во всём пространстве и влияем на другие трещины).

Если будем распространять эти трещины в режиме доминирующей вязкости, то они будут оставаться более-менее радиальными и параллельными друг другу (правда не очень равномерное раскрытие, но оно всё же более-менее радиальное).

Если же будем растить эти 10 трещин в режиме доминирующей трещиностойкости, то тогда решение будет нестабильным и все трещины будут расти в разные стороны.

Результатами этих численных экспериментов я хочу вам показать, что понимание физики и доминирующего процесса (вязкости или трещиностойкости) очень важно (это не просто некие аналитические выражения), так есть сильное влияние доминирующего процесса на форму трещин множественного ГРП.
Другими словами, морфология рассматриваемой системы трещин сильно меняется в зависимости от рассматриваемого режима (положения в параметрическом пространстве) -- это очень важно с практической точки зрения.
\\

В режиме трещиностойкости раскрытие трещины в центре очень маленькое, а в режиме доминирующей вязкости в центре всегда самое большое раскрытие трещины (это соответствует принципу минимума затрачиваемой энергии -- при доминировании вязкости есть существенный градиент давления вдоль трещины).
\\

Замечание 1: во все рассматриваемые трещины (10 штук) закачивался одинаковый расход; если это не так, то трещины начали бы бороться за расход (и тогда обычно выигрывают одна или две трещины, а остальные не растут); при использовании постоянного расхода для каждой из трещин они начинают взаимодействовать и при доминировании трещиностойкости возникает нестабильность и трещины начинают расти в разные стороны.
\\

Замечание 2: в режиме трещиностойкости давление по всей длине трещины постоянно и с точки зрения механики трещина не чувствует, где у неё находится источник, вообще говоря, эта радиальная трещина может немного смещаться в разные стороны, потому что она не знает, где у неё центр закачки (давление постоянно);
а в режиме вязкой диссипации давление в точке закачки всегда больше, чем на некотором удалении от источника, соответственно, трещина получается более устойчивой.

На самом деле и в численном решении есть проблемы с $K$ режимом: на грубых сетках трещина может распространяться немного не симметрично (может уходить чуть вправо или чуть влево).

\includegraphics[width=\textwidth, page=57]{HF_slides_2021.pdf}

На этом мы заканчиваем раздел про радиальную трещину.

Аналогично плоской трещине вы должны понимать, как можно на пальцах оценить решение, исходя из масштабов.

Важно понимать определение предельных режимов и их соотношение с асимптотиками на кончике трещины (идентично плоской трещине).

Параметрическое пространство для радиальной трещины существенно отличается от параметрического пространства для плоской трещины.

Важно знать о существовании приближённого полуаналитического решения:
вводите размерные входные параметры задачи, по ним считаются некоторые безразмерные параметры и далее, используя приближённое полуаналитическое решение, можно найти решение поставленной задачи.

Карту режимов можно использовать для тестирования численных схем, для проведения соответствий между лабораторными и полевыми экспериментами, для понимание морфологии системы трещин множественного ГРП.
\\

Если хотите подробно посмотреть на всю математику модели, то предлагаю вам ознакомиться со статьёй, название которой представлено на слайде.

В этой статье есть \href{https://datadryad.org/stash/dataset/doi:10.5061/dryad.gh469}{ссылка на код} (на MatLab), который рассчитывает приближённое решение для всех возможных значений входных параметров. 

\subsection{Математическая модель Перкинса-Керна-Нордгрена (модель PKN)}

\includegraphics[width=\textwidth, page=58]{HF_slides_2021.pdf}

Рассмотрим ещё одну очень важную модель.
Она ещё более важна с точки зрения практики, чем модель радиальной трещины.

Вспомним, что модель плоской трещины легче всего сформулировать и посчитать (она исторически была самой первой моделью), далее усложнили до радиальной модели, и ещё далее усложнили до модели PKN, которую сейчас и будем рассматривать.

Модель PKN (Перкинса-Керна-Нордгрена) тоже одномерная, но она более близка к реальным геометриям.
Ясно, что есть слои, есть разные типы пород и шансы получить радиальную трещину с точки зрения практики малы.

Основные предположения геометрии PKN: есть одна планарная трещина, распространяющаяся симметрично влево и вправо, мы рассматриваем одну сторону;
высота трещины постоянна (предполагается, что сверху и снизу расположены граничащие слои с очень большими напряжениями или с очень высокой трещиностойкостью и трещина туда не распространяется);
длина трещины много больше её высоты;
вертикальная компонента потока пренебрежимо мала и градиентом давления по оси $Oy$ тоже пренебрегаем (предполагаем, что давление постоянно вдоль оси $Oy$;
другими словами, давление является одномерной функцией от $x$);
в каждом вертикальном сечении открытие эллиптическое.

Напомню, что эллиптическое открытие как раз соответствует постоянному давлению, т.е. для плоской и радиальной трещины в случае постоянного давления ($K$-режим) открытие трещины эллиптическое.
\\

Используя эти основные предположения, мы сможем по сути усреднить уравнения по оси $Oy$ и получить одномерную систему уравнений для рассматриваемой геометрии вдоль оси $Ox$.

Почему одномерную?

Потому что это старая модель (80-х годов) и тогда ещё не было широкодоступных вычислительных мощностей.

Все одномерные модели можно относительно легко имплементировать, их можно относительно легко анализировать и по ним можно многое понять (быстро посчитать), т.е. одномерные модели имеют свои преимущества.

\includegraphics[width=\textwidth, page=59]{HF_slides_2021.pdf}

Перейдём к выводу уравнений модели PKN.

Стартуем с более общей формулировки для планарной трещины: двухмерный закон сохранения жидкости, связь потока и градиента давления, уравнение упругости и условие распространения.

Что делаем далее?

Используя предположения модели PKN, усредняем уравнения и из двухмерной системы уравнений получаем одномерную.
По сути мы аналитически решаем по вертикали (говорим, что в вертикальном сечении открытие эллиптическое -- это как бы аналитическое решение по вертикали).

Если присмотреться к полученной системе уравнений, то она практически совпадает с системой для плоской трещины.
Единственное, что сильно изменяется, это уравнение упругости.

Таким образом, главное отличие модели PKN и плоской трещины -- это уравнение упругости.

\includegraphics[width=\textwidth, page=60]{HF_slides_2021.pdf}

Давайте более внимательно посмотрим на уравнение упругости для модели PKN, в частности на ядро $G$ этого уравнения.

На слайде приведена явная форма для ядра; $E$ -- это эллиптический интеграл, но его не стоит боятся, т.к. он является регулярной гладкой функцией, которая меняется в пределах от $\pi$/2 до 1.

Т.е. основная зависимость зашита в множителе перед этим эллиптическим интегралом.

Важно, что при $\left|s\right|\gg 1$:
\beq
G(s)\approx\frac{\pi}{2}\text{sign}(s)
\eeq
и тогда под интегралом появляется дельта-функция.
Другими словами, от интеграла остаётся только среднее раскрытие $\bar{w}(x)$.
Это означает, что давление в трещине пропорционально среднему раскрытию.

Физически получили, что вдали от кончика трещины давление будет пропорционально среднему раскрытию.
Это называется локальным уравнением упругости.

Нелокальное уравнение упругости соответствует зависимости давления в точке от интеграла от раскрытия по всей трещине.
Локальное уравнение упругости соответствует зависимости давления в точке от локального раскрытия в этой точке.
\\

При $s\ll1$ (вблизи кончика трещины):
\beq
G(s)\approx\frac{1}{s}
\eeq
и тогда получаем ядро плоской трещины.
\\

Таким образом, получили как бы более общее уравнение упругости, которое описывает поведение трещины возле кончика в форме аналогичной плоской трещине, а при удалении от кончика оно плавно переходит в локальное уравнение упругости.

\includegraphics[width=\textwidth, page=61]{HF_slides_2021.pdf}

Есть два варианта поиска решения для модели PKN:

1) более простое решение, которое опирается на локальную упругость;
говорим, что вдали от кончика применимо локальное уравнение упругости, а возле кончика будем использовать специальные граничные условия (чтобы описать поведение возле кончика), но не будем использовать более сложные уравнения упругости возле кончика (итог: локальное уравнение упругости + специальный критерий распространения на кончике);

2) решение исходного интегрального уравнения упругости честно и численно;
используем честное граничное условие на критерий распространения и честное уравнение упругости (итог: полное уравнение упругости + стандартный критерий распространения).

На графиках приведён пример из статьи.
Показаны графики зависимости открытия от координаты $x$: представлены полная модель planar ILSA, модель PKN с нелокальной упругостью и две модели PKN с локальной упругостью (с различными специальными критериями распространения).
\\

Важно, что при низкой трещиностойкости все решения примерно совпадают вдали от кончика и есть различие лишь возле кончика.

Встаёт вопрос: какую задачу мы хотим решить?
Ведь в зависимости от задачи можем использовать разные методы.

Если задача состоит в том, чтобы оценить общую длину трещины и общее раскрытие, то использовать локальную упругость вполне можно.

Но если мы хотим точно описать поведение возле кончика и в принципе иметь более точное решение, то тогда лучше использовать нелокальную упругость и считать всё численно.

\includegraphics[width=\textwidth, page=62]{HF_slides_2021.pdf}

Цель этого и следующих нескольких слайдов -- это сделать анализ трещины PKN по тем же мотивам, что было сделано для плоской трещины и для радиальной трещины.

Можем ли мы построить карту режимов, чтобы понять, при каких параметрах будет режим трещиностойкости?

Для этого легче работать с моделью локальной упругости.

Первое: мы решаем задачу на кончике именно для модели PKN.
Делаем стандартную процедуру: переходим в движущуюся систему координат.
Здесь всё примерно то же самое, что и для модели полубесконечной трещины, но здесь более простые уравнение упругости и граничное условие.
Получаем достаточно простое дифференциальное уравнение с некоторыми параметрами, которое далее можем анализировать и решать.

\includegraphics[width=\textwidth, page=63]{HF_slides_2021.pdf}

Подобно модели полубесконечной трещины для модели кончика PKN получаются 3 предельных решения:
доминирование трещиностойкости, вязкости или утечек.

Но степени в решениях отличаются от полубесконечной и плоской трещин.
\\

Здесь можем пойти и дальше (т.к. уравнение несложное) и посчитать точное решение для K-M режима (учитываем одновременно и трещиностойкость, и вязкость, но утечками пренебрегаем).

Аналогично можем посчитать точное решение при трещиностойкости и утечках (вязкостью пренебрегаем).
\\

Но полностью решить полученное дифференциальное уравнение можем только приближённо.
Первая аппроксимация (менее точная): взять среднюю степень между предельными режимами.
Вторая аппроксимация (более точная): построить более сложное решение, к которому в дальнейшем ещё вернёмся.

\includegraphics[width=\textwidth, page=64]{HF_slides_2021.pdf}

Есть некое размерное уравнение, мы делаем масштабирование и после этого масштабирования получаем более простое дифференциальное уравнение.
Решаем и получаем предельные решения в безразмерной форме.
\\

Важно, что опять же можем построить параметрическое пространство для полубесконечной трещины PKN (в модели локальных деформаций).
Для этого численно решаем полученное дифференциальное уравнение во всём параметрическом пространстве (это делается очень быстро).

\includegraphics[width=\textwidth, page=65]{HF_slides_2021.pdf}

На данном слайде показаны карты ошибок между численным решением во всём параметрическом пространстве и двумя приближёнными аналитическими решениями.

Видно, что более сложное приближённое решение даёт более высокую точность.
\\

Зачем рассматриваем и анализируем полубесконечную трещину модели PKN?
Это необходимо для того, чтобы построить полное решение и параметрическое пространство для конечной трещины PKN.

\includegraphics[width=\textwidth, page=66]{HF_slides_2021.pdf}

Решение для полубесконечной трещины PKN будем использовать в качестве основы, чтобы построить полное решение для конечной трещины.

Сначала давайте построим узловые решения для конечной трещины PKN (аналогично плоской и радиальной трещинам есть 4 предельных решения).
\\

Давайте первым рассмотрим самый сложный предельный режим, а именно режим storage viscosity (нет трещиностойкости и нет утечек).

Ищем решение в следующем виде: решение на кончике, умноженное на некую поправку $f(\xi)$.

Подставляем это решение в закон сохранения объёма и видим, что можем посчитать длину как функцию времени.
\\

Дополнительно подставляем решение в дифференциальное уравнение и раскладываем полученное уравнение по степеням $(1-\xi)$.

Получаем, что в первом приближении $f(\xi)\approx1$.

Это говорит нам, что для трещины PKN мы просто можем взять решение для полубесконечной трещины и это будет достаточно точный ответ для конечной трещины.

\includegraphics[width=\textwidth, page=67]{HF_slides_2021.pdf}

В режиме leak-off viscosity доминирует вязкость и есть большие утечки.

Приравниваем утечки закачанному в трещину объёму.
Предполагаем, что $l$ пропорциональна $t^\alpha$.

Решаем уравнение и получаем, что $\alpha=1/2$.
\\

Решаем исходное дифференциальное уравнение и получаем ответ.
Сама функция выглядит сложно, но можно сделать её разложение и получить некое приближение.

\includegraphics[width=\textwidth, page=68]{HF_slides_2021.pdf}

При доминировании трещиностойкости всё гораздо проще, так как нет градиента давления вдоль трещины (и нет градиента открытия, так как рассматриваем локальную упругость).

Другими словами, раскрытие постоянно и определяется из граничного условия.
\\

В итоге мы получили 4 предельных решения для трещины PKN.

\includegraphics[width=\textwidth, page=69]{HF_slides_2021.pdf}

Полное решение для трещины PKN строим в следующем виде: произведение асимптотики на некую поправку ($\delta$ зависит от режима).

\includegraphics[width=\textwidth, page=70]{HF_slides_2021.pdf}

\includegraphics[width=\textwidth, page=71]{HF_slides_2021.pdf}

\includegraphics[width=\textwidth, page=72]{HF_slides_2021.pdf}

\includegraphics[width=\textwidth, page=51]{HF_slides_2022.pdf}


\end{document}
