\chapter*{Заключение} \label{ch-conclusion}
\addcontentsline{toc}{chapter}{Заключение}

В данной работе проведено моделирование роста нескольких трещин автоГРП с помощью совместного решения уравнений Кирхгофа и Кёнинга.

Проведённый анализ показал, что на скорость роста трещины автоГРП существенно влияет качество перфораций на этой трещине.

На языке Python написан код для моделирования перераспределения потоков между произвольным количеством трещин при любых заранее заданных параметрах закачиваемой ньютоновской жидкости, геолого-физических характеристиках пласта и параметрах перфораций.

Построены графики зависимости полудлины трещин автоГРП от времени, зависимости забойного давления от времени, расходов и чистого давления на каждой из трещин от времени.

Сделан вывод, что для поддержания роста трещин требуется увеличение забойного давления, которое влечёт за собой увеличение чистого давления на каждой из трещин.

Проведено моделирование роста трещин при внезапном ухудшении фильтрационных свойств на одной из них при условии поддержания постоянного расхода на забое.
Сделан вывод, что подобная ситуация может оказать негативное влияние на эффективность эксплуатации месторождения, так как может привести к неконтролируемому росту трещин.

В дальнейшем планируется провести моделирование с автоматическим изменением задаваемого на забое расхода, чтобы предотвратить неконтролируемый рост трещин.
