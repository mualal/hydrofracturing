\chapter*{Введение} % * не проставляет номер
\addcontentsline{toc}{chapter}{Введение} % вносим в содержание


Важным фактором, влияющим на эффективность добычи углеводородов при разработке месторождения, является система заводнения, которая организуется с целью поддержания пластового давления и увеличения нефтеотдачи пласта.
Часто для этого осуществляют перевод добывающих скважин, отработавших на истощение, в нагнетание.
Однако на момент перевода в нагнетание большинство эксплуатационного фонда скважин было предварительно простимулировано многостадийным гидроразрывом пласта.
Поскольку при нагнетании давление жидкости, как правило, превышает давление разрыва породы, возникает риск инициации самопроизвольного роста техногенных трещин.
Данное явление называется эффектом автоГРП, а длина трещины автоГРП может варьироваться от десятков метров до километра и более.

Отрицательные последствия роста трещины автоГРП зависят от её длины, высоты и ориентации и заключаются в том, что развитие трещины может стать причиной обводнения добывающих скважин, а также причиной прорыва воды в верхние или нижние горизонты, что снижает эффективность эксплуатации месторождения.

Поскольку нагнетание жидкости производится с большим расходом, то практически невозможно избежать инициации нескольких трещин автоГРП в начале закачки \cite{baikov_book}.
Поэтому важно научиться моделировать одновременный рост нескольких трещин автоГРП в длину.

\emph{Целью} данной работы является моделирование перераспределения потоков между несколькими трещинами автоГРП.
В качестве базовой модели трещины выбрана модель Перкинса-Керна-Нордгрена, для которой в работе \cite{dontsov2021analysis} представлены приближённые решения во всех предельных случаях.
Также на практике используют модель Христиановича-Желтова-Гиртсма-деКлерка \cite{dontsov1_book} и радиальную модель \cite{dontsov2_book} трещин ГРП, но они менее точны при моделировании роста трещин автоГРП, так как эти модели построены при строгом выполнении условия плоской деформации.

\emph{Объектом исследования} является горизонтальная нагнетательная скважина, на которой ранее (когда она работала в добывающем фонде) был проведён многостадийный гидроразрыв.
\emph{Предметом исследования} являются забойное давление на рассматриваемой скважине и расходы воды на каждой из трещин при заданном расходе воды на забое.

Для достижения поставленной цели предполагается выполнение следующих \emph{задач}:

1) обзор литературы по моделированию роста одной трещины автоГРП;

2) построение физико-математической модели задачи;

3) численная реализация алгоритма решения;

4) анализ зависимости полудлины каждой из трещин автоГРП от времени;

5) анализ зависимости забойного давления от времени;

6) анализ расходов на каждой из трещин в зависимости от времени;

7) анализ давления в каждой из трещин в зависимости от времени.

В данном отчёте представлены постановка задачи и численная реализация алгоритма решения с использованием модели Картера \cite{karter_book} и полной производной формулы Кёнинга \cite{koning_book} по времени.
