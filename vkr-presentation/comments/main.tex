\documentclass[a4paper, 12pt]{article}
\usepackage{comment}
\usepackage{lipsum}
\usepackage{fullpage}
\usepackage[a4paper, total={7in, 10in}]{geometry}
\usepackage{amsmath}
\usepackage[utf8]{inputenc}
\usepackage[russian]{babel}
\usepackage{amssymb,amsthm}

\newtheorem{theorem}{Theorem}
\newtheorem{corollary}{Corollary}
\usepackage{graphicx}
\usepackage{tikz}
\usetikzlibrary{arrows}
\usepackage{verbatim}
\usepackage{xcolor}
\usepackage{mdframed}
\usepackage[shortlabels]{enumitem}
\usepackage{indentfirst}
\setlength{\parindent}{0cm}
\usepackage{hyperref}
\usepackage{float}

\usepackage{setspace}
\setlength{\parindent}{20pt}
\setlength{\parskip}{4pt}

\graphicspath{{./images/}}

\newcommand{\beq}{\begin{equation}}
\newcommand{\eeq}{\end{equation}}
\begin{document}

\setstretch{1.2}

\textbf{Слайд 1.}

Тема моей выпускной квалификационной работы: "<Моделирование перераспределения потоков между трещинами гидроразрыва пласта">.\\

\textbf{Слайд 2.}

При эксплуатации месторождения перевод добывающих скважин в нагнетание может спровоцировать рост техногенных трещин автоГРП.
Если ранее на этой скважине уже был проведён многостадийный гидроразрыв пласта, то тогда часто одновременно растут несколько трещин автоГРП: по одной из каждого порта, на которых ранее был проведён гидроразыв.

Такой неконтролируемый рост нескольких трещин автоГРП может привести к существенному снижению эффективности эксплуатации месторождения, если трещина автоГРП прорвётся к добывающей скважине.

С другой стороны, при грамотно контролируемом росте трещин автоГРП можно наоборот увеличить площадь охвата заводнением и повысить эффективность эксплуатации месторождения.

Поэтому важно научиться моделировать одновременный рост нескольких трещин автоГРП.\\ 

\textbf{Слайд 3.}

Целью моей работы является построение модели совместного роста нескольких трещин автоГРП.

Для достижения поставленной цели проводится обзор имеющихся моделей роста трещины гидроразрыва;
осуществляется выбор модели, которая наиболее подходит для моделирования роста трещин автоГРП;
строится физико-математическая модель роста нескольких трещин автоГРП и численный алгоритм решения;
проводится анализ результатов.\\

\textbf{Слайд 4.}

Любая полная модель трещины ГРП состоит из пяти основных компонентов: баланса объёма жидкости, уравнения течения жидкости в трещине, уравнение упругости для горной породы, условие распространения трещины и транспорт проппанта.

Для трещин автоГРП не рассматриваем транспорт проппанта (так как закачиваем просто воду) и уравнение течения жидкости (так как пренебрегаем вязкостью воды).

В качестве закона сохранения используется закон сохранения объёма, так как вводится предположение, что жидкость несжимаема.

Уравнение упругости связывает давление в трещине и раскрытие этой трещины.
Обычно записывается в глобальной форме, т.е. любое локальное изменение открытия меняет давление глобально во всей трещине.

Условие распространения трещин задаётся с помощью классического результата Механики Линейно-Упругого Разрушения (LEFM).
На самом деле вблизи кончика есть нелинейные процессы и даже зона пластичности.
Но мы предполагаем, что эта зона мала по сравнению с размером трещины и можем использовать результаты Механики Линейно-Упругого Разрушения, т.е. напряжение $\sigma=\frac{K_I}{\sqrt{2\pi r}}$ и раскрытие вблизи кончика $w=\sqrt{\frac{32}{\pi}}\frac{K_{I}\left(1-\nu^2\right)}{E}\sqrt{r}$, где $K_I$ -- коэффициент интенсивности напряжений.
Когда раскрываем трещину, растёт коэффициент интенсивности напряжения.
Когда он превышает некое значение $K_{Ic}$ (называемое трещиностойкостью породы), трещина распространяется.\\

\textbf{Слайд 5.}

Исторически первой была получена классическая модель Христиановича для трещины ГРП.
В этой модели предполагается, что высота трещины много больше её длины, у трещины прямоугольное вертикальное сечение и верно допущение плоской деформации в горизонтальной плоскости.
Т.е. поведение трещины одинаково во всех горизонтальных сечениях.
Таким образом, задача сводится к одномерной.

Первое уравнение -- закон баланса объёма жидкости.
Первое слагаемое отвечает за изменение объёма трещины вследствие изменения раскрытия, второе слагаемое -- за счёт изменения расхода вдоль трещины и третье слагаемое позволяет учесть утечки из трещины в пласт по модели Картера.
$t_0(x)$ -- это время, за которое фронт трещины достиг координаты $x$.
В правой части равенства источниковое слагаемое.

Второе уравнение -- уравнение движения ньютоновской жидкости вдоль трещины.
Получено при условии прилипания на стенках трещины.
\beq
\begin{gathered}
v=v_x(y);\\
\frac{\partial p}{\partial x}=\frac{\partial\tau}{\partial y}\text{ (уравнение движения)};\\
\tau=\mu\frac{\partial v}{\partial y}\text{ (для ньютоновской жидкости)};\\
v|_{y=\pm w/2}=0\text{ (условие прилипания)}\\
\text{Общее решение: }v=\frac{\partial p}{\partial x}\frac{y^2}{2}+Ay+B.\text{ При заданном условии: }v=-\frac{\partial p}{\partial x}\frac{w^2-4y^2}{8\mu}.\\
\text{Суммарный поток: }q=\int_{-w/2}^{w/2}{v(y)}dy=-\frac{w^3}{12\mu}\frac{\partial p}{\partial x}.
\end{gathered}
\eeq


Третье уравнение -- уравнение упругости.
Интегральная связь между давлением в трещине и раскрытием.
Локальное раскрытие влияет глобально на давление во всей трещине.
$\sigma_0$ -- сжимающие напряжения, действующие на трещину снаружи (со стороны породы).

Четвёртое уравнение -- условие распространения.
Условие в виде предела на кончике.
На основе результатов Механики Линейно-Упругого разрушения (LEFM) корневая зависимость раскрытия от расстояния на кончике.

Данная модель не может быть применена для моделирования роста трещин автоГРП, так как длина трещин автоГРП обычно больше высоты, а в основном предположении модели Христиановича наоборот длина много меньше высоты.\\

\textbf{Слайд 6.}

В модели радиальной трещины уравнения похожи на уравнения модели Христиановича.
Основное отличие в том, что теперь рассматривается другая геометрия в цилиндрической системе координат.
В данном случае геометрия осесимметричная.
Такого рода геометрии могут быть при закачке жидкости из точечного перфорационного интервала и при неограниченном по всем направлениям однородном пласте.

В реальности же высота пласта ограничена, поэтому модель с такой осесимметричной геометрией не применима для трещин автоГРП, у которых длина трещины много больше высоты.

На слайде в уравнениях полные эллиптические интегралы первого и второго рода:
$$K(m)=\int\limits_{0}^{\pi/2}{\frac{d\varphi}{\sqrt{1-m^2\sin^2{\!\varphi}}}}\text{ и } E(m)=\int\limits_{0}^{\pi/2}{\sqrt{1-m^2\sin^2{\!\varphi}}\,d\varphi}\text{ соответственно.}$$\\

\textbf{Слайд 7.}

В следующей модели, называемой моделью Перкинса-Керна-Нордгрена (или моделью трещины постоянной высоты), вводятся два основных предположения: первое предположение -- то, что длина трещины много больше фиксированной высоты $H$; второе предположение -- то, что в любом вертикальном сечении давление постоянно.

Из постоянства давления в вертикальном сечении следует эллиптичность профиля трещины, что позволяет перейти от двумерной системы уравнений к одномерной, которая представлена на слайде.
Этот переход осуществляется как бы аналитическим решением вдоль направления оси $Oz$ путём усреденения.
В представленных на слайде уравнениях раскрытие $w$ и поток $q$ усреденены по высоте трещины.
Оператор усреденения представлен в рамке на примере раскрытия.

В данной работе в качестве базовой выбрана модель Перкинса-Керна-Нордгрена для моделирования роста трещин автоГРП, так как основные предположения этой модели соответствуют поведению трещин автоГРП (длина трещины много больше постоянной высоты $H$).

%Также многими исследователями отмечалось, что результаты решения модели PKN более адекватны.
%Например, для случая доминирования трещиностойкости и больших утечек по модели PKN давление увеличивается во времени, а по моделям Христиановича и радиальной -- уменьшается.
%Из предположений модели KGD и радиальной вытекает, что когда размеры трещины становятся очень большими, требуются очень малые эффективные давления для поддержания определённой ширины.
%Хотя это явление является следствием теории линейной упругости и способа применения условия плоской деформации, в целом это приводит к абсурдным выводам.
%Можно с уверенностью сказать, что модель PKN лучше описывает физику процесса гидроразрыва, чем две другие модели.

Также модель может быть расширена до так называемой улучшенной модели Перкинса-Керна-Нордгрена, когда трещина не ограничена границами пласта, а может распространяться в близлежащие слои породы.
Дополнительно есть расширения модели PKN, которые учитывают степенные жидкости, жидкости Гершеля-Балкли, жидкости Карро, эффект отставания фронта жидкости от фронта трещины, эффект турбулентного течения, эффект утечки зависящей от давления и т.д.

В данной работе буду использовать модель PKN без дополнительных расширений.
\\


\textbf{Слайд 8.}

В работе Егора Владимировича Донцова, название которой представлены на слайде 

\textbf{Слайд 9.}

\textbf{Слайд 10.}

Для моделирования перераспределения потоков между трещинами автоГРП я использую законы Кирхгофа, в которых учитывается падение давления на трение и падение давления на перфорациях.

\textbf{Слайд 11.}

\textbf{Слайд 12.}

\textbf{Слайд 13.}

\textbf{Слайд 14.}

\textbf{Слайд 15.}

По вектору невязок и вектору расходов составляется матрица Якоби.
И итеративная процедура решения с помощью метода Ньютона.

В качестве начального приближения закачиваемый расход перераспределяется по трещинам одинаково, а забойное давление равно сжимающему напряжению, действующему на трещину со стороны породы.
Условие остановки представлено на слайде.


\textbf{Слайд 16.}

\textbf{Слайд 17.}

\textbf{Слайд 18.}

\textbf{Слайд 19.}

\textbf{Слайд 20.}

\textbf{Слайд 21.}


\end{document}
