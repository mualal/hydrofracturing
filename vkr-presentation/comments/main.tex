\documentclass[a4paper, 12pt]{article}
\usepackage{comment}
\usepackage{lipsum}
\usepackage{fullpage}
\usepackage[a4paper, total={7in, 10in}]{geometry}
\usepackage{amsmath}
\usepackage[utf8]{inputenc}
\usepackage[russian]{babel}
\usepackage{amssymb,amsthm}

\newtheorem{theorem}{Theorem}
\newtheorem{corollary}{Corollary}
\usepackage{graphicx}
\usepackage{tikz}
\usetikzlibrary{arrows}
\usepackage{verbatim}
\usepackage{xcolor}
\usepackage{mdframed}
\usepackage[shortlabels]{enumitem}
\usepackage{indentfirst}
\setlength{\parindent}{0cm}
\usepackage{hyperref}
\usepackage{float}

\usepackage{setspace}
\setlength{\parindent}{20pt}
\setlength{\parskip}{4pt}

\graphicspath{{./images/}}

\newcommand{\beq}{\begin{equation}}
\newcommand{\eeq}{\end{equation}}
\begin{document}

\setstretch{1.2}

\textbf{Слайд 1 (Тема работы).}

Тема моей выпускной квалификационной работы: "<Моделирование перераспределения потоков между трещинами гидроразрыва пласта">.
\\


\textbf{Слайд 2 (Проблематика и актуальность работы).}

При эксплуатации месторождения перевод добывающих скважин в нагнетание может спровоцировать рост техногенных трещин автоГРП.

Если ранее на этой скважине уже был проведён многостадийный гидроразрыв пласта, то тогда при переводе в нагнетание часто происходит одновременный рост нескольких трещин автоГРП: по одной из каждого порта ранее проведённого многостадийного гидроразыва.

Такой неконтролируемый рост нескольких трещин автоГРП может привести к существенному снижению эффективности эксплуатации месторождения, если трещина автоГРП прорвётся к добывающей скважине.

С другой стороны, при грамотно контролируемом росте трещин автоГРП можно наоборот увеличить площадь охвата заводнением и повысить эффективность эксплуатации месторождения.

Поэтому важно научиться моделировать одновременный рост нескольких трещин автоГРП в длину.
\\ 


\textbf{Слайд 3 (Цель и задачи работы).}

Целью моей работы является построение модели совместного роста нескольких трещин автоГРП с учётом перераспределения потоков между ними.

Для достижения поставленной цели проводится обзор имеющихся моделей роста трещины гидроразрыва;
осуществляется выбор модели, которая наиболее подходит для моделирования роста трещин автоГРП;
строится физико-математическая модель роста нескольких трещин автоГРП и численный алгоритм решения;
проводится анализ результатов.
\\


\textbf{Слайд 4 (Основные компоненты полной модели трещины ГРП).}

Любая полная модель трещины ГРП состоит из пяти основных компонентов: баланса массы для жидкости, уравнения течения жидкости в трещине, уравнения упругости для горной породы, условия распространения трещины и транспорта проппанта.

Для трещин автоГРП не рассматриваем транспорт проппанта (так как закачиваем просто воду) и уравнение течения жидкости (так как пренебрегаем вязкостью воды).

Уравнение упругости связывает давление в трещине и раскрытие этой трещины.
Обычно записывается в глобальной форме, т.е. любое локальное изменение раскрытия меняет давление глобально во всей трещине.

Условие распространения трещин задаётся с помощью классического результата Механики Хрупкого Разрушения (LEFM), т.е. предполагается, что материал сохраняет свойство линейной упругости вплоть до разрушения.

На самом деле вблизи кончика есть нелинейные процессы и зона пластичности, но размер этой зоны пренебрежимо мал по сравнению с длиной трещины и поэтому вводится коэффициент интенсивности напряжений $K_I$ вблизи кончика трещины.
Когда раскрываем трещину, растёт коэффициент интенсивности напряжений.
Когда он превышает некое значение $K_{Ic}$ (называемое трещиностойкостью породы), трещина распространяется.
\\


\textbf{Слайд 5 (Модель Перкинса-Керна-Нордгрена (модель PKN)).}

В ходе работы был проведён обзор моделей трещины гидроразрыва пласта.
Рассмотрены модель Христановича-Желтова-Гиртсма-деКлерка, модель радиальной трещины и модель Перкинса-Керна-Нордгрена (также называемой моделью PKN или моделью трещины постоянной высоты).

Для моделирования трещин автоГРП была выбрана модель PKN, так как вводимые в ней предположения соответствуют процессу распространения трещин автоГРП.
В модели PKN вводятся два основных предположения: первое предположение -- то, что длина трещины много больше фиксированной высоты $H$; второе предположение -- то, что в любом вертикальном сечении давление постоянно.

Из постоянства давления в вертикальном сечении следует эллиптичность профиля трещины, что позволяет уменьшить размерность уравнений относительно более общих моделей (например, модели planar3D).
Уменьшение размерности осуществляется как бы аналитическим решением вдоль направления оси $Oz$ путём усреденения.

На слайде для модели PKN записаны закон сохранения объёма жидкости, уравнение течения жидкости в пласкопараллельном канале, уравнение упругости и условие распространения трещины из механики хрупкого разрушения.

На слайде в уравнениях полные эллиптические интегралы первого и второго рода:
$$K(m)=\int\limits_{0}^{\pi/2}{\frac{d\varphi}{\sqrt{1-m^2\sin^2{\!\varphi}}}}\text{ и } E(m)=\int\limits_{0}^{\pi/2}{\sqrt{1-m^2\sin^2{\!\varphi}}\,d\varphi}\text{ соответственно.}$$
\\


\textbf{Слайд 6 (Схема перераспределния потоков между трещинами гидроразрыва и правила Кирхгофа).}

При одновременном росте нескольких трещин автоГРП потоки между ними могут перераспределяться (например, из-за ухудшения свойств перфораций на одной из трещин).

В решаемой задаче я считаю, что закачиваемый в скважину расход фиксирован, а забойное давление и расход на каждой из трещин пересчитываю по правилам Кирхгофа.

Согласно первому правилу: весь закачиваемый расход равен сумме расходов по всем трещинам.

Согласно второму правилу: путь к каждой из трещин можно рассматривать независимо, т.е. забойное давление есть сумма давления в трещине, падения давления на перфорациях и падения давления на трение в трубе.
Слагаемое с гидростатическим давлением я использовать не буду, так как считаю, что все трещины распространяются от горизонтальной скважины в одном пласте на одной глубине.

Таким образом, необходимы ещё три замыкающих соотношения: для чистого давления в трещинах, для падения давления на перфорациях и для падения давления на трение.
\\


\textbf{Слайд 7 (Замыкающие соотношения).}

Известно, что при распространении трещин PKN в однородном пласте чистое давление в них не изменяется со временем.
Выражение для чистого давления в трещинах PKN представлено на слайде. Чистое давление в трещине зависит от трещиностойкости породы и высоты распространяющейся трещины.

Далее представлена эмпирическая формула для падения давления на перфорациях.
Падение давления на перфорациях зависит от плотности закачиваемой жидкости, числа перфораций, диаметра перфораций и коэффициента эрозии (изнашивание перфораций при долгой эксплуатации).
В случае закачки воды (жидкости без твёрдых частиц в потоке) известно, что коэффициент эрозии $C_d=0.5$.

Также представлена формула для падения давления на трение на горизонтальном участке скважины при ламинарном режиме течения.
В этом случае есть явное выражение для коэффициента трения Фаннинга $f_s=\frac{16}{Re}$, что позволяет представить формулу для падения давления трение в таком виде, в котором явно не участвует напряжение сдвига на стенке трубы.
\\


\textbf{Слайд 8 (Замкнутая постановка задачи).}

Здесь объединены правила Кирхгофа и замыкающие соотношения.
Получили замкнутую системы из $N+1$ уравнения с $N+1$ неизвестным.
\\


\textbf{Слайд 9 (Вектор невязок).}

Далее для полученной системы уравнений составляется вектор невязок $F$ и ставится задача минимизации невязок.
\\


\textbf{Слайд 10 (Итеративная процедура решения).}

Составляется матрица Якоби и с помощью метода Ньютона итеративно ищутся потоки на каждой из трещин и забойное давление.

В качестве начального приближения закачиваемый расход перераспределяется по трещинам одинаково, а забойное давление равно сжимающему напряжению, действующему на трещину со стороны породы.

Условие остановки представлено на слайде.
\\


\textbf{Слайд 11 (Итеративный процесс поиска расходов на трещинах и забойного давления).}

На этом слайде представлены примеры графиков итеративного процесса поиска решения.

В данном случае трещины отличаются друг от друга только своим положением в пространстве.
Расходы распределились между трещинами неодинаково из-за потерь давления на трение в трубе.
\\


\textbf{Слайд 12 (Распределение давления вдоль горизонтального участка скважины).}

На этом слайде синей линией представлено установившееся распределение давления вдоль горизонтального участка скважины.
Синими точками отмечены положения забоя и трещин.

Оранжевой горизонтальной прямой показано давление в трещинах.
Это давление одинаково во всех распространяющихся трещинах (так как у всех трещин одинаковая высота и они распространяются в одном пласте с фиксированным значением трещиностойкости породы).

Видим, что падение давления на трение от забоя до первой трещины относительно небольшое, но падение давления на перфорациях первой трещины высокое (так как на ней получился самый высокий расход жидкости, а падение давления на перфорациях как раз пропорционально квадрату расхода).

Для последней трещины наоборот: падение давления на трение от забоя до последней трещины высокое, а падение давления на перфорациях относительно небольшое (так как на этой трещине получился самый маленький расход жидкости).
\\


\textbf{Слайд 13 (Формулы Кёнинга).}

Теперь на основе найденных расходов, давления в трещине и времени необходимо найти приращение длины трещины автоГРП.

В работе Кёнинга получены формулы зависимости полудлины трещины гидроразрыва от расхода жидкости, фильтрационно-ёмкостных свойств пласта, репрессии на пласт и времени при условии доминирования утечек жидкости из трещины в пласт.

Представлено две формулы: первая -- для случая одномерных утечек, перпендикулярных трещине; вторая -- для случая двумерных радиальных утечек жидкости из трещины в пласт.

В этих формулах явная зависимость полудлины трещины от времени корневая.

Однако в задаче моделирования совместного роста нескольких трещин автоГРП поток и репрессия на пласт тоже меняются со временем, поэтому дополнительно в формуле Кёнинга есть неявная зависимость полудлины трещины от времени.

Для учёта этой неявной зависимости всё время моделирования разделяется на небольшие интервалы, на каждом интервале пересчитываются потоки и репрессии на пласт и вычисляются приращения полудлин трещин на этом шаге по времени.
\\


\textbf{Слайд 14 (Полная производная полудлины трещин по времени).}

Здесь представлены выражения для полных производных полудлин трещин по времени в случае одномерных утечек Картера и в случае двумерных радиальных утечек жидкости из трещины в пласт.
\\


\textbf{Слайд 15 (Приращение полудлины трещин).}

Затем получены выражения для приращений полудлин трещин, которые будут использоваться в алгоритме.
\\


\textbf{Слайд 16 (Алгоритм расчёта полудлин трещин в зависимости от времени).}

Здесь представлена блок-схема алгоритма, по которому проведено совмещение решателя уравнений Кирхгофа с формулами Кёнинга.

Задаются начальные значения длин трещин и значения входных параметров задачи.
С помощью решателя уравнений Кирхгофа рассчитываются расходы на каждой из трещин и забойное давление в текущий момент времени.

Далее на основе приращений времени, расходов и репрессии рассчитывается приращение полудлин трещин.
Обновляются значения полудлин трещин и времени.
При необходимости изменяются значения входных параметров задачи (например, если ухудшилось качество перфораций или изменился расход на забое).
Расчёт продолжается до заранее заданного момента времени.
\\

\textbf{Слайд 17 (Выбранные значения входных параметров).}

На этом слайде представлены выбранные значения входных параметров алгоритма.
Пласт залегает на глубине примерно 2.5 километра, поэтому пластовое давление 250 атмосфер, а давление смыкания трещин (минимальное горизонтальное напряжение в пласте) -- 400 атмосфер.
\\


\textbf{Слайд 18 (Результат совместного использования формулы Кёнинга с решателем уравнений Кирхгофа).}

Представлены графики зависимости полудлин трещин от времени, зависимость забойного давления от времени, зависимости расходов воды и чистого давления на каждой из трещин от времени.

Трещины отличаются друг от друга только своим положением в пространстве и растут по разному из-за разных расходов на трещинах вследствие падения давления на трение в трубе.

Также в синей рамке представлены результаты расчёта, когда падение давления на трение в трубе не учитывается и все трещины растут одинаково.

Видим, что важно учитывать падение давления на трение в трубе, так как оно приводит к большему расходу на первой трещине и её более интенсивному росту, что повышает риски прорыва этой трещины к добывающей скважине.
\\


\textbf{Слайд 19 (Сравнение роста трещин при одномерных утечках Картера и при двумерных радиальных утечках жидкости).}

На данном слайде представлено сравнение роста трещин при одномерных утечках Картера и при двумерных радиальных утечках жидкости из трещины в пласт.
При этом задан периодически меняющийся во времени расход на забое скважины.

Видим, что при одномерных утечках Картера трещина растёт более чем в 2 раза быстрее.
\\


\textbf{Слайд 20 (Результаты при линейном уменьшении расхода жидкости на забое скважины).}

Если же будем линейно уменьшать расход жидкости на забое скважины, то трещины со временем начнут закрываться.
\\


\textbf{Слайд 21 (Результаты при ухудшении качества перфораций на одной из трещин).}

При уменьшении диаметра перфораций на одной из трещин эта трещина постепенно будет закрываться и при этом соседние трещины будут расти более интенсивно.
\\


\textbf{Слайд 22 (Результаты при уменьшении горизонатльных напряжений в пласте).}

При уменьшении горизонтальных напряжений в пласте (например, за счёт закачки холодной воды) трещины растут более интенсивно.
\\


\textbf{Слайд 23 (Результаты при одновременном росте шести трещин).}

На данном слайде проверена работоспособность алгоритма при большем количестве трещин.

На третьей трещине уменьшался диаметр перфораций.
\\


\textbf{Слайд 24 (Выводы).}

Итак, в данной работе проведён обзор моделей трещины гидроразрыва пласта, реализован численный алгоритм расчёта потоков на каждой из трещин по правилам Кирхгофа, проведено совмещение формулы Кёнинга с решателем уравнений Кирхгофа (а именно реализован алгоритм расчёта приращения полудлины трещины на каждом шаге по времени с учётом изменяющихся расходов и входных параметров алгоритма со временем).

Анализ результатов показал, что предположение одномерности утечек жидкости из трещины в пласт по Картеру может завышать значения полудлин растущих трещин автоГРП; падение давления на трение в трубе приводит к существенной разнице расходов на нескольких трещинах автоГРП и, как следствие, трещина, расположенная близко к забою, растёт более интенсивно; уменьшение диаметра перфораций на одной из трещин приводит к постепенному закрытию этой трещины и одновременному более интенсивному росту соседних трещин; уменьшение расхода на забое скважины приводит к уменьшению расходов на трещинах и сокращению их длины; термоупругое уменьшение горизонтальных напряжений в пласте (например, при охлаждение породы в случае закачки холодной воды) приводит к более интенсивному росту трещин автоГРП.

В дальнейшем необходимо дополнить построенную модель: например, учесть эффекты пороупругости, когда большие утечки из трещин влияют на упругое состояние породы и тем самым влияют на соседние трещины.
В этом случае может происходить эффект закрытия трещин автоГРП.












\newpage
\ 
\newpage

















\textbf{\large Текст с предзащиты 19.05.2023 (к соответствующей версии презентации 18.05.2023).}\\

\textbf{Слайд 1 (тема работы).}

Тема моей выпускной квалификационной работы: "<Моделирование перераспределения потоков между трещинами гидроразрыва пласта">.\\

\textbf{Слайд 2 (проблематика и актуальность работы).}

При эксплуатации месторождения перевод добывающих скважин в нагнетание может спровоцировать рост техногенных трещин автоГРП.
Если ранее на этой скважине уже был проведён многостадийный гидроразрыв пласта, то тогда при переводе в нагнетание часто происходит одновременный рост нескольких трещин автоГРП: по одной из каждого порта, на которых ранее был проведён гидроразыв.

Такой неконтролируемый рост нескольких трещин автоГРП может привести к существенному снижению эффективности эксплуатации месторождения, если трещина автоГРП прорвётся к добывающей скважине.

С другой стороны, при грамотно контролируемом росте трещин автоГРП можно наоборот увеличить площадь охвата заводнением и повысить эффективность эксплуатации месторождения.

Поэтому важно научиться моделировать одновременный рост нескольких трещин автоГРП.\\ 

\textbf{Слайд 3 (цель и задачи работы).}

Целью моей работы является построение модели совместного роста нескольких трещин автоГРП.

Для достижения поставленной цели проводится обзор имеющихся моделей роста трещины гидроразрыва;
осуществляется выбор модели, которая наиболее подходит для моделирования роста трещин автоГРП;
строится физико-математическая модель роста нескольких трещин автоГРП и численный алгоритм решения;
проводится анализ результатов.\\

\textbf{Слайд 4 (основные компоненты полной модели трещины ГРП).}

Любая полная модель трещины ГРП состоит из пяти основных компонентов: баланса объёма жидкости, уравнения течения жидкости в трещине, уравнение упругости для горной породы, условие распространения трещины и транспорт проппанта.

Для трещин автоГРП не рассматриваем транспорт проппанта (так как закачиваем просто воду) и уравнение течения жидкости (так как пренебрегаем вязкостью воды).

В качестве закона сохранения используется закон сохранения объёма, так как вводится предположение, что жидкость несжимаема.

Уравнение упругости связывает давление в трещине и раскрытие этой трещины.
Обычно записывается в глобальной форме, т.е. любое локальное изменение открытия меняет давление глобально во всей трещине.

Условие распространения трещин задаётся с помощью классического результата Механики Линейно-Упругого Разрушения (LEFM).
На самом деле вблизи кончика есть нелинейные процессы и даже зона пластичности.
Но мы предполагаем, что эта зона мала по сравнению с размером трещины и можем использовать результаты Механики Линейно-Упругого Разрушения, т.е. напряжение $\sigma=\frac{K_I}{\sqrt{2\pi r}}$ и раскрытие вблизи кончика $w=\sqrt{\frac{32}{\pi}}\frac{K_{I}\left(1-\nu^2\right)}{E}\sqrt{r}$, где $K_I$ -- коэффициент интенсивности напряжений.
Когда раскрываем трещину, растёт коэффициент интенсивности напряжения.
Когда он превышает некое значение $K_{Ic}$ (называемое трещиностойкостью породы), трещина распространяется.\\

\textbf{Слайд 5 (модель KGD = модель плоской трещины).}

Исторически первой была получена классическая модель Христиановича для трещины ГРП.
В этой модели предполагается, что высота трещины много больше её длины, у трещины прямоугольное вертикальное сечение и верно допущение плоской деформации в горизонтальной плоскости.
Т.е. поведение трещины одинаково во всех горизонтальных сечениях.
Таким образом, задача сводится к одномерной.

Первое уравнение -- закон баланса объёма жидкости.
Первое слагаемое отвечает за изменение объёма трещины вследствие изменения раскрытия, второе слагаемое -- за счёт изменения расхода вдоль трещины и третье слагаемое позволяет учесть утечки из трещины в пласт по модели Картера.
$t_0(x)$ -- это время, за которое фронт трещины достиг координаты $x$.
В правой части равенства источниковое слагаемое.

Второе уравнение получено из уравнения движения вязкой ньютоновской жидкости в случае ламинарного режима течения.
Получено при условии прилипания на стенках трещины.
Связывает суммарный поток с градиентом давления.
\beq
\begin{gathered}
v=v_x(y);\\
\frac{\partial p}{\partial x}=\frac{\partial\tau}{\partial y}\text{ (из уравнения Навье-Стокса)};\\
\tau=\mu\frac{\partial v}{\partial y}\text{ (для ньютоновской жидкости)};\\
v|_{y=\pm w/2}=0\text{ (условие прилипания)}\\
\text{Общее решение: }v=\frac{\partial p}{\partial x}\frac{y^2}{2}+Ay+B.\text{ При заданном условии: }v=-\frac{\partial p}{\partial x}\frac{w^2-4y^2}{8\mu}.\\
\text{Суммарный поток: }q=\int_{-w/2}^{w/2}{v(y)}dy=-\frac{w^3}{12\mu}\frac{\partial p}{\partial x}.
\end{gathered}
\eeq


Третье уравнение -- уравнение упругости.
Интегральная связь между давлением в трещине и раскрытием.
Локальное раскрытие влияет глобально на давление во всей трещине.
$\sigma_0$ -- сжимающие напряжения, действующие на трещину снаружи (со стороны породы).

Четвёртое уравнение -- условие распространения.
Условие в виде предела на кончике.
На основе результатов Механики Линейно-Упругого разрушения (LEFM) корневая зависимость раскрытия от расстояния на кончике.

Данная модель не может быть применена для моделирования роста трещин автоГРП, так как длина трещин автоГРП обычно больше высоты, а в основном предположении модели Христиановича наоборот длина много меньше высоты.\\

\textbf{Слайд 6 (модель радиальной трещины).}

В модели радиальной трещины уравнения похожи на уравнения модели Христиановича.
Основное отличие в том, что теперь рассматривается другая геометрия в цилиндрической системе координат.
В данном случае геометрия осесимметричная.
Такого рода геометрии могут быть при закачке жидкости из точечного перфорационного интервала и при неограниченном по всем направлениям однородном пласте.

В реальности же высота пласта ограничена, поэтому модель с такой осесимметричной геометрией не применима для трещин автоГРП, у которых длина трещины много больше высоты.

На слайде в уравнениях полные эллиптические интегралы первого и второго рода:
$$K(m)=\int\limits_{0}^{\pi/2}{\frac{d\varphi}{\sqrt{1-m^2\sin^2{\!\varphi}}}}\text{ и } E(m)=\int\limits_{0}^{\pi/2}{\sqrt{1-m^2\sin^2{\!\varphi}}\,d\varphi}\text{ соответственно.}$$\\

\textbf{Слайд 7 (модель PKN).}

В следующей модели, называемой моделью Перкинса-Керна-Нордгрена (или моделью трещины постоянной высоты), вводятся два основных предположения: первое предположение -- то, что длина трещины много больше фиксированной высоты $H$; второе предположение -- то, что в любом вертикальном сечении давление постоянно.

Из постоянства давления в вертикальном сечении следует эллиптичность профиля трещины, что позволяет перейти от двумерной системы уравнений к одномерной, которая представлена на слайде.
Этот переход осуществляется как бы аналитическим решением вдоль направления оси $Oz$ путём усреденения.
В представленных на слайде уравнениях раскрытие $w$ и поток $q$ усреденены по высоте трещины.
Оператор усреденения представлен в рамке на примере раскрытия.

В данной работе в качестве базовой выбрана модель Перкинса-Керна-Нордгрена для моделирования роста трещин автоГРП, так как основные предположения этой модели соответствуют поведению трещин автоГРП (длина трещины много больше постоянной высоты $H$).

%Также многими исследователями отмечалось, что результаты решения модели PKN более адекватны.
%Например, для случая доминирования трещиностойкости и больших утечек по модели PKN давление увеличивается во времени, а по моделям Христиановича и радиальной -- уменьшается.
%Из предположений модели KGD и радиальной вытекает, что когда размеры трещины становятся очень большими, требуются очень малые эффективные давления для поддержания определённой ширины.
%Хотя это явление является следствием теории линейной упругости и способа применения условия плоской деформации, в целом это приводит к абсурдным выводам.
%Можно с уверенностью сказать, что модель PKN лучше описывает физику процесса гидроразрыва, чем две другие модели.

Также модель может быть расширена до так называемой улучшенной модели Перкинса-Керна-Нордгрена, когда трещина не ограничена границами пласта, а может распространяться в близлежащие слои породы.
Дополнительно есть расширения модели PKN, которые учитывают степенные жидкости, жидкости Гершеля-Балкли, жидкости Карро, эффект отставания фронта жидкости от фронта трещины, эффект турбулентного течения, эффект утечки зависящей от давления и т.д.

В данной работе буду использовать модель PKN без дополнительных расширений.
\\


\textbf{Слайд 8 (параметрическая карта решения модели PKN).}

В работе Егора Владимировича Донцова, название которой представлено на слайде, получена параметрическая карта решения модели PKN.
Полное решение получено в безразмерном виде приближённо полуаналитическим способом с помощью совмещения асимптотического решения на кончике и глобального баланса объёма.
В статье проведено сравнение этого полуаналитического решения с более точным численным решением (ошибка составила менее 1\%).

На карте цветом показаны области предельных решений в различных режимах распространения.
При моделировании трещин автоГРП реализуется $\tilde{K}$-режим, т.е. режим доминирования трещиностойкости и больших утечек.
Для этого режима решение для среднего раскрытия трещины представлено на слайде:
$$
\bar{w}_{\tilde{K}}=\frac{K_{Ic}\sqrt{\pi H}}{E'}
$$\\


\textbf{Слайд 9 (схема перераспределения потоков между трещинами гидроразрыва и правила Кирхгофа).}

При одновременном росте нескольких трещин автоГРП потоки между ними могут перераспределяться (например, из-за ухудшения свойств перфораций на одной из трещин).

В решаемой задаче я считаю, что закачиваемый в скважину расход фиксирован, а забойное давление и расход на каждой из трещин пересчитываю по правилам Кирхгофа.

Согласно первому правилу: весь закачиваемый расход равен сумме расходов по всем трещинам.

Согласно второму правилу: путь к каждой из трещин можно рассматривать независимо, т.е. забойное давление есть сумма давления в трещине, падения давления на перфорациях и падения давления на трение в трубе.
Слагаемое с гидростатическим давлением я использовать не буду, так как считаю, что все трещины распространяются от горизонтальной скважины в одном пласте на одной глубине.\\

\textbf{Слайд 10 (чистое давление в трещинах PKN).}

На этом слайде представлена формула для чистого давления в трещине и формула для объёма трещины при её распространении в режиме доминирования трещиностойкости и больших утечках.

Формула для чистого давления получена для жидкости с произвольной степенной реологией.
Я использую частный случай для ньютоновской жидкости.
Саму формулу взял из лекции Новосибирского государственного университета.
Источник не был указан.\\

\textbf{Слайд 11 (падение давления на перфорациях).}

На этом слайде представлена эмпирическая формула для падения давления на перфорациях.
Падение давления на перфорациях зависит от плотности смеси, числа перфораций, диаметра перфораций и коэффициента эрозии (изнашивание перфораций при долгой эксплуатации).\\

\textbf{Слайд 12 (падение давления на трение).}

На этом слайде представлена эмпирическая формула для падения давления на трение.
В формуле используется коэффициент трения Фаннинга, который является показателем сопротивления потоку жидкости на стенке трубы.

Здесь важно смотреть, сколько жидкости протекает на конкретном участке трубы (общий расход минус расходы, еоторые зашли в предыдущие порты).\\

\textbf{Слайд 13 (запись правил Кирхгофа в векторной форме).}

Далее правила Кирхгофа записаны в векторной форме.
Составлен вектор невязок $F$.

Стоит задача минимизации невязок.\\

\textbf{Слайд 14 (итеративная процедура решения).}

Составляется матрица Якоби и с помощью метода Ньютона итеративно ищутся потоки на каждой из трещин и забойное давление.

В качестве начального приближения закачиваемый расход перераспределяется по трещинам одинаково, а забойное давление равно сжимающему напряжению, действующему на трещину со стороны породы.

Условие остановки представлено на слайде.\\

\textbf{Слайд 15 (формула Кёнинга).}

Теперь на основе найденных расходов, давления в трещине и времени необходимо найти приращение длины трещины автоГРП.

В работе Кёнинга получена формула зависимости полудлины трещины гидроразрыва при условии доминирования утечек в рамках модели Картера.

По этой формуле явная зависимость полудлины трещины от времени представляет собой корневую зависимость.

Однако в задаче моделирования совместного роста нескольких трещин автоГРП поток и давление в трещинах тоже меняются со временем, поэтому дополнительно по формуле Кёнинга есть неявная зависимость полудлины трещины от времени.

Для учёта этой неявной зависимости всё время моделирования разделяется на небольшие интервалы, на каждом интервале пересчитываются поток и давление в трещине и вычисляется приращение полудлины трещины на этом шаге по времени.\\

\textbf{Слайд 16 (приращение полудлины трещины).}

Приращение полудлины трещины рассчитывается через полную производную полудлины трещины по времени.

Выражение для полной производной полудлины трещины по времени, а также частные производные полудлины трещины по расходу, давлению в трещине и времени представлены на слайде.\\

\textbf{Слайд 17 (приращение полудлины трещины).}

Здесь представлено выражение для приращения полудлины трещины, которое будет использовано на каждом временном шаге.

В формуле три слагаемых: первое отвечает за приращение полудлины трещины за счёт изменения времени, второе -- за счёт изменения расхода на рассматриваемой трещине; третье -- за счёт изменения давления в трещине.\\

\textbf{Слайд 18 (выбранные значения входных параметров).}

На этом слайде представлены выбранные значения входных параметров, которые были использованы при проведении расчётов.\\

\textbf{Слайд 19 (результат совместного использования формулы Кёнинга с решателем уравнений Кирхгофа).}

Представлены графики зависимости полудлины трещин от времени, зависимость забойного давления от времени, расходов воды и чистого давления на каждой из трещин от времени. 

Для трещины 1 расчёт проводился со значениями параметров с предыдущего слайда.

Для остальных трещин было снижено качество перфораций в течение всего времени расчёта.\\

\textbf{Слайд 20 (влияние резкого ухудшения качества перфораций на рост трещин).}

На этом слайде приведены результаты в случае, когда первые 50 секунд на всех трещинах были исходные входные параметры.
А затем в моменты времени 50, 100 и 150 секунд качество перфорации ухудшалось на второй, третьей и четвёртой трещинах соответственно.

Из графиков видим, что (при условии фиксированного расхода на забое) внезапное ухудшение качества перфорации на одной из трещин может привести к неконтролируемому росту соседних трещин, что может вызвать их прорыв к добывающим скважинам и существенное снижение эффективности эксплуатации месторождения.\\

\textbf{Слайд 21 (выводы).}

Итак, в данной работе проведён обзор моделей трещины гидроразрыва пласта, реализован численный алгоритм расчёта потоков на каждой из трещин по правилам Кирхгофа, проведено совмещение формулы Кёнинга с решателем уравнений Кирхгофа (а именно реализован алгоритм расчёта приращения полудлины трещины на каждом шаге по времени с учётом изменяющихся расходов и давления в трещинах).
А также сделан вывод, что внезапное ухудшение качества перфораций на одной из трещин может привести к неконтролируемому росту соседних трещин при фиксированном расходе на забое.

В дальнейшем необходимо дополнить построенную модель: например, учесть эффекты пороупругости, когда большие утечки из трещин влияют на упругое состояние породы и тем самым влияют на соседние трещины.
В этом случае может происходить эффект закрытия трещин автоГРП.

\end{document}
