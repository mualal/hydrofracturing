\chapter{Формула Кёнинга в задаче о перераспределении потоков} \label{ch3}

На основе модели Картера \cite{karter_book} в работе \cite{koning_book} получена формула Кёнинга, которая представляет собой зависимость полудлины трещины автоГРП от расхода жидкости и фильтрационно-ёмкостных свойств пласта:
\beq\label{2_1}
x_{\!f}=\frac{Q\mu\sqrt{\pi\kappa t}}{2\pi k_eh\left(p_{\!f}-p_e\right)},
\eeq
где $Q$ -- расход нагнетаемой в рассматриваемую трещину жидкости;\newline
$\mu$ -- вязкость жидкости;\newline
$\kappa=k_e/(\varphi_e\mu c_t)$ -- коэффициент пьезопроводности пласта;\newline
$t$ -- время закачки;\newline
$k_e$ -- проницаемость пласта;\newline
$\varphi_e$ -- пористость пласта;\newline
$c_t$ -- общая сжимаемость;\newline
$h$ -- эффективная толщина (мощность) пласта;\newline
$\Delta p=p_{\!f}-p_e$ -- разница между средним давлением в трещине и пластовым давлением.\\

В данной работе рассматривается одновременный рост нескольких трещин автоГРП, поэтому расход жидкости на каждой из них динамично изменяется согласно законам Кирхгофа.
Кроме того, давление в трещинах тоже изменяется согласно выражению \eqref{1_3} по мере увеличения объёма трещин и изменения расхода на них.

Таким образом, для корректного применения формулы Кёнинга приращение полудлины каждой из трещин необходимо найти как произведение полной производной формулы Кёнинга по времени на рассматриваемый шаг по времени.\newline
Полная производная полудлины трещины $x_{\!f}$ по времени $t$:
\beq\label{2_2}
\frac{dx_{\!f}}{dt}=\frac{\partial x_{\!f}}{\partial t}+\frac{\partial x_{\!f}}{\partial Q}\frac{dQ}{dt}+\frac{\partial x_{\!f}}{\partial p_{\!f}}\frac{dp_{\!f}}{dt}
\eeq
Частная производная полудлины трещины $x_{\!f}$ по времени $t$:
\beq\label{2_3}
\frac{\partial x_{\!f}}{\partial t}=\frac{Q\mu}{4\pi k_eh\left(p_{\!f}-p_e\right)}\sqrt{\frac{\pi\kappa}{t}}
\eeq
Частная производная полудлины трещины $x_{\!f}$ по расходу $Q$:
\beq\label{2_4}
\frac{\partial x_{\!f}}{\partial Q}=\frac{\mu\sqrt{\pi\kappa t}}{2\pi k_eh\left(p_{\!f}-p_e\right)}
\eeq
Частная производная полудлины трещины $x_{\!f}$ по давлению в трещине $p_{\!f}$:
\beq\label{2_5}
\frac{\partial x_{\!f}}{\partial p_{\!f}}=-\frac{Q\mu\sqrt{\pi\kappa t}}{2\pi k_eh\left(p_{\!f}-p_e\right)^2}
\eeq
Подставляя \eqref{2_3}, \eqref{2_4} и \eqref{2_5} в выражение \eqref{2_2}, получаем:
\beq\label{2_6}
\frac{dx_{\!f}}{dt}=\frac{Q\mu}{4\pi k_eh\left(p_{\!f}-p_e\right)}\sqrt{\frac{\pi\kappa}{t}}+\frac{\mu\sqrt{\pi\kappa t}}{2\pi k_eh\left(p_{\!f}-p_e\right)}\frac{dQ}{dt}-\frac{Q\mu\sqrt{\pi\kappa t}}{2\pi k_eh\left(p_{\!f}-p_e\right)^2}\frac{dp_{\!f}}{dt}
\eeq

Итак, приращение полудлины трещины на каждом шаге по времени записывается в следующем виде:
\beq\label{2_7}
dx_{\!f}=\frac{Q\mu}{4\pi k_eh\left(p_{\!f}-p_e\right)}\sqrt{\frac{\pi\kappa}{t}}dt+\frac{\mu\sqrt{\pi\kappa t}}{2\pi k_eh\left(p_{\!f}-p_e\right)}dQ-\frac{Q\mu\sqrt{\pi\kappa t}}{2\pi k_eh\left(p_{\!f}-p_e\right)^2}dp_{\!f}
\eeq\\
Совмещение формулы Кёнинга с уравнениями Кирхгофа будет проведено следующим образом:

1) на текущем шаге по времени по имеющимся значениям полудлины трещин автоГРП предыдущего шага будут рассчитаны давления и расходы на каждой трещине;

2) на основе полученных значений приращения давления и расхода будет найдено приращение полудлины трещины $dx_{\!f}$ на данном временном шаге по формуле \eqref{2_7};

3) по формуле $x_{\!f}^{\text{current}}=x_{\!f}^{\text{last}}+dx_{\!f}$ будут найдены полудлины каждой из трещин на текущем временном шаге;

4) описанные действия будут проделаны до требуемого шага по времени (условия остановки).

