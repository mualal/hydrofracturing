\chapter{Обзор методов моделирования трещин автоГРП} \label{ch1}

По мере разработки первых моделей гидроразрыва пласта появляются первые статьи, посвящённые эффекту самопроизвольного роста трещин вследствие закачки в нагнетательную скважину жидкости под высоким давлением.

В одной из первых таких работ \cite{hagoort_phd} рассматриваются различные аспекты моделирования трещин на нагнетательных скважинах, проведён обзор критериев распространения трещин, полученных методами Гриффитса \cite{griffith}, Ирвина \cite{irwin} и Баренблатта \cite{barenblatt}, а также представлено решение задачи механики трещины в линейно-пороупругой однородной изотропной среде.

При выводе решения в работе \cite{hagoort_phd} рассматривается трещина с полудлиной $L_f$, распространяющаяся от нагнетательной скважины в пористой проницаемой породе.
Предполагается, что высота трещины много больше её длины, а давление по всей длине трещине одинаково и равно $p_f$, откуда вытекает эллиптичность горизонтального сечения трещины.
Другими словами, для рассматриваемой трещины используется модель KGD с пренебрежимо малой вязкостью закачиваемой жидкости.
Также предполагается, что напряжение, действующее на трещину со стороны породы, постоянно и равно $S_h$, а пластовое давление вдали от трещины равно $p_e$.

Чтобы решить задачу аналитически, вводится упрощённый профиль порового давления, напоминающий истинный профиль порового давления:
\beq
p(\xi)=p_e+\Delta p \exp{\left(-\frac{\xi-\xi_f}{\lambda}\right)},
\eeq
где $\xi$ -- координата в эллиптической системе координат;
$\Delta p=p_f-p_e$ -- репрессия на пласт;
$\lambda$ -- константа темпа падения.\newline
Этот профиль давления удовлетворяет граничным условиям на скважине ($p=p_f$) и на бесконечности ($p=p_e$).
Глубина проникновения давления определяется постоянной темпа падения $\lambda$, которую можно рассматривать как функцию времени.
Зависимость между глубиной проникновения давления $L_p$ и константой $\lambda$ задана в следующем виде:
\beq\label{Lp_via_lambda}
L_p=L_f\sinh{\lambda}
\eeq
При $\lambda\ll 1$ выражение \eqref{Lp_via_lambda} запишется в виде $L_p=\lambda L_f$, а при $\lambda\gg 1$ выражение \eqref{Lp_via_lambda} примет следующий вид: $\lambda=\ln{\left(2L_p/L_f\right)}$.

Распределение напряжений вокруг трещины найдено в виде суммы трёх функций упругих напряжений и специальной функции пороупругих напряжений.
На основе результатов и найденного распределения напряжений получено выражение для раскрытия трещины:
\beq\label{HaggoortFractureOpening}
u(x)=\frac{2\left(1-\nu^2\right)L_{\!f}\sqrt{1-\dfrac{x^2}{L_f^2}}}{E}\left(p_f-S_h-\frac{\lambda}{1+2\lambda}A\left(p_f-p_e\right)\right),
\eeq
где
$A=\dfrac{1-2\nu}{1-\nu}\left(1-\dfrac{c_r}{c_b}\right)$ -- пороупругая константа;
$c_r$ -- сжимаемость материала породы;
$c_b$ -- сжимаемость породы, насыщенной флюидами.

Максимальное раскрытие трещины (вблизи скважины) запишется в виде:
\beq
w_f=\frac{2\left(1-\nu^2\right)L_f}{E}\left(p_f-S_h-\frac{\lambda}{1+2\lambda}A\left(p_f-p_e\right)\right)
\eeq
Видим, что ширина трещины уменьшается при увеличении глубины проникновения давления.

Также из \eqref{HaggoortFractureOpening} вытекает формула для давления открытия/закрытия трещины ($u=0$):
\beq\label{HaggoortOpenCrit}
p_{foc}=\dfrac{S_h-\dfrac{\lambda}{1+2\lambda}Ap_e}{1-\dfrac{\lambda}{1+2\lambda}A}
\eeq
Согласно \eqref{HaggoortOpenCrit} давление открытия/закрытия трещины увеличивается при увеличении глубины проникновения давления.
А при малых значениях $\lambda$ давление закрытия (оно же давление смыкания) трещины равно напряжению, действующему на трещину со стороны породы $p_{foc}=S_h$.

Далее с помощью метода Гриффитса \cite{griffith} найдено давление распространения трещины:
\beq
p_{fp}=p_{foc}+\frac{K_{Ic}/\sqrt{\pi L_f}}{\left(1-\dfrac{\lambda}{1+2\lambda}A\right)},
\eeq
где $K_{Ic}$ -- критический коэффициент интенсивности напряжений (трещиностойкость породы).

Таким образом, трещина остаётся стабильной (не распространяется), если давление в трещине выше давления открытия/закрытия трещины не более, чем на
$$
\frac{K_{Ic}/\sqrt{\pi L_f}}{\left(1-\dfrac{\lambda}{1+2\lambda}A\right)}.
$$
Это максимальное избыточное давление уменьшается при увеличении длины трещины (для длинных трещин давление распространения практически равно давлению смыкания) и увеличивается при увеличении глубины проникновения давления.

Важное исследование распространения трещин автоГРП было представлено в работе \cite{hagoort}, в которой путём совмещения аналитической модели трещины с численной моделью пласта изучена скорость распространения трещин.
В этой работе сделан вывод, что предположение одномерности утечек (модель Картера \cite{karter}) часто приводит к ошибочным результатам.

Позже в работе \cite{perkins_gonzalez} представлена модель распространения одной трещины автоГРП, в которой учтены двумерность утечек и влияние термоупругих изменений на скорость распространения трещин.
Было показано, что охлаждение породы вследствие закачки холодной воды может привести к очень длинным трещинам, так как при охлаждении порода сжимается и происходит термоупругое уменьшение горизонтальных напряжений пласта.

Далее в статье \cite{koning} предложена модель трещины автоГРП, которая может включать как одномерные утечки, перпендикулярные трещине, так и двумерные радиальные утечки.

Показано, что если скорость распространения трещины существенно выше скорости распространения возмущения пластового давления, то применима модель одномерных утечек Картера \cite{karter}, перпендикулярных трещине.
В этом случае получено выражение для полудлины трещины автоГРП:
\beq
x_f=\frac{Q\mu\sqrt{\pi\kappa t}}{2\pi k_e h\left(p_f-p_e\right)}
\eeq

Если же трещина распространяется существенно медленнее возмущения пластового давления, то модель утечек Картера несомненно неприменима и требуется рассматривать более сложные модели утечек, например, двумерные радиальные утечки.
В этом случае в статье \cite{koning} также получена формула для полудлины трещины автоГРП:
\beq
x_f=3\exp{\left(-\frac{2\pi k_e h\left(p_f-p_e\right)}{Q\mu}\right)}\sqrt{\kappa t}
\eeq








