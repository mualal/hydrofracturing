\chapter{Модели трещины гидроразрыва пласта и их основные компоненты} \label{ch2}

В настоящее время в симуляторах для моделирования процесса ГРП нефтяные компании используют модели Pseudo3D, Planar3D и Full3D.

Наиболее общая модель Full3D позволяет моделировать сложные варианты развития трещины, решение проводится численно с применением метода конечных элементов (МКЭ), но эта модель используется редко, так как имеет низкую скорость расчёта.

В модели Planar3D предполагается, что направление минимальных горизонтальных напряжений в пласте не изменяется в зависимости от координаты, то есть трещина распространяется в одной плоскости.
В то же время модель Planar3D не использует приближение малости высоты в сравнении с длиной трещины, то есть учитывает двумерное течение жидкости.

Модель Pseudo3D использует предположение о том, что высота трещины много меньше её длины, то есть рассматривается случай одномерного течения жидкости.

Каждая из моделей Pseudo3D, Planar3D и Full3D плохо поддаётся аналитическому анализу.
Однако при введении дополнительных предположений и допущений модель Planar3D преобразуется в хорошо известные модели (исторически были изучены раньше модели Planar3D), для которых можно провести аналитический анализ (найти раскрытие, давление в зависимости от координаты и времени).
В таблице представлены основные модели трещины ГРП с их допущениями.

\noindent % for correct centering
\begingroup
\centering
\small %выставляем шрифт в 12bp
\begin{longtable}[c]{|p{2.5cm}|p{6.5cm}|p{6.5cm}|}
	\caption{Предположения основных моделей трещины ГРП}%
	\label{tab:long:invest}% label всегда желательно идти после caption
	\\
	\hline
	\multicolumn{1}{|c|}{\textbf{Модель}}&\multicolumn{1}{|c|}{\textbf{Допущения}}&\multicolumn{1}{|c|}{\textbf{Схематичный рисунок}}\\ \hline
	\endfirsthead%
	\captionsetup{format=tablenocaption,labelformat=continued} % до caption!
	\caption[]{}\\ % печать слов о продолжении таблицы
	\hline
	\multicolumn{1}{|c|}{\textbf{Модель}}&\multicolumn{1}{|c|}{\textbf{Допущения}}&\multicolumn{1}{|c|}{\textbf{Схематичный рисунок}}\\ \hline
	\endhead
	\hline
	\endfoot
	\hline
	\endlastfoot
	Full3D&отсутствуют&тест\\ \hline
	Planar3D&распространение в плоскости&тест\\ \hline
	Pseudo3D&одномерное течение жидкости&тест\\ \hline
	KGD&прямоугольное вертикальное сечение; плоская деформация в горизонтальной плоскости&тест\\ \hline
	Радиальная&проверить&тест\\ \hline
	PKN&эллиптическое вертикальное сечение; плоская деформация в вертикальной плоскости&тест\\ \hline
	
\end{longtable}
\normalsize% возвращаем шрифт к нормальному
\endgroup


Любая модель трещины гидроразрыва пласта состоит из нескольких основных компонентов:

1) уавнения баланса жидкости с учётом утечек;

2) модели жидкости;

3) уравнения упругости;

4) условия распространения;

5) модели транспорта проппанта.

Далее будут представлены уравнения, описывающие компоненты трещины ГРП.

\section{Уравнения баланса жидкости с учётом утечек}

Для планарной трещины (planar3D) верно равенство:
\begin{multline}
w(t+dt)dxdy = w(t)dxdy+q_x(x)dtdy-q_x(x+dx)dtdy+\\+q_y(y)dtdx-q_y(y+dy)dtdx-2gdxdydt
\end{multline}

Откуда получаем уравнение баланса жидкости:
\beq
\frac{\partial w}{\partial t}+\frac{\partial q_x}{\partial x}+\frac{\partial q_y}{\partial y}+2g=Q_0\delta(x,y)
\eeq

Из модели Картера:
\beq
g=\frac{C_l}{\sqrt{t-t_0(x,y)}}
\eeq


\section{Модели жидкости}

\section{Уравнения упругости}

\section{Условия распространения}

\section{Модели транспорта проппанта}



%\beq
%\frac{\partial\sigma_{xx}}{\partial x}+\frac{\partial\sigma_{xy}}{\partial y}=0
%\eeq


%\section{Название параграфа} \label{ch2:sec1}


