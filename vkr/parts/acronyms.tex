\chapter*{Список сокращений и условных обозначений}             % Заголовок
\addcontentsline{toc}{chapter}{Список сокращений и условных обозначений}  % Добавляем его в оглавление
\noindent
\addtocounter{mytotaltables}{-1}% Нужно откатить на единицу счетчик номеров таблиц, так как следующая таблица сделана для удобства представления информации по ГОСТ
%\begin{longtabu} to \dimexpr \textwidth-5\tabcolsep {r X}
\begin{longtabu} to \textwidth {r X} % Таблицу не прорисовываем!
% Жирное начертание для математических символов может иметь
% дополнительный смысл, поэтому они приводятся как в тексте
% диссертации
\textbf{ГРП} & Гидроразрыв пласта \\
\textbf{ЛУМР} & Линейно-упругая механика разрушения \\
\textbf{МКЭ} & Метод конечных элементов \\
\textbf{KGD} & Модель Христиановича-Желтова-Гиртсма-деКлерка \\
\textbf{PKN} & Модель Перкинса-Керна-Нордгрена \\

\end{longtabu}
