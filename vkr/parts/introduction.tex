\chapter*{Введение} % * не проставляет номер
\addcontentsline{toc}{chapter}{ВВЕДЕНИЕ} % вносим в содержание


Важным фактором, влияющим на эффективность добычи углеводородов при разработке месторождения, является система заводнения, которая организуется с целью поддержания пластового давления (ППД) и увеличения нефтеотдачи пласта.
Часто при организации системы заводнения осуществляют перевод добывающих скважин, отработавших на истощение, в нагнетание.
Поскольку нагнетание производится с большим расходом и давление жидкости, как правило, превышает давление разрыва породы, то на таких скважинах возникает риск инициации самопроизвольного роста техногенных трещин.
Данное явление называется эффектом автоГРП, а длина трещин автоГРП может варьироваться от десятков метров до километра и более.

На момент перевода в нагнетание большинство эксплуатационного фонда скважин было ранее (во время работы в добывающем фонде) простимулировано многостадийным гидроразрывом пласта.
На таких скважинах может инициироваться одновременный рост нескольких трещин автоГРП (по одной трещине из каждого порта ранее проведённого многостадийного гидроразрыва).

Неконтролируемый рост трещины автоГРП может привести к негативным последствиям, которые зависят от геометрических размеров и ориентации трещины и заключаются в том, что развитие трещины может стать причиной обводнения добывающих скважин, а также причиной прорыва воды в верхние или нижние горизонты, что снижает эффективность эксплуатации месторождения.

С другой стороны, контролируемый рост трещин автоГРП может значительно увеличить приёмистость нагнетательных скважин и существенно повысить эффективность заводнения, что приведёт к увеличению эффективности эксплуатации месторождения \cite{bazyrov_shel, yakupov}.

Чтобы проводить грамотный контроль роста трещин автоГРП и снизить риски их неконтролируемого распространения важно научиться моделировать одновременный рост нескольких трещин автоГРП в длину.

Важным фактором, влияющим на скорость распространения трещины автоГРП, является расход жидкости на рассматриваемой трещине.
При этом сам рост трещины может провоцировать изменение расхода на этой и соседних трещинах из-за постепенного изменения параметров, характеризующих физическое состояние породы, скважины, перфораций и так далее.
Поэтому важно на основе имеющих численных алгоритмов расчёта перераспределения потоков между трещинами ГРП \cite{elbel} построить алгоритм расчёта перераспределения потоков между трещинами автоГРП и провести совмещение этого алгоритма с известными моделями роста трещины автоГРП в длину \cite{koning}.

\emph{Целью} данной работы является моделирование роста нескольких трещин автоГРП с учётом перераспределения закачиваемого в скважину расхода жидкости между этими трещинами.

\emph{Объектом исследования} является горизонтальная нагнетательная скважина с несколькими трещинами автоГРП, которые образуются вследствие закачки в скважину воды под высоким давлением. 
\emph{Предметом исследования} являются забойное давление на рассматриваемой скважине, расходы воды на каждой из трещин при заданном расходе воды на забое и зависимость полудлины каждой из трещин автоГРП от времени.

Для достижения поставленной цели будут решены следующие \emph{задачи}:

1) обзор моделей трещины гидроразрыва пласта и литературы по моделированию роста трещины автоГРП;

2) расчёт потоков на каждой из нескольких трещин автоГРП при заданных входных параметрах, определяющих физическое состояние породы, скважины и трещин;

3) построение физико-математической модели роста нескольких трещин автоГРП с учётом перераспределения потоков между ними при изменении входных параметров со временем;

4) анализ зависимости полудлины каждой из трещин автоГРП, забойного давления и расходов жидкости на каждой из трещин от времени при различных сценариях изменения входных параметров со временем.
