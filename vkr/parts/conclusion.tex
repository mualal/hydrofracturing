\chapter*{Заключение} \label{ch-conclusion}
\addcontentsline{toc}{chapter}{ЗАКЛЮЧЕНИЕ}

В данной работе были проанализированы известные модели роста трещин гидроразрыва, для каждой из трещин автоГРП выбрана модель PKN, после чего был реализован алгоритм расчёта перераспределения потоков между трещинами автоГРП и с использованием формул Кёнинга было проведено моделирование роста трещин автоГРП при различных сценариях изменения входных параметров со временем.

Проведённый анализ показал, что на скорость роста трещины автоГРП существенно влияет качество перфораций на этой трещине.

На языке Python написан код для моделирования перераспределения потоков между трещинами автоГРП и построены графики зависимости полудлины трещин автоГРП от времени, зависимости забойного давления от времени, расходов и чистого давления на каждой из трещин от времени.
