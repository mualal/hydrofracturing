\chapter*{Заключение} \label{ch-conclusion}
\addcontentsline{toc}{chapter}{Заключение}

В данной работе были проанализированы известные модели роста трещин гидроразрыва, для каждой из трещин автоГРП выбрана модель PKN, после чего была построена модель одновременного роста нескольких трещин автоГРП и модель перераспределения потоков между трещинами автоГРП.

Проведённый анализ показал, что на скорость роста трещины автоГРП существенно влияет качество перфораций на этой трещине.

На языке Python написан код для моделирования перераспределения потоков между трещинами автоГРП и построены графики зависимости полудлины трещин автоГРП от времени, зависимости забойного давления от времени, расходов и чистого давления на каждой из трещин от времени.

Сделан вывод, что для поддержания роста трещин автоГРП требуется увеличение забойного давления, которое влечёт за собой увеличение чистого давления на каждой из трещин и дальнейшее увеличение длины каждой из трещин автоГРП.

В дальнейшем планируется учесть возможную остановку роста одной или нескольких трещин автоГРП из-за ухудшения фильтрационных свойств на ней 