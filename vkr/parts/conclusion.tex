\chapter*{Заключение} \label{ch-conclusion}
\addcontentsline{toc}{chapter}{ЗАКЛЮЧЕНИЕ}

В данной работе был проведён обзор известных моделей трещины гидроразрыва пласта и методов моделирования роста трещин автоГРП.
На основе этого обзора выбрана модель PKN и подход Кёнинга для моделирования роста трещин автоГРП, так как их предположения наиболее точно соответствуют условиям распространения трещин на нагнетательных скважинах (длина трещины много больше высоты и практически вся закачиваемая в трещину жидкость утекает в пласт). 

Далее на основе правил Кирхгофа и введённых определяющих соотношений (которые определяют чистое давление в распространяющейся трещине PKN, падение давления на перфорациях и падение давления на трение в скважине) был построен алгоритм расчёта потоков на каждой из нескольких трещин автоГРП при заданных входных параметрах, определяющих физическое состояние породы, скважины и трещин.
Другими словами, на языке программирования Python с помощью метода Ньютона реализован решатель уравнений Кирхгофа.

На основе формул Кёнинга найдены формулы для приращения полудлины трещин в случае одномерных утечек Картера и в случае двумерных радиальных утечек жидкости из трещины в пласт.
Проведено совмещение формул Кёнинга с решателем уравнений Кирхгофа, а именно построена модель роста нескольких трещин автоГРП с учётом перераспределения потоков между ними при изменении входных параметров, определяющих физическое состояние породы скважины и трещин, со временем.

Проведён анализ зависимости полудлины каждой из трещин, забойного давления и расходов жидкости на каждой из трещин от времени при различных сценариях изменения входных параметров со временем. Сделаны следующие выводы:
\begin{itemize}
	\item предположение одномерности утечек жидкости из трещины в пласт по Картеру может завышать значения полудлин растущих трещин автоГРП;
	\item уменьшение диаметра перфораций на одной из трещин приводит к постепенному закрытию этой трещины и одновременному более интенсивному росту соседних трещин;
	\item уменьшение расхода на забое скважины приводит к уменьшению расходов на трещинах и сокращению их длины;
	\item термоупругое уменьшение горизонтальных напряжений в пласте (например, при охлаждение породы в случае закачки холодной воды) приводит к более интенсивному росту трещин автоГРП;
	\item падение давления на трение в трубе приводит к существенной разнице расходов на нескольких трещинах автоГРП и, как следствие, трещины растут с разной скоростью.
\end{itemize}

\vspace*{3mm}

На основе построенной модели можно проводить расчёт перераспределения потоков между любыми трещинами гидроразрыва, но перед этим необходимо изменить определяющие соотношения, использованные в решателе уравнений Кирхгофа, на соотношения, которые наиболее точно описывают рассматриваемый случай роста трещин ГРП.

В дальнейших работах необходимо дополнить построенную модель роста нескольких трещин автоГРП, а именно обратить особое внимание на эффекты пороупругости, когда большие утечки жидкости из трещины в пласт влияют на упругое состояние породы и тем самым влияют на направление и темп изменения длины соседних трещин.
Учёт этих эффектов важен, так как может приводить к внезапному закрытию трещин автоГРП.





