\documentclass[main.tex]{subfiles}

\begin{document}

\section*{Приближённое решение для плоской трещины ГРП, учитывающее трещиностойкость, вязкость жидкости и утечки (Е.В. Донцов)}
\addcontentsline{toc}{section}{Приближённое решение для плоской трещины ГРП, учитывающее трещиностойкость, вязкость жидкости и утечки (Е.В. Донцов)}

\textbf{Аннотация}

Цель этой статьи состоит в том, чтобы разработать приближенное решение для распространяющейся плоской трещины гидроразрыва пласта, поведение которой определяется комбинированным взаимодействием вязкости жидкости, трещиностойкости и утечки жидкости.
Аппроксимация построена на предположении, что поведение трещины в первую очередь определяется трехпроцессной (вязкостью, трещиностойкостью и утечкой) многомасштабной асимптотикой кончика и глобальным балансом объема жидкости.
Во-первых, рассматриваются решения предельных режимов распространения, которые можно привести к явному виду.
После этого исследуются области применимости предельных решений и анализируются переходы от одного предельного решения к другому.
Для количественной оценки погрешности построенного приближенного решения его предсказания сравниваются с эталонным численным решением.
Результаты показывают, что аппроксимация способна прогнозировать параметры трещины гидроразрыва для всех предельных и переходных режимов с погрешностью менее одного процента.
Следовательно, результаты данной работы могут быть использованы для получения быстрого решения для плоской трещины гидроразрыва с утечкой, которое можно использовать для быстрой оценки геометрии трещины или в качестве опорного решения для оценки точности более совершенных симуляторов гидроразрыва пласта.

\subsection{Введение}

Гидроразрыв пласта -- это технология, которая в основном используется в нефтяной промышленности для стимуляции нефтяных и газовых скважин, см. к примеру (Economides and Nolte 2000).
Большие объемы жидкости для гидроразрыва закачиваются глубоко в недра для создания трещин, которые служат высокопроницаемыми путями, увеличивающими добычу углеводородов.
Несмотря на то, что гидроразрыв пласта известен в нефтяных приложениях, он также используется для восстановления отходов (Frank and Barkley 2005), удаления отходов (Abou-Sayed et al. 1989) и предварительного кондиционирования при добыче горных пород (Jeffrey and Mills 2000).
Естественные проявления гидроразрывов включают подъем магмы через литосферу (дайки), который широко изучался в последние десятилетия (Spence and Turcotte 1985; Lister 1990; Lister and Kerr 1991; Rubin 1995; Roper and Lister 2005, 2007; Dontsov, 2016b), и заполненные жидкостью трещины в ложах ледников (Tsai and Rice 2010).
Модели гидроразрыва обычно включают поток вязкой жидкости внутри трещины, баланс между жидкостью внутри трещины и жидкостью, просачивающейся в окружающий пласт, упругое равновесие породы и критерий распространения, учитывающий сопротивление породы.
Существует множество моделей гидроразрыва пласта, различающихся геометрией трещины, реологией жидкости гидроразрыва и типом критерия распространения, и это лишь некоторые из них.
Поскольку численные симуляторы для гидроразрыва пласта выходят за рамки данного исследования, читатели могут обратиться к недавним обзорным статьям, которые содержат более подробные описания различных моделей гидроразрыва пласта (Adachi et al. 2007; Weng 2015; Peirce 2016).

Известно, что трещины гидроразрыва имеют сложное многомасштабное поведение даже для простейших случаев полубесконечной или конечной плоской трещины (KGD) и радиальной трещины, см. к примеру подробный обзорный документ (Detournay 2016).
Полубесконечная трещина ГРП является моделью кончика трещины гидроразрыва [см. (Peirce and Detournay 2008) об этом упрощении в случае плоской трещины] и, следовательно, является фундаментальной проблемой, которая проливает свет на глобальное поведение более сложной трещины гидроразрыва.
Проблема концевой области трещины гидроразрыва хорошо изучена в работах (Desroches et al. 1994; Lenoach 1995; Garagash and Detournay 2000; Detournay and Detournay 2003; Garagash et al. 2011; Dontsov and Peirce 2015; Dontsov 2016c), где включены различные комбинации доминирующих физических процессов, связанных с вязкостью флюида, трещиностойкостью породы, запаздыванием между трещиной и фронтом флюида (fluid lag), утечкой флюида и (в последнее время) турбулентным течением флюида.
Эволюция во времени и режимы распространения плоской трещины KGD и радиальной трещины гидроразрыва (или трещины в форме копейки) также были тщательно изучены, соответственно: (Adachi and Detournay 2002; Garagash and Detournay 2005; Garagash 2006; Adachi and Detournay 2008; Hu and Garagash 2010, Detournay 2004) и (Detournay 2004; Savitski and Detournay 2002; Bunger et al. 2005; Bunger and Detournay 2007; Madyarova 2003).
В недавнем обзорной статье (Detournay 2016) представлено более подробное резюме результатов и подчеркнуто сложное многомасштабное поведение пространственного изменения и временной эволюции трещин гидроразрыва.
Ввиду многомасштабного поведения трещин ГРП большинство исследований сосредоточено на предельных режимах распространения, при которых доминирует один из механизмов, определяющих отклик.
Есть только несколько исследований, которые строго рассматривают полное решение, в которых учтены вязкость жидкости, трещиностойкость и утечка жидкости одновременно.
В контексте трещин KGD это сделано в (Hu and Garagash 2010).
Для радиальных трещин ГРП (трещин в форме копейки) аналогичные полные многомасштабные решения были получены в работах (Madyarova 2003; Dontsov 2016a).
Исследования (Hu and Garagash 2010; Madyarova 2003) были сосредоточены на разработке численных решений, а также на анализе многомасштабного поведения.
Напротив, цель исследования (Dontsov 2016a) состояла в том, чтобы разработать приближенное аналитическое решение, способное достаточно точно отразить всю сложность многомасштабного поведения, что позволяет быстро получить решение и еще больше увеличивает понимание поведения решения.
Следуя процедуре, описанной в (Dontsov 2016a) для радиальной трещины, целью данного исследования является разработка аналитического решения для плоской трещины, которое учитывает одновременное взаимодействие эффектов трещиностойкости, вязкости жидкости и утечки жидкости.
Численное решение и анализ этой задачи были выполнены в (Hu and Garagash 2010).
По этой причине в данном исследовании ограничен анализ исходной задачи, но в то же время исследование сосредоточено на разработке, анализе и оценке точности приближенного решения.

Важность разработанного приближенного решения для плоской трещины гидроразрыва можно обозначить следующим образом.
Во-первых, показано, что область кончика играет решающую роль в моделировании гидроразрыва, поскольку аппроксимация основана на предположении, что глобальная эволюция трещины в основном определяется поведением около кончика и глобальным балансом объема жидкости.
Аппроксимация относительно проста в реализации и позволяет быстро получить решение, что может быть полезно для быстрых оценок параметров трещины при любых значениях трещиностойкости, вязкости жидкости и утечек.
Решение еще больше повышает уровень понимания проблемы, позволяя быстро и более тщательно исследовать параметрическое пространство.
В частности, с помощью этого решения построена карта решений, которая указывает области применимости предельных решений и позволяет легко определить, соответствует ли решение при заданном наборе параметров задачи одному из предельных случаев (для которого приведены явные решения).
Наконец, разработанное приближенное решение может быть использовано в качестве опорного решения для оценки точности различных симуляторов ГРП, а также может быть использовано в качестве начального условия для повышения устойчивости численных схем на ранних временах.

Эта работа организована следующим образом.
В разделе 2 описаны основные уравнения для плоской трещины ГРП, которая приводится в движение ньютоновской жидкостью в проницаемой породе с заданной трещиностойкостью.
В разделе 3 описана процедура получения приближенного решения.
Анализ предельных режимов распространения представлен в разделе 4.
Раздел 5 посвящен описанию структуры решения путем анализа областей перехода от одного предельного решения к другому.
Наконец, в разделе 6 представлена оценка точности аппроксимации в сравнении с эталонным численным решением.
Результаты обобщены в разделе 7.

\subsection{Основные уравнения для плоской трещины ГРП (модели KGD)}

В этом исследовании рассматривается распространение плоской (или KGD) трещины ГРП в проницаемой породе, и в этом разделе кратко описаны основные уравнения, связанные с математической моделью трещины.
Во-первых, удобно ввести четыре основных параметра материала, которые появляются в модели, как
\beq
\begin{gathered}
\mu'=12\mu,\,\,\,\,\,\,E'=\frac{E}{1-\nu^2}\\
K'=4\left(\frac{2}{\pi}\right)^{1/2}K_{Ic},\,\,\,\,\,\,C'=2C_L,
\end{gathered}
\eeq
где $\mu$ -- вязкость жидкости (жидкость предполагается ньютоновской),
$E$ -- модуль Юнга,
$\nu$ -- коэффициент Пуассона,
$K_{Ic}$ -- первая мода трещиностойкости породы;
$C_L$ -- параметр утечки Картера.
Обратите внимание, что указанные выше параметры материала будут появляться исключительно в масштабированной форме (1) в остальной части статьи.

Используя обозначения, согласующиеся с (1), баланс объёма несжимаемой ньютоновской жидкости внутри одномерной трещины можно записать как
\beq
\frac{\partial w}{\partial t}+\frac{\partial q}{\partial x}+\frac{C'}{\sqrt{t-t_0(x)}}=Q_0\delta(x),\,\,\,\,\,q=-\frac{w^3}{\mu'}\frac{\partial p}{\partial x},
\eeq
где $w(x,t)$ -- ширина трещины,
$q$ -- поток в направлении $x$ (т.е. вдоль трещины),
член, пропорциональный $C'$, учитывает утечку по модели Картера, $t_0(x)$ -- это момент времени, когда фронт трещины находился в точке $x$,
$p$ -- давление жидкости,
$Q_0$ -- закачиваемый расход жидкости (считается постоянным во времени).

Упругое равновесие горной породы характеризуется уравнением упругости, которое для симметричной плоской трещины даётся выражением (см. к примеру Crouch and Starfield 1983; Hills et al. 1996)
\beq
p(x,t)=-\frac{E'}{2\pi}\int\limits_{0}^{l}{\frac{x}{x^2-x'^2}}\frac{\partial w(x',t)}{\partial x'}dx',
\eeq
где $l$ -- длина одиночного крыла трещины (полудлина трещины).
Эффект обратного напряжения, который объясняет изменение давления флюида в окружающей породе из-за утечки, не учитывается в модели для простоты.
В этом случае необходимо соблюдать осторожность при использовании результатов для очень больших значений утечки.

Распространение трещины моделируется в соответствии с Механикой Линейно-Упругого Разрушения (LEFM) (Rice 1968)
\beq
w\to\frac{K'}{E'}\left(l-x\right)^{1/2},\,\,\,\,\,x\to l,
\eeq
что означает, что первая мода коэффициента интенсивности напряжения равна трещиностойкости распространяющейся трещины.
В дополнение к условию распространения (4) реализуется условие нулевого потока на кончике трещины, т.е. $q(l,t)=0$.

Другим важным соотношением является глобальный баланс объёма жидкости, который получается путем интегрирования (2) по времени и пространству и может быть записан как
\beq
\int\limits_{0}^{l}{\left(w(x,t)+2C'\sqrt{t-t_0(x)}\right)dx}=\frac{Q_0t}{2},
\eeq
где для получения результата использовались равенства $q(l,t)=0$ и $w(l,t)=0$.

\subsection{Приближённое решение для плоской трещины ГРП}

\subsubsection{Краткое описание методологии}

Основное допущение, которое используется в этом исследовании для получения приближенного решения, заключается в том, что на эволюцию трещины в основном влияет поведение около кончика и глобальный баланс объёма жидкости (5).
Следовательно, решение для ширины трещины аппроксимируется функцией, которая автоматически удовлетворяет поведению вблизи кончика и аппроксимирует решение при удалении от кончика, как
\beq
w(x,t)=\left(\frac{l+x}{2l}\right)^{\lambda}w_a(l-x),
\eeq
где $w_a$ -- асимптотическое решение вблизи кончика трещины, а $\lambda$ -- параметр, который будет определен позже.
Асимптотическое решение на кончике $w_a$ удовлетворяет основным уравнениям (2), (3) и (4) в области кончика, определяемой как $(l-x)/l\ll 1$.
В результате построенная аппроксимация (6) автоматически решает (2), (3) и (4) в области кончика и только аппроксимирует решение вдали от кончика.
Обратите внимание, что решение для кончика $w_a$ получается при рассмотрении полубесконечной трещины, которая стационарно распространяется в условиях плоской упругой деформации (Garagash et al. 2011; Dontsov and Peirce 2015), а также зависит от параметров материала (1) и времени через $l(t)$ и $\dot{l}(t)$.

Чтобы приступить к аппроксимации, необходимо задать решение вблизи кончика $w_a$, которое можно быстро вычислить.
Поэтому численное решение для $w_a$ не подходит, и для данного исследования используется заамкнутое приближенное решение для $w_a$ (Dontsov and Peirce 2015).
Это приближенное асимптотическое решение для кончика с тремя процессами (т.е. трещиностойкостью, вязкостью и утечкой) имеет максимальную ошибку 0.14\%, согласуется с принятой моделью плоской трещины и учитывает влияние трещиностойкости, вязкости жидкости и утечек, а также незамедлительно обеспечивает нас решением.
В статье (Dontsov and Peirce 2015) показано, что приближенное решение на кончике удовлетворяет условию $w_a(s)\propto s^{\delta}$, где $s=l-x$ и $\delta$ -- медленно меняющаяся функция.
В этой ситуации приближение для ширины трещины (6) сводится к
\beq
w(x,t)=\left(\frac{l+x}{2l}\right)^{\lambda}\left(1-\frac{x}{l}\right)^{\bar{\delta}}w_a(l).
\eeq
Изменение длины трещины во времени для предельных режимов распространения плоской трещины (Bunger et al. 2005; Adachi and Detournay 2002, 2008; Garagash and Detournay 2005; Garagash 2006; Detournay 2004; Hu and Garagash 2010) всегда представляется в форме $l(t)\propto t^{\alpha}$ и $\alpha$ -- число, равное либо $2/3$, либо $1/2$.
На основании этого результата далее предполагается, что длина трещины для приближенного решения имеет вид $l(t)\propto t^{\alpha}$, где $\alpha$ -- медленно меняющаяся со временем функция.
В этом приближении функция времени срабатывания $t_0(x)$ может быть определена из соотношения $x/l=\left(t_0/t\right)^{\alpha}$.
Результат можно подставить в глобальный баланс объёма жидкости (5) вместе с (7), чтобы получить
\beq
w_a(l)\int\limits_{0}^{1}{\left(\frac{1+\xi}{2}\right)^{\lambda}\left(1-\xi\right)^{\bar{\delta}}d\xi}+2C't^{1/2}+\int\limits_{0}^{1}{\sqrt{1-\xi^{1/\alpha}}d\xi}=\frac{Q_0t}{2l},
\eeq
где $\xi=x/l$ -- масштабированная пространственная координата.
Интегралы в (8) могут быть вычислены и тогда уравнение преобразуется к следующему виду:
\beq
w_a(l)\,2^{1+\bar{\delta}}\,B_0\!\left(\frac{1}{2};\lambda+1,\bar{\delta}+1\right)+2C't^{1/2}\alpha\,B\!\left(\alpha,\frac{3}{2}\right)=\frac{Q_0t}{2l},
\eeq
где $B(a,b)$ -- бета-функция и
$$
B_0(x;a,b)\equiv\int\limits_{x}^{1}{t^{a-1}(1-t)^{b-1}dt}=B(a,b)-B(x;a,b),
$$
где $B(x;a,b)$ -- неполная бета-функция.
Сведение интегрального уравнения (5) к алгебраическому уравнению (9) является одним из ключевых шагов в выводе приближенного решения для плоской трещины ГРП.

Оставшаяся часть этого раздела кратко описывает процедуру вывода приближенного решения, а в разд.3.2 даётся более подробное описание.
Первоначально должны быть заданы параметры материала (1), скорость нагнетания $Q_0$ и функция $w_a(l)$.
Отметим, что $w_a(l)$ также зависит от параметров материала (1) и времени через $\dot{l}=\alpha l/t$.
Кроме того, выражение для $\bar{\delta}$ появляется из асимптотического решения для кончика трещины.
Затем, задав $\lambda$ (процедура описана позже в разделе 4.5) и приняв начальное предположение $\alpha=2/3$ (соответствующее нулевой утечке), уравнение (9) можно решить относительно $l(t)$ (например, с помощью метода Ньютона).
После историю изменения длины трещины $l(t)$ во времени можно использовать для обновления значения $\alpha$, используя выражение $\alpha=d\log{(l)}/d\log{(t)}$.
Затем уравнение (9) снова решается с новыми значениями $\alpha$.
Такая итерационная процедура выполняется до тех пор, пока не будет достигнута сходимость, которая обычно достигается быстро (две или три итерации) из-за относительно небольшого изменения $\alpha$.
История длины трещины $l(t)$ является основным параметром, который необходимо рассчитать, и через него могут быть выражены другие величины.
Как только $l(t)$ получено, пространственное изменение ширины может быть выведено из (7).
Эффективность, определяемая как отношение текущего объема трещины к общему количеству закачиваемой жидкости, может быть рассчитана как
\beq
\eta(t)=\frac{2^{2+\bar{\delta}}lw_a(l)}{Q_0t}\,B_0\!\left(\frac{1}{2};\lambda+1,\bar{\delta}+1\right)
\eeq
Один из простейших способов вычисления давления жидкости состоит в том, чтобы подставить (7) в (3) и вычислить интеграл, так что
\beq
\begin{gathered}
p=\frac{E'w_a(l)}{l}\mathcal{F}\left(\lambda,\bar{\delta},\xi\right),\\
\mathcal{F}\left(\xi,\lambda,\bar{\delta}\right)=\frac{1}{2^{1+\lambda}\pi}\int\limits_{0}^{1}{\frac{\partial M(\xi,s)}{\partial s}(1+s)^{\lambda}(1-s)^{\bar{\delta}}ds},\\
M(\xi,s)=\frac{\xi}{\xi^2-s^2},
\end{gathered}
\eeq
где функцию $\mathcal{F}\left(\xi,\lambda,\bar{\delta}\right)$ можно оценить численно.

\subsubsection{Решение в отмасштабированных переменных}

Для того, чтобы решить (9) численно, необходимо сначала задать асимптотическое решение для кончика $w_a$.
Как показано в (Dontsov and Peirce 2016, 2017), решение $w_a$ неявно задается следующим уравнением


\subsection{Сравнение с вершинными решениями}

\subsubsection{Предельное решение в $M$ вершине}

\subsubsection{Предельное решение в $\tilde{M}$ вершине}

\subsubsection{Предельное решение в $K$ вершине}

\subsubsection{Предельное решение в $\tilde{K}$ вершине}

\subsubsection{Интерполяция параметра $\lambda$}

\subsection{Структура решения}

\subsubsection{Переход вдоль границы $MK$}

\subsubsection{Переход вдоль границы $M\tilde{M}$}

\subsubsection{Переход вдоль границы $K\tilde{K}$}

\subsubsection{Переход вдоль границы $\tilde{M}\tilde{K}$}

\subsection{Сравнение с численным решением}

\subsection{Резюме}

\subsection{Данные по работе}

\subsection*{Приложение A. Функции $g_{\delta}\!\left(\hat{K},\hat{C}\right)$ и $\Delta\!\left(\hat{K},\hat{C}\right)$}
\addcontentsline{toc}{subsection}{Приложение A. Функции $g_{\delta}\!\left(\hat{K},\hat{C}\right)$ и $\Delta\!\left(\hat{K},\hat{C}\right)$}


\subsection*{Приложение B. Численная схема}
\addcontentsline{toc}{subsection}{Приложение B. Численная схема}


\subsection*{Список использованной литературы}
\addcontentsline{toc}{subsection}{Список использованной литературы}

Abou-Sayed A, Andrews D, Buhidma I (1989) Evaluation of oily waste injection below the permafrost in prudhoe bay field. In: Proceedings of the California regional meetings. CA, Society of Petroleum Engineers. Richardson, Bakersfield, pp 129–142

Adachi J, Siebrits E, Peirce A, Desroches J (2007) Computer simulation of hydraulic fractures. Int J Rock Mech Min Sci 44:739–757

Adachi J, Detournay E (2002) Self-similar solution of a plane strain fracture driven by a power-law fluid. Int J Numer Anal Methods Geomech 26:579–604 

Adachi JI, Detournay E (2008) Plane-strain propagation of a hydraulic fracture in a permeable rock. Eng Fract Mech 75:4666–4694

Bunger A, Detournay E, Garagash D (2005) Toughness dominated hydraulic fracture with leak-off. Int J Fract 134:175–190

Bunger A, Detournay E (2007) Early time solution for a penny-shaped hydraulic fracture. ASCE J Eng Mech 133:175–190

Crouch S, Starfield A (1983) Boundary element methods in solid mechanics. George Allen and Unwin, London

Desroches J, Detournay E, Lenoach B, Papanastasiou P, Pearson J, Thiercelin M, Cheng AD (1994) The crack tip region in hydraulic fracturing. Proc R Soc Lond A 447:39–48

Detournay E (2004) Propagation regimes of fluid-driven fractures in impermeable rocks. Int J Geomech 4:35–45

Detournay E (2016) Mechanics of hydraulic fractures. Annu Rev fluid Mech 48(31):139

Detournay E, Garagash D (2003) The tip region of a fluid-driven fracture in a permeable elastic solid. J fluid Mech 494:1–32

Dontsov E (2016a) An approximate solution for a penny-shaped hydraulic fracture that accounts for fracture toughness, fluid viscosity, and leak-off. R Soc Open Sci 3(160):737

Dontsov E (2016b) Propagation regimes of buoyancy-driven hydraulic fractures with solidification. J fluid Mech 797:128

Dontsov E (2016c) Tip region of a hydraulic fracture driven by a laminar-to-turbulent fluid flow. J fluid Mech 797:R2

Dontsov E, Peirce A (2016) Implementing a universal tip asymptotic solution into an implicit level set algorithm (ILSA) for multiple parallel hydraulic fractures. In: Proceedings of the 50th US rock mechanics symposium, Houston, TX, ARMA-2016-268. American Rock Mechanics Association, Houston

Dontsov E, Peirce A (2015) A non-singular integral equation formulation to analyze multiscale behaviour in semi-infinite hydraulic fractures. J fluid Mech 781:R1

Dontsov E, Peirce A (2017) A multiscale implicit level set algorithm (ILSA) to model hydraulic fracture propagation incorporating combined viscous, toughness, and leak-off asymptotics. Comput Methods Appl Mech Eng 313:53–84

Economides M, Nolte K (eds) (2000) Reservoir stimulation, 3rd edn. Wiley, Chichester

Frank U, Barkley N (2005) Remediation of low permeability subsurface formations by fracturing enhancements of soil vapor extraction. J Hazard Mater 40:191–201

Garagash D (2006) Plane-strain propagation of a fluid-driven fracture during injection and shut-in: asymptotics of large toughness. Eng Fract Mech 73:456–481

Garagash D, Detournay E, Adachi J (2011) Multiscale tip asymptotics in hydraulic fracture with leak-off. J fluid Mech 669:260–297

Garagash D, Detournay E (2000) The tip region of a fluid-driven fracture in an elastic medium. J Appl Mech 67:183–192

Garagash D, Detournay E (2005) Plane-strain propagation of a fluid-driven fracture: small toughness solution. ASME J Appl Mech 72:916–928

Hills D, Kelly P, Dai D, Korsunsky A (1996) Solution of crack problems, the distributed dislocation technique, solid mechanics and its applications, vol 44. Kluwer Academic Publisher, Dordrecht

Hu J, Garagash D (2010) Plane-strain propagation of a fluid-driven crack in a permeable rock with fracture toughness. J Eng Mech 136:1152–1166

Jeffrey R, Mills K (2000) Hydraulic fracturing applied to inducing longwall coal mine goaf falls. Pacific Rocks 2000. Balkema, Rotterdam, pp 423–430

Lenoach B (1995) The crack tip solution for hydraulic fracturing in a permeable solid. J Mech Phys Solids 43:1025–1043

Lister JR (1990) Buoyancy-driven fluid fracture: the effects of material toughness and of low-viscosity precursors. J fluid Mech 210:263–280

Lister J, Kerr R (1991) fluid-mechanical models of crack propagation and their application to magma transport in dykes. J Geophys Res 96:10,049–10,077

Madyarova M (2003) fluid-driven penny-shaped fracture in elastic medium. Master’s thesis, University of Minnesota

Peirce A (2016) Implicit level set algorithms for modelling hydraulic fracture propagation. Phil Trans R Soc A 374(20150):423. doi:10.1098/rsta.2015.0423 

Peirce A, Detournay E (2008) An implicit level set method for modeling hydraulically driven fractures. Comput Methods Appl Mech Eng 197:2858–2885

Rice J (1968) Mathematical analysis in the mechanics of fracture. In: Liebowitz H (ed) Fracture: an advanced treatise, Chap 3, vol II. Academic Press, New York, pp 191–311

Roper S, Lister JR (2005) Buoyancy-driven crack propagation from an over-pressured source. J fluid Mech 536:79–98

Roper S, Lister JR (2007) Buoyancy-driven crack propagation: the limit of large fracture toughness. J fluid Mech 580:359380

Rubin A (1995) Propagation of magma-filled cracks. Annu Rev Earth Planet 23:287–336

Savitski A, Detournay E (2002) Propagation of a fluid-driven penny-shaped fracture in an impermeable rock: asymptotic solutions. Int J Solids Struct 39:6311–6337

Spence D, Turcotte D (1985) Magma-driven propagation of cracks. J Geophys Res 90:575–580

Tsai V, Rice J (2010) A model for turbulent hydraulic fracture and application to crack propagation at glacier beds. J Geophys Res 115(F03):007

Weng X (2015) Modeling of complex hydraulic fractures in naturally fractured formation. J Unconv Oil Gas Res 9:114–135

\newpage
\setcounter{figure}{0}
\setcounter{subsection}{0}
\setcounter{equation}{0}

\end{document}
