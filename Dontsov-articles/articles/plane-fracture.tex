\documentclass[main.tex]{subfiles}

\begin{document}

\section*{Приближённое решение для плоской трещины ГРП, учитывающее трещиностойкость, вязкость жидкости и утечки (Е.В. Донцов)}
\addcontentsline{toc}{section}{Приближённое решение для плоской трещины ГРП, учитывающее трещиностойкость, вязкость жидкости и утечки (Е.В. Донцов)}

\textbf{Аннотация}

\subsection{Введение}

\subsection{Основные уравнения для плоской трещины ГРП (модели KGD)}

\subsection{Приближённое решение для плоской трещины ГРП}

\subsubsection{Краткое описание методологии}

\subsubsection{Решение в отмасштабированных переменных}

\subsection{Сравнение с вершинными решениями}

\subsubsection{Предельное решение в $M$ вершине}

\subsubsection{Предельное решение в $\tilde{M}$ вершине}

\subsubsection{Предельное решение в $K$ вершине}

\subsubsection{Предельное решение в $\tilde{K}$ вершине}

\subsubsection{Интерполяция параметра $\lambda$}

\subsection{Структура решения}

\subsubsection{Переход вдоль границы $MK$}

\subsubsection{Переход вдоль границы $M\tilde{M}$}

\subsubsection{Переход вдоль границы $K\tilde{K}$}

\subsubsection{Переход вдоль границы $\tilde{M}\tilde{K}$}

\subsection{Сравнение с численным решением}

\subsection{Резюме}

\subsection{Данные по работе}

\subsection*{Приложение A. Функции $g_{\delta}\!\left(\hat{K},\hat{C}\right)$ и $\Delta\!\left(\hat{K},\hat{C}\right)$}
\addcontentsline{toc}{subsection}{Приложение A. Функции $g_{\delta}\!\left(\hat{K},\hat{C}\right)$ и $\Delta\!\left(\hat{K},\hat{C}\right)$}


\subsection*{Приложение B. Численная схема}
\addcontentsline{toc}{subsection}{Приложение B. Численная схема}


\subsection*{Список использованной литературы}
\addcontentsline{toc}{subsection}{Список использованной литературы}

\newpage
\setcounter{figure}{0}
\setcounter{subsection}{0}
\setcounter{equation}{0}

\end{document}
