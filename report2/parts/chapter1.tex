\chapter{Постановка задачи моделирования роста трещин автоГРП в длину} \label{ch1}

На основе модели одномерных утечек Картера \cite{karter_book} в работе \cite{koning_book} получена первая формула Кёнинга, которая представляет собой зависимость полудлины трещины автоГРП от расхода жидкости, фильтрационно-ёмкостных свойств (ФЕС) пласта, репрессии на пласт и времени:
\beq\label{Koning_first}
x_{\!f}=\frac{Q\mu\sqrt{\pi\kappa t}}{2\pi k_eh\left(p_{\!f}-p_e\right)},
\eeq
где $Q$ -- расход нагнетаемой в рассматриваемую трещину жидкости;
$\mu$ -- вязкость жидкости;
$\kappa=k_e/(\varphi_e\mu c_t)$ -- коэффициент пьезопроводности пласта;
$t$ -- время закачки;
$k_e$ -- проницаемость пласта;
$\varphi_e$ -- пористость пласта;
$c_t$ -- общая сжимаемость системы (состоит из сжимаемости флюидов и сжимаемости порового пространства);
$h$ -- эффективная толщина (мощность) пласта;
$\Delta p=p_{\!f}-p_e$ -- разница между средним давлением в трещине и пластовым давлением (репрессия на пласт).\\

В текущей работе рассматривается одновременный рост нескольких трещин гидроразрыва, поэтому расход жидкости на каждой из них может динамично меняться согласно правилам Кирхгофа.
Кроме того, давление в трещинах в общем случае тоже может меняться по мере увеличения объёма трещин и изменения расхода на них.
Получается, что согласно формуле \eqref{Koning_first} есть зависимость полудлины трещины $x_{\!f}$ от расхода на трещине $Q$, но при этом на расход в общем случае может влиять текущая полудлина каждой из трещин.

Таким образом, для корректного применения формулы Кёнинга приращение полудлины каждой из трещин на текущем шаге по времени необходимо найти как произведение полной производной формулы Кёнинга по времени и рассматриваемого временного шага.\newline
Полная производная полудлины трещины $x_{\!f}$ по времени $t$:
\beq\label{FullDerivative}
\frac{dx_{\!f}}{dt}=\frac{\partial x_{\!f}}{\partial t}+\frac{\partial x_{\!f}}{\partial Q}\frac{dQ}{dt}+\frac{\partial x_{\!f}}{\partial (p_{\!f}-p_e)}\frac{d(p_{\!f}-p_e)}{dt}
\eeq
Частная производная полудлины трещины $x_{\!f}$ по времени $t$:
\beq\label{PartialDerivative_t_first}
\frac{\partial x_{\!f}}{\partial t}=\frac{Q\mu}{4\pi k_eh\left(p_{\!f}-p_e\right)}\sqrt{\frac{\pi\kappa}{t}}
\eeq
Частная производная полудлины трещины $x_{\!f}$ по расходу $Q$:
\beq\label{PartialDerivative_Q_first}
\frac{\partial x_{\!f}}{\partial Q}=\frac{\mu\sqrt{\pi\kappa t}}{2\pi k_eh\left(p_{\!f}-p_e\right)}
\eeq
Частная производная полудлины трещины $x_{\!f}$ по репрессии на пласт $\left(p_{\!f}-p_e\right)$:
\beq\label{PartialDerivative_p_first}
\frac{\partial x_{\!f}}{\partial (p_{\!f}-p_e)}=-\frac{Q\mu\sqrt{\pi\kappa t}}{2\pi k_eh\left(p_{\!f}-p_e\right)^2}
\eeq
Подставляя \eqref{PartialDerivative_t_first}, \eqref{PartialDerivative_Q_first} и \eqref{PartialDerivative_p_first} в выражение \eqref{FullDerivative}, получаем:
\beq\label{FullDerivativeExplicit_first}
\frac{dx_{\!f}}{dt}=\frac{\mu}{2\pi k_e h(p_{\!f}-p_e)}\left(\frac{Q}{2}\sqrt{\frac{\pi\kappa}{t}}+\sqrt{\pi\kappa t}\,\frac{dQ}{dt}-\frac{Q\sqrt{\pi\kappa t}}{\left(p_{\!f}-p_e\right)}\frac{d(p_{\!f}-p_e)}{dt}\right)
\eeq
             
Итак, приращение полудлины трещины на каждом шаге по времени при рассмотрении случая одномерных утечек Картера записывается в следующем виде:
\beq\label{IncrementExplicit_first}
dx_{\!f}=\frac{\mu}{2\pi k_e h(p_{\!f}-p_e)}\left(\frac{Q}{2}\sqrt{\frac{\pi\kappa}{t}}dt+\sqrt{\pi\kappa t}\,dQ-\frac{Q\sqrt{\pi\kappa t}}{\left(p_{\!f}-p_e\right)}d(p_{\!f}-p_e)\right)
\eeq\\

Дополнительно в работе \cite{koning_book} получена вторая формула Кёнинга, которая применяется в случае двумерных радиальных утечек жидкости из трещины в пласт и представляет собой зависимость полудлины трещины автоГРП от расхода жидкости, фильтрационно-ёмкостных свойств (ФЕС) пласта, репрессии на пласт и времени:
\beq\label{Koning_second}
x_{\!f}=3\exp{\!\left(-\frac{2\pi k_e h\left(p_{\!f}-p_e\right)}{Q\mu}\right)}\sqrt{\kappa t}
\eeq
В этом случае частная производная полудлины трещины $x_{\!f}$ по времени $t$:
\beq\label{PartialDerivative_t_second}
\frac{\partial x_{\!f}}{\partial t}=\frac{3}{2}\exp{\left(-\frac{2\pi k_e h \left(p_{\!f}-p_e\right)}{Q\mu}\right)}\sqrt{\frac{\kappa}{t}}
\eeq
Частная производная полудлины трещины $x_{\!f}$ по расходу $Q$:
\beq\label{PartialDerivative_Q_second}
\frac{\partial x_{\!f}}{\partial Q}=\frac{6\pi k_e h\left(p_{\!f}-p_e\right)}{Q^2\mu}\exp{\!\left(-\frac{2\pi k_e h\left(p_{\!f}-p_e\right)}{Q\mu}\right)}\sqrt{\kappa t}
\eeq
Частная производная полудлины трещины $x_{\!f}$ по репрессии на пласт $\left(p_{\!f}-p_e\right)$:
\beq\label{PartialDerivative_p_second}
\frac{\partial x_f}{\partial (p_{\!f}-p_e)}=-\frac{6\pi k_e h}{Q\mu}\exp{\!\left(-\frac{2\pi k_e h\left(p_{\!f}-p_e\right)}{Q\mu}\right)}\sqrt{\kappa t}
\eeq

Подставляя \eqref{PartialDerivative_t_second}, \eqref{PartialDerivative_Q_second} и \eqref{PartialDerivative_p_second} в выражение \eqref{FullDerivative}, получаем:
\beq
\begin{gathered}
\frac{dx_{\!f}}{dt}=\exp{\!\left(-\frac{2\pi k_e h\left(p_{\!f}-p_e\right)}{Q\mu}\right)}\left(\frac{3}{2}\sqrt{\frac{\kappa}{t}}\,+\right.\\[10pt]
+\left.\frac{6\pi k_e h\left(p_{\!f}-p_e\right)}{Q^2\mu}\sqrt{\kappa t}\,\frac{dQ}{dt}-\frac{6\pi k_e h}{Q\mu}\sqrt{\kappa t}\,\frac{d(p_{\!f}-p_e)}{dt}\right)
\end{gathered}
\eeq

Итак, приращение полудлины трещины на каждом шаге по времени при рассмотрении случая двумерных радиальных утечек жидкости из трещины в пласт записывается в следующем виде:
\beq\label{IncrementExplicit_second}
\begin{gathered}
dx_{\!f}=\exp{\!\left(-\frac{2\pi k_e h\left(p_{\!f}-p_e\right)}{Q\mu}\right)}\left(\frac{3}{2}\sqrt{\frac{\kappa}{t}}dt\,+\right.\\[10pt]
+\left.\frac{6\pi k_e h\left(p_{\!f}-p_e\right)}{Q^2\mu}\sqrt{\kappa t}\,dQ-\frac{6\pi k_e h}{Q\mu}\sqrt{\kappa t}\,d(p_{\!f}-p_e)\right)
\end{gathered}
\eeq

Постановка задачи принимает следующий вид: используя решатель уравнений Кирхгофа и формулы для приращения полудлины трещины \eqref{IncrementExplicit_first} и \eqref{IncrementExplicit_second}, построить графики зависимостей полудлины трещины и расхода на трещинах от времени при одномерных утечках Картера и при двумерных радиальных утечках жидкости из трещины в пласт.
