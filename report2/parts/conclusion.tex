\chapter*{Заключение} \label{ch-conclusion}
\addcontentsline{toc}{chapter}{Заключение}

В данной работе на основе формул Кёнинга найдены формулы для приращения полудлины трещин в случае одномерных утечек Картера и в случае двумерных радиальных утечек жидкости из трещины в пласт.
Проведено совмещение формул Кёнинга с решателем уравнений Кирхгофа, а именно построена модель роста нескольких трещин автоГРП с учётом перераспределения потоков между ними при изменении входных параметров, определяющих физическое состояние породы скважины и трещин, со временем.

Проведён анализ зависимости полудлины каждой из трещин, забойного давления и расходов жидкости на каждой из трещин от времени при различных сценариях изменения входных параметров со временем. Сделаны следующие выводы:
\begin{itemize}
	\item предположение одномерности утечек жидкости из трещины в пласт по Картеру может завышать значения полудлин растущих трещин автоГРП;
	\item уменьшение диаметра перфораций на одной из трещин приводит к постепенному закрытию этой трещины и одновременному более интенсивному росту соседних трещин;
	\item термоупругое уменьшение горизонтальных напряжений в пласте (например, при охлаждение породы в случае закачки холодной воды) приводит к более интенсивному росту трещин автоГРП.
\end{itemize}

В дальнейшем необходимо дополнить построенную модель, а именно обратить особое внимание на эффекты пороупругости, когда большие утечки жидкости из трещины в пласт влияют на упругое состояние породы и тем самым влияют на направление и темп изменения длины соседних трещин.
Учёт этих эффектов важен, так как может приводить к внезапному закрытию трещин автоГРП.
