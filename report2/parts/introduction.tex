\chapter*{Введение} % * не проставляет номер
\addcontentsline{toc}{chapter}{Введение} % вносим в содержание


Важным фактором, влияющим на эффективность добычи углеводородов при разработке месторождения, является система заводнения, которая организуется с целью поддержания пластового давления (ППД) и увеличения нефтеотдачи пласта.
Часто при организации системы заводнения осуществляют перевод добывающих скважин, отработавших на истощение, в нагнетание.
Поскольку нагнетание производится с большим расходом и давление жидкости, как правило, превышает давление разрыва породы, то на таких скважинах возникает риск инициации самопроизвольного роста техногенных трещин.
Данное явление называется эффектом автоГРП, а длина трещин автоГРП может варьироваться от десятков метров до километра и более.

На момент перевода в нагнетание большинство эксплуатационного фонда скважин было ранее (во время работы в добывающем фонде) простимулировано многостадийным гидроразрывом пласта.
На таких скважинах может инициироваться одновременный рост нескольких трещин автоГРП (по одной трещине из каждого порта ранее проведённого многостадийного гидроразрыва).

Неконтролируемый рост трещины автоГРП может привести к негативным последствиям, которые зависят от геометрических размеров и ориентации трещины и заключаются в том, что развитие трещины может стать причиной обводнения добывающих скважин, а также причиной прорыва воды в верхние или нижние горизонты, что снижает эффективность эксплуатации месторождения.

С другой стороны, контролируемый рост трещин автоГРП может значительно увеличить приёмистость нагнетательных скважин и существенно повысить эффективность заводнения, что приведёт к увеличению эффективности эксплуатации месторождения \cite{bazyrov_shel, yakupov}.

Чтобы проводить грамотный контроль роста трещин автоГРП и снизить риски их неконтролируемого распространения важно научиться моделировать одновременный рост нескольких трещин автоГРП в длину.


\emph{Целью} данной работы является моделирование перераспределения потоков между несколькими трещинами автоГРП.
В качестве базовой модели трещины выбрана модель Перкинса-Керна-Нордгрена, для которой в работе \cite{dontsov2021analysis} представлены приближённые решения во всём параметрическом пространстве и точные аналитические решения в предельных случаях.
Также на практике используют модель Христиановича-Желтова-Гиртсма-деКлерка \cite{dontsov1_book} и радиальную модель \cite{dontsov2_book} трещин ГРП, но они менее точны при моделировании роста трещин автоГРП, так как используют дополнительные предположения, которые не соответствуют процессу роста трещин на нагнетательных скважинах.

\emph{Объектом исследования} является горизонтальная нагнетательная скважина, на которой ранее (когда она работала в добывающем фонде) был проведён многостадийный гидроразрыв.
\emph{Предметом исследования} являются забойное давление на рассматриваемой скважине и расходы воды на каждой из трещин при заданном расходе воды на забое.

В данном отчёте представлены численная реализация алгоритма расчёта полудлин одновременно растущих трещин автоГРП с использованием полных производных формул Кёнинга \cite{koning_book} по времени (рассматриваются первая формула Кёнинга для случая одномерных утечек жидкости по Картеру \cite{karter_book} и вторая формула Кёнинга в случае двумерных радиальных утечек жидкости из трещины в пласт), а также проведён анализ полученных результатов.
