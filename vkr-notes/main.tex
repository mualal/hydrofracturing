\documentclass[a4paper, 11pt]{article}
\renewcommand{\baselinestretch}{1.1}
\usepackage{comment}
\usepackage{lipsum}
\usepackage{fullpage}
\usepackage[a4paper, total={7in, 10in}]{geometry}
\usepackage[fleqn]{amsmath}
\usepackage[utf8]{inputenc}
\usepackage[russian]{babel}
\usepackage{amssymb,amsthm}

\newtheorem{theorem}{Theorem}
\newtheorem{corollary}{Corollary}
\usepackage{graphicx}
\usepackage{tikz}
\usetikzlibrary{arrows}
\usepackage{verbatim}
\usepackage{xcolor}
\usepackage{mdframed}
\usepackage[shortlabels]{enumitem}
\usepackage{indentfirst}
\setlength{\parindent}{0cm}
\usepackage{hyperref}

\begin{document}
\section{Общая структура работы.}

Модель KGD: трещина с прямоугольным вертикальным сечением; применима в случаях, когда высота трещины много больше её длины; допущение о плоской деформации в горизонтальной плоскости;
\\

Модель PKN: трещина с эллиптическим вертикальным сечением; применима в случаях, когда полудлина трещины много больше её высоты; допущение о плоской деформации в вертикальной плоскости;
\\

\section{Важные источники.}

1) Ткаченко Д.Р. Анализ влияния режима работы нагнетательной скважины на рост трещны автоГРП.

2) Hagoort J. Waterflood-induced hydraulic fracturing. PhD. Thesis, Delft Technical Univeёrsity, 1981 (моделирование трещин на нагнетательных скважинах; для трещины KGD получена формула для давления распространения трещины)

3) Hagoort J., Weatherill B.D. and Settari A. Modeling the propagation of waterflood-induced hydraulic fractures. (объединили аналитическую модель трещины с численной моделью пласта и изучили скорость распространения трещины; сделан вывод, что обычная модель Картера для одномерных утечек, перпендикулярных трещине, не всегла верна; здесь показано, что в бесконечном пласте при любой скорости распространения трещины её длина будет пропорциональна квадратному корню времени, различия будут только в коэффициентах)

4) Koning E.J.L. Fractured water-injection wells. Analytical modelling of fracture propagation.

5) Кабанова П.К. Моделирование давления инициации трещины гидроразрыва пласта на нагнетательной скважине в пороупругой постановке (поведение линейной изотропной пороупругой среды)

6) T.K. Perkins, L.R. Kern. Widths of hydraulic fractures

7) R.P. Nordgren. Propagation of vertical hydraulic fractures

8) Тримонова М., Дубиня Н., Основные закономерности развития трещины автоГРП

9) Perkins T.K., Gonzalez, J.A. The effect of thermo-elastic stresses on injection well fracturing ()

10) Gringarten A. C., Ramey H. J., Raghavan R. Unsteady-State Pressure Distributions Created by a Well With a Single Infinite-Conductivity Vertical Fracture. Society of Petroleum Engineers (аналитическое выражение для нахождения давления с постоянной скоростью расхода в стационарной трещине бесконечной проводимости)

11)
\\

\section{Дополнительные источники.}

1) Economides. Unified Fracture Design. Bridging the gap between theory and practice.

2) Логвинюк А.В. Комплексный анализ и моделирование разработки Приобского месторождения для оптимизации системы поддержания пластового давления

3) Старобинский Е.Б. Разработка модели распространения планарной трещины ГРП в слоистой среде

4) Дегтерев Д.А. Интегральные преобразования в планарной модели трещины гидроразрыва пласта

5) Краева С.О. Моделирование переноса и оседания проппанта в трещине ГРП

6) Барсуков С.С. Задача экспресс-оценки корректности моделирования трещины ГРП на примере постановки planar3D.

7) Perkins T. K., Kern L. R. Widths of Hydraulic Fractures

8) Koning E.J.L. Poro- and thermo-elastic rock stresses around a wellbore

9) 
\\


\section{Общие заметки.}

1) Если скорость давления, проходящего через пласт, имеет порядок скорости распространения трещины, распределение утечек будет двумерным в плоскости пласта. Т.е. одномерная модель утечек (модель Картера) не работает.

2) Предположение о малости полудлины трещины по сравнению с толщиной пласта (модель KGD).

3) Модель Картера перестаёт быть верной, когда скорость распространения трещины становится меньше скорости пластового давления.
Возможные режимы утечек: линейный 1D режим (Картер), эллиптический 2D режим (Грингартен), радиальный режим.

4) Повлиять на состояние напряжения в пласте может изменение температуры и давления в нём.
Когда пласт охлаждается, то порода начинает сжиматься и, следовательно, происходит термоупругое уменьшение горизонтальных напряжений пласта.
Поэтому опасно закачивать холодную воду в пласт (неконтроллируемый рост трещин автоГРП).
Изменения горизонтального напряжения в пласте зависит от соотношения высоты трещины и глубины проникновения давления / фронта температур.

5) Насыщение пласта водой по мере закачки может сдерживать рост трещины за счёт эффектов пороупругости.

6) Для случая нулевой утечки: по модели PKN эффективное давление увеличивается во времени, а по модели KGD и радиальной -- уменьшается во времени.
Из постулатов модели KGD (и в радиальной тоже) вытекает, что когда размеры трещины становятся очень большими, требуются очень малые эффективные давления для поддержания определенной ширины.
Хотя это является следствием теории линейной упругости и того способа, как применено допущение о плоской деформации, в целом это приводит к абсурдным результатам.
Можно с уверенностью сказать, что модель PKN лучше описывает физику процесса гидроразрыва, чем две другие модели.

7)
\\

\section{Вопросы.}

1) Важно ли предположение о малости полудлины трещины (по сравнению с высотой пласта) для возможности рассмотрения плоско-деформированной задачи?
Вроде это не связано.
Предположение о малости высоты или полудлины равносильно тому, используем ли двухмерный или одномерный поток жидкости по трещине.
В KGD плоская деформация в горизонтальной плоскости, а в PKN -- в вертикальной.

2) 

3)


\end{document}
