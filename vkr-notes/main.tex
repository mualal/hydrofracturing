\documentclass[a4paper, 11pt]{article}
\renewcommand{\baselinestretch}{1.1}
\usepackage{comment}
\usepackage{lipsum}
\usepackage{fullpage}
\usepackage[a4paper, total={7in, 10in}]{geometry}
\usepackage[fleqn]{amsmath}
\usepackage[utf8]{inputenc}
\usepackage[russian]{babel}
\usepackage{amssymb,amsthm}

\newtheorem{theorem}{Theorem}
\newtheorem{corollary}{Corollary}
\usepackage{graphicx}
\usepackage{tikz}
\usetikzlibrary{arrows}
\usepackage{verbatim}
\usepackage{xcolor}
\usepackage{mdframed}
\usepackage[shortlabels]{enumitem}
\usepackage{indentfirst}
\setlength{\parindent}{0cm}
\usepackage{hyperref}

\begin{document}
\section{Общая структура работы.}

Модель KGD: трещина с прямоугольным сечением; применима в случаях, когда высота трещины много больше её длины;
\\

Модель PKN: трещина с эллиптическим сечением; применима в случаях, когда полудлина трещины много больше её высоты;
\\

\section{Важные источники.}

1) Ткаченко Д.Р. Анализ влияния режима работы нагнетательной скважины на рост трещны автоГРП.

2) Hagoort J. Waterflood-induced hydraulic fracturing. PhD. Thesis, Delft Technical Univeёrsity, 1981.

3) Hagoort J., Weatherill B.D. and Settari A. Modeling the propagation of waterflood-induced hydraulic fractures. (здесь показано, что в бесконечном пласте при любой скорости распространения трещины её длина будет пропорциональна квадратному корню времени, различия будут только в коэффициентах)

4) Koning E.J.L. Fractured water-injection wells. Analytical modelling of fracture propagation.

5) Кабанова П.К. Моделирование давления инициации трещины гидроразрыва пласта на нагнетательной скважине в пороупругой постановке

6) T.K. Perkins, L.R. Kern. Widths of hydraulic fractures

7) R.P. Nordgren. Propagation of vertical hydraulic fractures

8) Тримонова М., Дубиня Н., Основные закономерности развития трещины автоГРП

9) 
\\

\section{Дополнительные источники.}

1) Economides. Unified Fracture Design. Bridging the gap between theory and practice.

2) Логвинюк А.В. Комплексный анализ и моделирование разработки Приобского месторождения для оптимизации системы поддержания пластового давления

3)
\\


\section{Общие заметки.}

1) Если скорость давления, проходящего через пласт, имеет порядок скорости распространения трещины, распределение утечек будет двумерным в плоскости пласта. Т.е. одномерная модель утечек (модель Картера) не работает.

2) Предположение о малости полудлины трещины по сравнению с толщиной пласта (модель KGD).

3) Модель Картера перестаёт быть верной, когда скорость распространения трещины становится меньше скорости пластового давления.
Возможные режимы утечек: линейный 1D режим (Картер), эллиптический 2D режим (Грингартен), радиальный режим.

4) Повлиять на состояние напряжения в пласте может изменение температуры и давления в нём.
Когда пласт охлаждается, то порода начинает сжиматься и, следовательно, происходит термоупругое уменьшение горизонтальных напряжений пласта.
Поэтому опасно закачивать холодную воду в пласт (неконтроллируемый рост трещин автоГРП).
Изменения горизонтального напряжения в пласте зависит от соотношения высоты трещины и глубины проникновения давления / фронта температур.

5)
\\

\section{Вопросы.}

1) Важно ли предположение о малости полудлины трещины (по сравнению с высотой пласта) для возможности рассмотрения плоско-деформированной задачи?

2)

3)


\end{document}
