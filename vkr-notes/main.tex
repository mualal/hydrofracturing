\documentclass[a4paper, 12pt]{article}
\usepackage{comment}
\usepackage{lipsum}
\usepackage{fullpage}
\usepackage[a4paper, total={7in, 10in}]{geometry}
\usepackage{amsmath}
\usepackage[utf8]{inputenc}
\usepackage[russian]{babel}
\usepackage{amssymb,amsthm}

\newtheorem{theorem}{Theorem}
\newtheorem{corollary}{Corollary}
\usepackage{graphicx}
\usepackage{tikz}
\usetikzlibrary{arrows}
\usepackage{verbatim}
\usepackage{xcolor}
\usepackage{mdframed}
\usepackage[shortlabels]{enumitem}
\usepackage{indentfirst}
\setlength{\parindent}{0cm}
\usepackage{hyperref}
\usepackage{float}

\usepackage{setspace}
\setlength{\parindent}{20pt}
\setlength{\parskip}{4pt}

\graphicspath{{./images/}}

\newcommand{\beq}{\begin{equation}}
\newcommand{\eeq}{\end{equation}}

\begin{document}

\tableofcontents
\newpage

\setstretch{1.1}

\section{Общая структура работы}

Модель KGD: трещина с прямоугольным вертикальным сечением; применима в случаях, когда высота трещины много больше её длины; допущение о плоской деформации в горизонтальной плоскости;
\\

Модель PKN: трещина с эллиптическим вертикальным сечением; применима в случаях, когда полудлина трещины много больше её высоты; допущение о плоской деформации в вертикальной плоскости;
\\

\section{Важные источники}

1) Ткаченко Д.Р. Анализ влияния режима работы нагнетательной скважины на рост трещны автоГРП.

2) Hagoort J. Waterflood-induced hydraulic fracturing. PhD. Thesis, Delft Technical Univeёrsity, 1981 (моделирование трещин на нагнетательных скважинах; для трещины KGD получена формула для давления распространения трещины)

3) Hagoort J., Weatherill B.D. and Settari A. Modeling the propagation of waterflood-induced hydraulic fractures. (объединили аналитическую модель трещины с численной моделью пласта и изучили скорость распространения трещины; сделан вывод, что обычная модель Картера для одномерных утечек, перпендикулярных трещине, не всегла верна; здесь показано, что в бесконечном пласте при любой скорости распространения трещины её длина будет пропорциональна квадратному корню времени, различия будут только в коэффициентах)

4) Koning E.J.L. Fractured water-injection wells. Analytical modelling of fracture propagation.

5) Кабанова П.К. Моделирование давления инициации трещины гидроразрыва пласта на нагнетательной скважине в пороупругой постановке (поведение линейной изотропной пороупругой среды)

6) T.K. Perkins, L.R. Kern. Widths of hydraulic fractures

7) R.P. Nordgren. Propagation of vertical hydraulic fractures

8) Тримонова М., Дубиня Н., Основные закономерности развития трещины автоГРП

9) Perkins T.K., Gonzalez, J.A. The effect of thermo-elastic stresses on injection well fracturing ()

10) Gringarten A. C., Ramey H. J., Raghavan R. Unsteady-State Pressure Distributions Created by a Well With a Single Infinite-Conductivity Vertical Fracture. Society of Petroleum Engineers (аналитическое выражение для нахождения давления с постоянной скоростью расхода в стационарной трещине бесконечной проводимости)

11) Базыров И.Ш. Контроль и регулирование роста техногенных трещин при вытеснении нефти из низкопроницаемых коллекторов

12) Калинин С.А. и другие. Анализ слияния трещин автоГРП в рядной системе разработки с помощью математического моделирования

13) 
\\

\section{Дополнительные источники}

1) Economides. Unified Fracture Design. Bridging the gap between theory and practice.

2) Логвинюк А.В. Комплексный анализ и моделирование разработки Приобского месторождения для оптимизации системы поддержания пластового давления

3) Старобинский Е.Б. Разработка модели распространения планарной трещины ГРП в слоистой среде

4) Дегтерев Д.А. Интегральные преобразования в планарной модели трещины гидроразрыва пласта

5) Краева С.О. Моделирование переноса и оседания проппанта в трещине ГРП

6) Барсуков С.С. Задача экспресс-оценки корректности моделирования трещины ГРП на примере постановки planar3D.

7) Perkins T. K., Kern L. R. Widths of Hydraulic Fractures

8) Koning E.J.L. Poro- and thermo-elastic rock stresses around a wellbore

9) 
\\


\section{Общие заметки}

1) Если скорость давления, проходящего через пласт, имеет порядок скорости распространения трещины, распределение утечек будет двумерным в плоскости пласта. Т.е. одномерная модель утечек (модель Картера) не работает.

2) Предположение о малости полудлины трещины по сравнению с толщиной пласта (модель KGD).

3) Модель Картера перестаёт быть верной, когда скорость распространения трещины становится меньше скорости пластового давления.
Возможные режимы утечек: линейный 1D режим (Картер), эллиптический 2D режим (Грингартен), радиальный режим.

4) Повлиять на состояние напряжения в пласте может изменение температуры и давления в нём.
Когда пласт охлаждается, то порода начинает сжиматься и, следовательно, происходит термоупругое уменьшение горизонтальных напряжений пласта.
Поэтому опасно закачивать холодную воду в пласт (неконтроллируемый рост трещин автоГРП).
Изменения горизонтального напряжения в пласте зависит от соотношения высоты трещины и глубины проникновения давления / фронта температур.

5) Насыщение пласта водой по мере закачки может сдерживать рост трещины за счёт эффектов пороупругости.

6) Для случая нулевой утечки: по модели PKN эффективное давление увеличивается во времени, а по модели KGD и радиальной -- уменьшается во времени.
Из постулатов модели KGD (и в радиальной тоже) вытекает, что когда размеры трещины становятся очень большими, требуются очень малые эффективные давления для поддержания определенной ширины.
Хотя это является следствием теории линейной упругости и того способа, как применено допущение о плоской деформации, в целом это приводит к абсурдным результатам.
Можно с уверенностью сказать, что модель PKN лучше описывает физику процесса гидроразрыва, чем две другие модели.

7) Технология поддержания пластового давления связана с рядом проблем, связанных с развитием трещин заводнения, образующихся вследствие высокого давления нагнетания (трещины автоГРП).
Одной из таких проблем является возможность прорыва закачиваемой жидкости в соседнюю скважину через заводняющую трещину, что приводит к необходимости остановки всего ряда нагнетательных скважин на ремонт и, соответственно, снижает эффективность поддержания пластового давления.
Таким образом, важно понимать, какие факторы влияют на процесс слияния трещин. \hspace{10pt} В статье рассмотрена технология заводнения на участке системы разработки с рядной системой заводнения и проведена оценка времени слияния нескольких трещин, инициированных из соседних нагнетательных скважин с помощью математического моделирования.
Для участка системы разработки был проведен комплексный анализ промысловых данных с использованием известных подходов (график Холла, ступенчатый тест, анализ пластового давления).
На основе законов сохранения механики сплошной среды и определяющих уравнений для пороупругой среды разработана численная модель распространения трещин автоГРП.
В качестве исходных данных для модели были взяты характеристические параметры месторождения, расположение скважин и схема ввода скважин в эксплуатацию.
Численное моделирование показывает, что на процесс роста трещины автоГРП влияет сложный фильтрационный процесс между нагнетательными и добывающими скважинами в секторе системы разработки.
При этом существенную роль играет взаимное расположение и расстояние между скважинами.
Поскольку слияние трещин автоГРП влияет на поддержание пластового давления, то для проектирования этого процесса необходимо моделирование трещин автоГРП в рамках совместной геомеханической и гидродинамической задачи.
Численное моделирование позволяет оценить тенденцию зависимости времени слияния трещин заводнения от расстояния между скважинами для конкретного места расположения скважины.

8)
\\

\section{Вопросы}

1) Важно ли предположение о малости полудлины трещины (по сравнению с высотой пласта) в модели KGD для возможности рассмотрения плоско-деформированной задачи?
Вроде это не связано.
Предположение о малости высоты или полудлины равносильно тому, используем ли двухмерный или одномерный поток жидкости по трещине.
В KGD плоская деформация в горизонтальной плоскости, а в PKN -- в вертикальной.

2) 

3)

\newpage
\section{Перевод статьи Нордгрена (Распространение вертикальной трещины гидроразрыва пласта)}

\subsection{Реферат}
В данной статье рассматривается распространение трещин гидроразрыва с ограниченной высотой и эллиптическим вертикальным сечением с учётом эффекта утечек жидкости.
Представлены численные и асимптотические приближённые решения в безразмерной форме, которые дают длину и ширину трещины в любой момент времени и при любом наборе физических параметров.
Ознакомление с безразмерными результатами и приближёнными решениями должно быть полезно при разработке дизайна ГРП.

\subsection{Введение}
Теория и практика гидроразрыва пласта были рассмотрены Howard-ом и Fast-ом.
Поэтому мы ограничим обсуждение предыдущих исследований только темами, которые имеют отношение к текущему исследованию распространения вертикальных трещин.

Важным теоретическим результатом является формула Картера для трещины постоянной ширины, образованной закачкой жидкости с постоянным расходом, с учётом потери жидкости в пласт.
Для вертикальной трещины постоянной высоты формула Картера даёт длину трещины как функцию времени.
В общем случае предположение Картера о постоянной ширине трещины не реалистично.
Однако на больших временах ошибка, вызванная этим предположением, становится несущественной, так как доминируют утечки жидкости в пласт.

Ширина вертикальной трещины впервые была исследована Христиановичем и Желтовым в предположении, что ширина не изменяется в вертикальном направлении.
Таким образом, в горизонтальных плоскостях преобладает состояние плоской деформации и ширина может быть определена как решение плоской задачи упругости.
Приближённое решение найдено в [3] при пренебрежении утечками жидкости, изменением объёма трещины и изменением давления вдоль трещины.
Длина трещины определена из условия конечных напряжений на кончике трещины.
Baron et al. [4] и Гиртсма и деКлерк [5] включили эффект утечек жидкости к подходу [3].
Гиртсма и деКлерк дают простые приближённые формулы для длины и ширины трещины.

Другой подход к определению ширины трещины был использован Перкинсом и Керном.
Они рассмотрели трещину постоянной высоты в предположении плоской деформации в вертикальных плоскостях, перпендикулярных плоскости трещины.
Поперечное сечение трещины в таком предположении получается эллиптическим, а максимальная ширина уменьшается вдоль трещины по простой формуле, содержащей длину трещины.
При выводе этой формулы [6] использовались допущения отсутствия утечек и неизменности объёма трещины в уравнении сплошности, и длина трещины не определяется.
В последующих приложениях [7] предполагалась "<разумная"> длина трещины.
Формула Картера для длины трещины и формула Перкинса и Керна для ширины трещины были процитированы Howard-ом и Fast-ом, и считается, что совместное использование этих двух формул является обычной практикой.

В настоящем теоретическом исследовании рассматривается трещина ограниченной высоты типа, изученного Перкинсом и Керном.
Однако мы включаем эффекты потери жидкости и изменения объёма трещины в уравнение сплошности.
Следовательно, длина трещины определяется как часть решения.
Общие результаты по изменению ширины и длины трещины во времени получены в безразмерном виде численным методом.
Кроме того, выводятся асимптотические решения для больших и малых значений времени.
Решение для малого времени также является точным решением для случая отсутствия утечек жидкости в пласт.
Для больших значений времени наша асимптотическая формула для длины трещины идентична формуле Картера [2] для больших времён.
Наша формула для больших времён для ширины трещины отличается от формулы Перкинса и Керна [6] числовым коэффициентом, который меняется вдоль трещины.
В сравнении с нашей формулой эта формула [6] переоценивает ширину трещины около скважины на 12\% и на середине трещины -- на 24\%. 
На ранних временах формула Перкинса и Керна [6] для ширины трещины снова даёт хорошее приближение к нашему результату.
Однако наша формула для длины трещины отличается от формулы Картера [2], которая неприменима, так как пренебрегаемое в ней изменение ширины трещины важно на ранних временах (формула Картера выведена в предположении неизменности ширины трещины, но это изменение важно учитывать на ранних временах).

Результаты для ширины трещины, ограниченной по высоте, полученные здесь и в работе [6] отличаются от результатов для трещин, постоянных по высоте [3-5].
В частности наше решение для ширины на больших временах отличается от решения Гиртсмы и деКлерка коэффициентом, пропорциональным корню четвёртой степени из высоты трещины, делённому на длину трещины.
Это отличие может быть существенным в практических приложениях, и мы обсудим его далее.

В следующем разделе мы сформулируем задачу об ограниченных по высоте трещинах в духе работы [6].
Основные предположения обсуждаются более подробно.
Затем основные уравнения приводятся в безразмерную форму, пригодную для численного решения, которое выполняется в Приложении А.
В последнем разделе статьи даётся сравнение с предыдущими исследованиями и обсуждение применимости настоящих результатов.

\subsection{Формулировка задачи}

Рассмотрим вертикальную трещину, распространяющуюся по прямой от скважины (рис. \ref{fig:kgd-model-geometry}).
Как и в [6] предполагается, что вертикальная высота трещины ограничена постоянной величиной $h$ слоями устойчивой к разрушению породы.
Сопротивление разрушению может быть связано с более высоким горизонтальным тектоническим напряжением или более высокой прочностью на разрыв в породе, окружающей коллектор.

\begin{figure}[H]
\center
\includegraphics[width=0.5\textwidth]{fracture_geometry}
\caption{Геометрия трещины} 
\label{fig:kgd-model-geometry}  
\end{figure}

Пусть $x$, $y$, $z$ -- система прямоугольных декартовых координат с осью $x$ в направлении распространения трещины, осью $z$, параллельной оси скважины, и началом в забое (рис. \ref{fig:kgd-model-geometry}).
Таким образом, трещина лежит в плоскости $Oxz$ при $0\leqslant x\leqslant L$ и $|z|\leqslant \frac{1}{2}h$.

Следуя за [6], мы предполагаем, что трещину окружает изотропный однородный упругий материал.
Предположение об однородности может не полностью согласовываться с нашим более ранним предположением о том, что трещиностойкая порода ограничивает трещиноватый коллектор.
Однако на основе принципа Сен-Венана в теории упругости упругое поведение вблизи трещины контролируется в основном породой коллектором, и умеренно отличающаяся окружающая порода должна иметь небольшое влияние на ширину трещины.
Мы ограничиваем наше внимание стадией распространения трещины, когда её длина $L$ много больше высоты $h$, и все величины будут изменяться медленно по оси $x$ вдоль большей части трещины.
Это предположение будет проверено после того, как мы получим решение.
Ввиду ожидаемых медленных изменений величин вдоль оси $x$ в вертикальных плоскостях ($x=\textrm{const}$), перпендикулярных трещине, в некотором приближении преобладает состояние плоской деформации.

Перед гидроразрывом сжимающее напряжение ($S$), действует на плоскости, параллельные трещине в коллекторе.
Другие значения $S$ в породе, окружающей резервуар, опять же должны иметь пренебрежимо малый эффект.
Жидкость в трещине находится под давлением $S+\Delta p$ и мы пренебрегаем изменением $\Delta p$ вдоль координаты $z$, т.е. $\Delta p =\Delta p(x,t)$.
Тогда в соответствии с решением задачи о плоской деформации для трещины при постоянном давлении [8], вертикальное сечение трещины имеет форму эллипса и ширина $w(x,t)$ определяется выражением:
\beq\label{width}
w=
\begin{cases}
\dfrac{1-\nu}{G}\left(h^2-4z^2\right)^{1/2}\Delta p,\text{ если }|z|\leqslant\dfrac{1}{2}h,\\[10pt]
0,\text{ если }|z|\geqslant\dfrac{1}{2}h,
\end{cases}
\eeq
где $G$ и $\nu$ -- объёмный модуль сдвига и коэффициент Пуассона для пласта соответственно.
Решение с плоской деформацией не учитывает влияние изменение порового давления на упругую реакцию (упругое состояние породы), которое является небольшим эффектом в задачах разрушения.

Далее рассмотрим течение жидкости в трещине.
Уравнение неразрывности течения несжимаемой жидкости в трещинеможно записать в виде:
\beq\label{Continuity}
\frac{\partial q}{\partial x}+q_l+\frac{\partial A}{\partial t}=0,
\eeq
где $q(x,t)$ -- объёмный расход через поперечное сечение ($x=\textrm{const}$) трещины, $q_l(x,t)$ -- объёмная скорость утечек жидкости в пласт на единицу длины трещины и $A(x,t)$ -- площадь поперечного сечения трещины.

Мы предполагаем, что $q$ связан с градиентом давления классическим решением для ламинарного потока ньютоновской вязкой жидкости в эллиптической трубе с полуосями $\frac{1}{2}h$ и $\frac{1}{2}w_{max}$, где $h\gg w_{max}$, а именно
\beq\label{q_vs_gradient_p}
q=-\frac{\pi W^3h}{64\mu}\frac{\partial\Delta p}{\partial x},\,\,\,\,\,\,\,W=w_{max}
\eeq
где $\mu$ -- вязкость жидкости гидроразрыва. Диапазон применимости ламинарного потока обсуждался в [6].
Из формулы \eqref{width}:
\beq\label{Max_W}
W=w|_{z=0}=\frac{(1-\nu)h}{G}\Delta p
\eeq
и выражение \eqref{q_vs_gradient_p} перепишется в следующем виде:
\beq\label{q_vs_w}
q=-\frac{\pi G}{256(1-\nu)\mu}\frac{\partial}{\partial x}\left(W^4\right)
\eeq

Площадь поперечного сечения эллиптической трещины:
\beq\label{A_area}
A=\int\limits_{-h/2}^{h/2}w\,dz=\frac{\pi}{4}Wh
\eeq

Скорость утечек жидкости в пласт $q_l$ связана со скоростью утечек на единицу площади поверхности трещины следующим равенством:
\beq
q_l=2\int\limits_{-h/2}^{h/2}u_l\,dz,
\eeq
и если $u_l$ не зависит от $z$, то
\beq\label{leak-offs}
q_l=2hu_l
\eeq

Эксперименты [1, 11] показывают, что для многих жидкостей гидроразрыва скорость утечек можно принять в виде:
\beq\label{leak-off}
u_l=\frac{C}{\sqrt{t-\tau}},
\eeq
где $C$ -- коэффициент утечек жидкости и $\tau$ -- время начала течения.
Коэффициент $C$ зависит от различных параметров жидкости и пласта и обычно определяется экспериментально.
Для некоторых жидкостей формулу \eqref{leak-off} следует изменить, включив в неё так называемую "<скачкообразную утечку"> в момент времени $t=\tau$.
Установлено, что влияние скачкообразной утечки на длину трещины в целом невелико (см. приложение E).
Мы принимаем формулу \eqref{leak-off} в настоящей работе, хотя условия, при которых она была проверена экспериментально, строго не выполняются.
А именно разность между давлением в трещине $S+\Delta p$ и пластовым поровым давлением $p_f$, т.е.
\beq\label{pressures_difference}
S+\Delta p-p_f
\eeq
содержит $\Delta p$, которое меняется в зависимости от координаты $x$ и времени $t$.
В некоторых случаях $\Delta p\ll S-p_f$, так что предположение о постоянной разнице давлений должно привести к значимым результатам.
Более того, мы обнаружим, что $\Delta p$ медленно меняется в зависимости от $t$ и $x$ и использование среднего значения для $\Delta p$ в выражении \eqref{pressures_difference} может быть приемлемым.
Условие одномерности течения также строго не выполняется, хотя ввиду медленного изменения $\Delta p$ в зависимости от $x$ результирующая ошибка должна быть малой.
Теоретические уточнения в определении $q_l$ потребуют дополнительных экспериментальных исследований поведения фильтрационной корки под давлением, зависящим от времени.
Кроме того, рассмотрение двухфазного потока необходимо, если жидкость гидроразрыва существенно отличается от жидкости в порах пласта.

В скважине ($x=0$) необходимо задать граничное условие, включающее $q$ и $p$.
Это условие будет зависеть от зависимости $q(p)$ для насосного оборудования, используемого в процессе ГРП.
Здесь мы рассматриваем случай постоянной скорости притока:
\beq\label{gu1}
q(0,t)=q_i=\textrm{const}.
\eeq
В случае других граничных условий задача может быть решена численным методом, представленным в Приложении А.

Изначально трещина закрыта, т.е.
\beq\label{Initial_cond}
W(x,0)=0
\eeq
Более того, трещина остаётся закрытой при $x\geqslant L(t)$, т.е.
\beq\label{GU_1}
W(x,t)=0,\,\,\,\,\,x\geqslant L(t),
\eeq
где длина трещины $L(t)$ должна быть определена как часть решения.

После подстановки \eqref{q_vs_w}, \eqref{A_area}, \eqref{leak-offs} и \eqref{leak-off} в уравнение \eqref{Continuity}, получаем
\beq\label{Main_eq}
\frac{G}{64(1-\nu)\mu h}\frac{\partial^2}{\partial x^2}\left(W^4\right)=\frac{8C}{\pi\sqrt{t-\tau(x)}}+\frac{\partial W}{\partial t},\,\,\,\,\,\,\,0\leqslant x\leqslant L,\,\,\,\,\,\,\,t>0,
\eeq
где $\tau(x)$ -- время, когда трещина открылась в точке с координатой $x$, т.е.
\beq\label{15}
\tau[L(t')]=t',\,\,\,\,\,\,\,\,0\leqslant t'\leqslant t
\eeq
Другими словами, $\tau(x)$ -- это функция, обратная к $L(t)$.

Подстановка формулы \eqref{q_vs_w} в граничное условие \eqref{gu1} даёт:
\beq\label{GU_2}
-\left[\frac{\partial}{\partial x}(W^4)\right]_{x=0}=\frac{256(1-\nu)\mu}{\pi G}q_i.
\eeq

Теперь нашей задачей является решение нелинейного дифференциального уравнения в частных производных \eqref{Main_eq} для раскрытия $W(x,t)$ при начальном условии \eqref{Initial_cond} и граничных условиях \eqref{GU_1} и \eqref{GU_2}.
Длину $L(t)$ тоже необходимо найти.

Чтобы свести к минимуму объём вычислений, удобно привести основные уравнения к безразмерному виду.
Таким образом, если мы введём $x=ax_{d}$, $L=aL_{d}$, $t=Bt_{d}$, $W=eW_d$, где
\beq\label{Dimensionless}
a=\pi\left[\frac{(1-\nu)\mu q_i^5}{256C^8Gh^4}\right]^{1/3},\,\,\,\,\,B=\pi^2\left[\frac{(1-\nu)\mu q_i^2}{32C^5Gh}\right]^{2/3},\,\,\,\,\,e=\left[\frac{16(1-\nu)\mu q_i^2}{C^2Gh}\right]^{1/3},
\eeq
то математическая постановка задачи перепишется в следующем виде:
\beq\label{FullTask}
\begin{cases}
\dfrac{\partial^2}{\partial x_d^2}(W_d^4)=\dfrac{1}{\sqrt{t_d-\tau_d(x)}}+\dfrac{\partial W_d}{\partial t_d},\\[15pt]
\left[\dfrac{\partial}{\partial x_d}(W_d^4)\right]_{x_d=0}=1,\\[15pt]
W(x_d,0)=0,\\[5pt]
W_d(x_d,t_d)=0\text{ для } x_d\geqslant L_d,\\[5pt]
\tau_d[L_d(t_d')]=t_d'\text{ для }0\leqslant t_d'\leqslant t_d.
\end{cases}
\eeq

Задача \eqref{FullTask} на $W_d(x_d,t_d)$ решена численно в Приложении A.


\subsection{Обсуждение результатов}

Результаты были получены тремя разными методами.
Численный метод из приложения A был применён к безразмерной задаче \eqref{FullTask}.
Зависимости длины трещины $L_d(t_d)$ и ширины трещины около скважины $W(0,t_d)$ от времени $t_d$ представлены на рис. \ref{fig:length-vs-time} и \ref{fig:width-at-well-vs-time}.

Приближённое решение в элементарных функциях выводится в приложении B при пренебрежении слагаемы $\frac{\partial A}{\partial t}$ в уравнении \eqref{Continuity}.
Это приближение справедливо на достаточно больших временах (как указано в приложении B) и подтверждается сравнением с численными результатами на рис. \ref{fig:length-vs-time} и \ref{fig:width-at-well-vs-time}.
При погрешности приближения менее 10\% при $t_d>1.0$.

\begin{figure}[H]
\center
\includegraphics[width=0.5\textwidth]{length_vs_time}
\caption{Безразмерная длина трещины в зависимости от безразмерного времени} 
\label{fig:length-vs-time}  
\end{figure}


\begin{figure}[H]
\center
\includegraphics[width=0.8\textwidth]{maximum_width_at_well_vs_time}
\caption{Максимальная безразмерная ширина трещины около скважины в зависимости от безразмерного времени} 
\label{fig:width-at-well-vs-time}  
\end{figure}

Случай отсутствия утечек жидкости в пласт рассматривается в приложении C.
Преобразование подобия в сочетании с численным методом приложения A даёт решение для любого значения времени.
Результаты приложения C также представляют собой приближённое решение для случая потери жидкости из трещины за достаточно малое время.
По сравнению с численным решением на рис. \ref{fig:length-vs-time} и \ref{fig:width-at-well-vs-time} ошибка приближения составляет менее 10\% при $t_d<0.01$.
Кроме того, $t_d$ должно быть достаточно большим, чтобы выполнялось условие $L\gg h$, как обсуждается ниже.

На рис. \ref{fig:width-vs-distance-from-well} показано изменение максимальной ширины трещины вдоль её длины (по координате $x$) в безразмерной форме для приближённых решений Приложения B и Приложения C.
Эти две зависимости ширины трещины от координаты $x$ удивительно похожи.
Зависимость ширины от координаты $x$, полученная численным методом, лежит близко к зависимостям на рис. \ref{fig:width-vs-distance-from-well} и не показана.

\begin{figure}[H]
\center
\includegraphics[width=0.8\textwidth]{maximum_width_vs_distance_from_well}
\caption{Максимальная безразмерная ширина трещины около скважины в зависимости от расстояния от скважины} 
\label{fig:width-vs-distance-from-well}
\end{figure}


Для любой конкретной задачи длину и ширину трещины в любой момент времени можно получить из рис. \ref{fig:length-vs-time}-\ref{fig:width-vs-distance-from-well} с помощью формул \eqref{Dimensionless}.
Можно просто вычислить значение $t_d$ из \eqref{Dimensionless}, найти $L_d(t_d)$ и $W_d(0,t_d)$ из рис.\ref{fig:length-vs-time} и \ref{fig:width-at-well-vs-time}, затем вычислить $L(t)$ и $W(0,t)$ из \eqref{Dimensionless}.
Затем примерное изменение максимальной ширины трещины по мере удаления от скважины следует из рис. \ref{fig:width-vs-distance-from-well}.
Кроме того, давление $\Delta p$ получается из $W$ по равенству \eqref{Max_W}.

Приближённые решения, представленные в приложениях B и C, дают существенное представление о влиянии физических параметров на длину, ширину и давление трещины.
Для случая большой скорости утечек или большого времени (Приложение B) длина трещины и максимальная ширина трещины около скважины определяются выражениями:
\beq\label{L_and_W_for_large_leak-off}
L(t)=\frac{q_i t^{1/2}}{\pi Ch}\,\,\,\,\,\,\,\,\,\,\,\,\,\,\,W(0,t)=4\left[\frac{2(1-\nu)\mu q_i^2}{\pi^3 GCh}\right]^{1/4}t^{1/8}
\eeq

Для случая отсутствия утечек или малого времени (Приложение C) эти же самые величины определяются выражениями:
\beq\label{L_and_W_without_leak-off}
L(t)=0.68\left[\frac{Gq_i^3}{(1-\nu)\mu h^4}\right]^{1/5}t^{4/5}\,\,\,\,\,\,\,\,\,\,\,\,\,\,\,W(0,t)=2.5\left[\frac{(1-\nu)\mu q_i^2}{Gh}\right]^{1/5}t^{1/5}
\eeq
Как видно из выражений \eqref{L_and_W_for_large_leak-off} и \eqref{L_and_W_without_leak-off} длина и ширина трещины растут быстрее со временем в случае отсутствия утечек (малое время), чем в случае больших утечек (большое время).
То, как другие параметры влияют на размер трещины, также видно из выражений \eqref{L_and_W_for_large_leak-off} и \eqref{L_and_W_without_leak-off}.
Например, $L$ пропорциональна $q_i$ в выражении \eqref{L_and_W_for_large_leak-off} и пропорциональна $q_i^{3/5}$ в выражении \eqref{L_and_W_without_leak-off}.
В выражении \eqref{L_and_W_for_large_leak-off} параметры $G$ и $\mu$ не влияют на длину $L$.

В приложении D проведено сравнение с приближённым решением Перкинса и Керна.
Установлено, что если выражения \eqref{L_and_W_for_large_leak-off} и \eqref{L_and_W_without_leak-off} свести к форме зависимостей $W(0,t)$ от $L$ (путём исключения времени $t$), то приближённое решение [6] довольно точно.
Тем не менее на ранних временах выражение \eqref{L_and_W_without_leak-off} для $L(t)$ отличается от формулы Картера [2], которая основана на предположении о постоянной ширине трещины.

На больших временах выражение \eqref{L_and_W_for_large_leak-off} для $L(t)$ согласуется с формулой Картера на больших временах [2], поскольку теперь доминирует эффект утечек жидкости в пласт.

Подробные исследования эффектов изменения параметров могут быть сделаны с помощью выражений \eqref{Dimensionless} и результатами на рисунках \ref{fig:length-vs-time} и \ref{fig:width-at-well-vs-time}, и эти исследования могут быть полезны при планировании дизайна ГРП.
Например, по этим результатам можно определить коэффициент утечек $C$, необходимый для достижения желаемой длины трещины после определённого времени закачки.
Можно определить давление закачки, необходимое для достижения желаемой скорости закачки.
Кроме того, трещина может быть спроектирована таким образом, чтобы раскрытие трещины пропускало проппант определённого размера.

Применяя изложенные выше результаты, необходимо иметь в виду основные предположения, сделанные при решении и постановке задачи.
Вероятно, наиболее важные предположения заключаются в том, что высота трещины ограничена по вертикали и что длина трещины намного больше высоты ($L\gg h$).
Как видно из рисунка \ref{fig:width-vs-distance-from-well}, последнее предположение обеспечивает медленное изменение ширины $w$ в зависимости от $x$, что подтверждает основное предположение о плоской деформации.
Если условие $L\gg h$, в частности сразу после образования трещины, не выполняется или если трещина не ограничена по вертикали, то более подходящими являются другие теоретические модели распространения трещины.
Для вертикальной трещины с $L\ll h$ и нагнетанием по всей высоте, модель плоской деформации в горизонтальной плоскости из [3,4,5] является более подходящей.
Если закачка ограничена коротким интервалом, то подходящей моделью должна быть радиально распространяющаяся вертикальная трещина, как предложили Перкинс и Керн [6].
Эта модель приблизительно была рассмотрена Гиртсмой, деКлерком [5] и другими.
Между решениями [3,4,5] и текущим решением есть существенные отличия.
В частности, для линейно распространяющейся (односторонней) вертикальной трещины приблизительная ширина трещины около скважины, данная Гиртсмой и деКлерком (для $\nu=0.25$), равна
$$
W(0)=2.1\left(\frac{\mu q_iL^2}{2Gh}\right)^{1/4},
$$
что отличается от нашего решения \eqref{L_and_W_for_large_leak-off} на больших временах на множитель $0.67\left(L/h\right)^{1/4}$.
Этот множитель велик, когда $L/h\ll1$ или $L/h\gg1$.
Как было сказано выше, наше решение неприменимо при $L/h<<1$.
С другой стороны, мы не считаем, что подход [3,4,5] верен при $L/h\gg 1$, поскольку предположение о вертикальной постоянной ширине в вертикальном направлении исключает смыкание трещины вверху и внизу.
Вносимая таким образом ошибка примерно пропорциональна $L$ и, таким образом, ошибка становится большой при $L\gg h$.
Напротив, ошибка, полученная в нашем решении с быстрыми изменениями вблизи кончика трещины примерно пропорциональна $h$, и эта ошибка становится малой при $L\gg h$.
Для $L\approx h$ необходимо трёхмерное решение, но в настоящее время оно отсутствует.
Такое решение также окончательно определило бы области применимости наших результатов и результатов работ [3,4,5].


\subsection{Список обозначений}
\setlength{\parindent}{0pt}
$A$ -- площадь поперечного сечения трещины\\
$C$ -- коэффициент утечек жидкости в пласт\\
$G$ -- объёмный модуль сдвига пласта\\
$h$ -- толщина пласта\\
$K$ -- константа (определяемая выражением \eqref{C-3})\\
$L$ -- длина трещины\\
$L_d$ -- безразмерная $L$\\
$p_f$ -- пластовое поровое давление\\
$q$ -- расход (объёмная скорость потока) жидкости через поперечное сечение трещины\\
$q_l$ -- скорость утечек на единицу длины трещины\\
$q_i$ -- постоянный расход жидкости в трещину на скважине\\
$S$ -- нормальное сжимающее напряжение плоскости трещины до гидроразрыва\\
$t$ -- время\\
$t_d$ -- безразмерное $t$\\
$u_l$ -- скорость утечек на единицу площади поверхности трещины\\
$v$ -- скорость жидкости\\
$W$ -- ширина трещины в центре трещины ($w_{max}$)\\
$W_d$ -- безразмерная $W$\\
$w$ -- ширина трещины\\
$x,y,z$ -- координаты в прямоугольной декартовой системе координат (рис. \ref{fig:kgd-model-geometry})\\
$x_d$ -- безразмерная координата $x$\\
$\Delta p$ -- давление в трещине минус $S$\\
$\mu$ -- вязкость жидкости гидроразрыва\\
$\nu$ -- объёмный коэффициент Пуассона пласта\\
$\tau(x)$ -- время начала утечек жидкости на расстоянии $x$ от скважины
\setlength{\parindent}{20pt}

\subsection{Список цитированной литературы}
\setlength{\parindent}{0pt}
[1] Howard, G.C. and Fast, C.R.: Hydraulic Fracturing, Monograph Series, Society of Petroleum Engineers, Dallas (1970) Vol. II.

[2] Carter, R.D.: "<Derivation of the General Equation for Estimating the Extent of the Fractured Area"> Appendix to: "<Optimum Fluid Characteristics for Fracture Extension"> by G. C. Howard and C. R. Fast, Drill. and Prod. Prac., API (1957) 261-270.

[3] Khristianovic, S.A. and Zheltov, Y.P.: "<Formation of Vertical Fractures by Means of Highly Viscous Liquid"> Proc., Fourth World Pet. Cong., Rome (1955) Sec. 2, 579-586.

[4] Baron, G. et al.: "<Fracturation hydraulique; bases the oriques, estudes de laboratorie, essais sur champ"> Proc., Seventh World Pet. Cong., Mexico City (1967) Sec. 3, 371.

[5] Geertsma, J. and de Klerk, F.: "<A Rapid Method of Predicting Width and Extent of Hydraulically Induced Fractures">, J. Pet. Tech. (Dec., 1969) 1571-1681.

[6] Perkins, T.K. and Kern, L.R.: "<Widths of Hydraulic Fractures">, J. Pet. Tech. (Sep., 1961) 937-949.

[7] Kern, L.R., Perkins, T.K and Wyant, R.E.: "<Designing Aluminum-Pellet Fracturing Treatments">, Drill. and Prod. Prac., API (1961).

[8] England, A.H. and Green, A.E.: "<Some Two-Dimensional Punch and Crack Problems in Classical Elasticity"> Proc., Cambridge Phil. Soc. (1963) Vol. 5, 489.

[9] Geertsma, J.: "<Problems of Rock Mechanics in Petroleum Engineering"> Proc., First Congress International Society of Rock Mechanics, Lisbon (1966) 585-594.

[10] Lamb, H.: Hydrodynamics, 6th Ed., Dover Publications, Inc., New York (1932).

[11] Hall C.D., Jr. and Dollarhide, F.E.: "<Performance of Fracturing Fluid Loss Agents Under Dynamic Conditions">, J. Pet. Tech. (July, 1968) 763-769.

[12] Ames, W.F.: Nonlinear Partial Differential Equations in Engineering, Academic Press, Inc., New York (1965).

\setlength{\parindent}{20pt}

\subsection{Приложение A. Численное решение}

Здесь мы получим конечно-разностный аналог уравнений \eqref{Initial_cond}-\eqref{GU_2} и дадим численный метод решения.
Чтобы получить конечно-разностные уравнения, удовлетворяющие общему условию баланса жидкости, начнём вывод с уравнения неразрывности \eqref{Continuity} вместе с выражениями \eqref{leak-offs}, \eqref{leak-off} и \eqref{gu1}, т.е.

\beq\label{A-1}
\frac{\partial q}{\partial x}+\frac{2hC}{\sqrt{t-\tau(x)}}+\frac{\partial A}{\partial t}=0,\,\,\,\,\,\,\,\,\,\,0<x<L
\tag{A-1}
\eeq

\beq\label{A-2}
q(0,t)=q_i
\tag{A-2}
\eeq

Интегрирование уравнения \eqref{A-1} по времени от $t_m$ до $t_{m+1}$ с использованием формулы трапеций даёт:
\beq\label{A-3}
\frac{\Delta t_m}{2}\frac{\partial}{\partial x}\left(q^{m+1}+q^m\right)+\left[4hC\sqrt{t-\tau(x)}+A\right]_{t_m}^{t_{m+1}}=0
\tag{A-3}
\eeq
где
$$
\Delta t_m=t_{m+1}-t_m\,\,\,\,\,,\,\,\,\,\,q^m=q(x,t_m).
$$

Интегрирование уравнения \eqref{A-3} по $x$ от $x_{i-1/2}$ до $x_{i+1/2}$ с использованием уравнения \eqref{A_area} и центрального значения для интеграла даёт:
\beq\label{A-4}
\frac{\Delta t_m}{2\Delta x}\left(q_{i+\frac{1}{2}}^{m+1}-q_{i-\frac{1}{2}}^{m+1}+q_{i+\frac{1}{2}}^{m}-q_{i-\frac{1}{2}}^{m}\right)+4hC\left(\sqrt{t_{m+1}-\tau_i}-\sqrt{t_m-\tau_{i}}\right)+\frac{\pi h}{4}\left(W_i^{m+1}-W_i^m\right)=0
\tag{A-4}
\eeq
где
$$
\Delta x=x_{i+\frac{1}{2}}-x_{i-\frac{1}{2}}\,\,\,,\,\,\,W_i^m=W(x_i,t_m)\,\,\,,\,\,\,x_i=\frac{1}{2}\left(x_{i+\frac{1}{2}}+x_{i-\frac{1}{2}}\right).
$$

Если мы примем $x_{\frac{1}{2}}=0$, то из уравнения \eqref{A-2}:
\beq\label{A-5}
q_{\frac{1}{2}}^m=q_i
\tag{A-5}
\eeq
и из уравнения \eqref{Continuity}:
\beq\label{A-6}
W_i^0=0
\tag{A-6}
\eeq

Конечно-разностный аналог уравнения \eqref{q_vs_w}:
\beq\label{A-7}
q_{i+\frac{1}{2}}=-\frac{\pi G}{256(1-\nu)\mu\Delta x}\left[\left(W_{i+1}^m\right)^4-\left(W_i^m\right)^4\right]
\tag{A-7}
\eeq

Результатом подстановки уравнения \eqref{A-7} в уравнение \eqref{A-4} является нелинейная система уравнений с неизвестным $W_i^{m+1}$ и с считаемым известным $W_i^m$.
Чтобы решить эту систему запишем:
\beq\label{A-8}
W_i^{m+1}=W_i^m+\Delta W_i^m
\tag{A-8}
\eeq
и примем, что
\beq\label{A-9}
\left(W_i^{m+1}\right)^4\approx\left(W_i^m\right)^4+4\left(W_i^m\right)^3\Delta W_i^m
\tag{A-9}
\eeq

При подстановке уравнений \eqref{A-6}, \eqref{A-8} и \eqref{A-9} в уравнение \eqref{A-4} мы получим алгебраическую линейную трёхдиагональную систему уравнений, которая может быть просто решена для $\Delta W_i^m$ при $m=1,2,...$. 

\subsection{Приложение B. Решение при доминировании утечек жидкости}

Здесь мы предполагаем, что слагаемое $\partial A/\partial t$ в уравнении \eqref{Continuity} пренебрежимо мало по сравнению со скоростью утечек $q_l$.
Тогда с использованием выражений \eqref{leak-offs} и \eqref{leak-off} уравнение неразрывности \eqref{Continuity} перепишется в следующем виде:
\beq\label{B-1}
\frac{\partial q}{\partial x}=-\frac{2hC}{\sqrt{t-\tau(x)}}
\tag{B-1}
\eeq
где $\tau$ зависит от $x$ так, чтобы выполнялось равенство
\beq\label{B-2}
\tau\left[L(t')\right]=t'\,\,\,,\,\,\,0\leqslant t'\leqslant t
\tag{B-2}
\eeq

Интегрирование уравнения \eqref{B-1} с использованием граничного условия \eqref{gu1} даёт
\beq\label{B-3}
q=q_i-2hC\int\limits_{0}^{x}{\frac{dx}{\sqrt{t-\tau(x)}}}
\tag{B-3}
\eeq

Так как из кончика трещины нет потока $\left[q(L,t)=0\right]$, из уравнения \eqref{B-3} получаем
\beq\label{B-4}
\int\limits_{0}^{L}{\frac{dx}{\sqrt{t-\tau(x)}}}=\frac{q_i}{2hC}
\tag{B-4}
\eeq

Решение уравнения \eqref{B-4} с использованием уравнения \eqref{B-2} даёт
\beq\label{B-5}
\tau=\left(\frac{\pi hCx}{q_i}\right)^2\,\,\,,\,\,\,L=\frac{q_it^{\frac{1}{2}}}{\pi Ch},
\tag{B-5}
\eeq
которое было ранее получено в [2, 5, 9] несколько иным образом.

С использованием уравнения \eqref{B-5}, уравнение \eqref{B-3} перепишется в следующем виде:
\beq\label{B-6}
q=q_i\left[1-\frac{2}{\pi}\sin^{-1}{\zeta}\right]\,\,\,,\,\,\,\zeta=\frac{x}{L}
\tag{B-6}
\eeq

Из уравнений \eqref{q_vs_w}, \eqref{B-6} и условия $W(L,t)=0$ получаем
\beq\label{B-7}
W=4\left[\frac{2(1-\nu)\mu q_i^2}{\pi^3 CGh}\right]^{\frac{1}{4}}t^{\frac{1}{8}}\phi(\zeta),
\tag{B-7}
\eeq
где
\beq\label{B-8}
\phi(\zeta)=\left[\frac{W(x,t)}{W(0,t)}\right]_{x=L\zeta}=\left[\zeta\sin^{-1}{\zeta}+(1-\zeta^2)^{\frac{1}{2}}-\frac{\pi}{2}\zeta\right]^{\frac{1}{4}}
\tag{B-8}
\eeq

Функция $\phi(\zeta)$ представлена на рис. \ref{fig:width-vs-distance-from-well}.
Из выражений \eqref{Max_W}, \eqref{B-5} и \eqref{B-7} максимальное раскрытие трещины и давление в центре трещины имеют вид:
\beq\label{B-9}
W(0,t)=4\left[\frac{2(1-\nu)\mu q_i^2}{\pi^3 GCh}\right]^{\frac{1}{4}}t^{\frac{1}{8}}\,\,\,\,\,\,\,\,;\,\,\,\,\,\,\,\,\Delta p(0,t)=4\left[\frac{2G^3\mu q_i^2}{\pi^3(1-\nu)^3 Ch^5}\right]^{\frac{1}{4}}t^{\frac{1}{8}}
\tag{B-9}
\eeq

Имея решение, мы можем проверить слагаемое $\partial A/\partial t$, которым мы пренебрегли при выводе.
Из выражений \eqref{A_area}, \eqref{leak-offs}, \eqref{leak-off} и \eqref{B-9}:
\beq\label{B-10}
\left[\frac{\partial A}{\partial t}/q_l\right]_{x=0}=\frac{1}{16}\left[\frac{2\pi(1-\nu)\mu q_i^2}{GC^5h}\right]^{\frac{1}{4}}t^{-\frac{3}{8}}
\tag{B-10}
\eeq

Таким образом, $\partial A/\partial t$ пренебрежимо мало по сравнению с $q_l$ при $x=0$ (и вероятно на большей части трещины) на достаточно больших временах $t$.

Выражение той же формы, что и \eqref{B-10} может быть получено для
$$
\int\limits_{0}^{L}{\frac{\partial A}{\partial t}dx/q_i}.
$$

\subsection{Приложение C. Решение при отсутствии утечек жидкости в пласт}

В этом разделе мы рассмотрим случай отсутствия утечек жидкости из трещины в пласт, т.е.
\beq\label{C-1}
q_l=0
\tag{C-1}
\eeq

Подстановка \eqref{q_vs_w}, \eqref{A_area} и \eqref{C-1} в уравнение неразрывности \eqref{Continuity} даёт
\beq\label{C-2}
K\frac{\partial^2}{\partial x^2}\left(W^4\right)=\frac{\partial W}{\partial t},
\tag{C-2}
\eeq
где
\beq\label{C-3}
K=\frac{G}{64(1-\nu)\mu h}
\tag{C-3}
\eeq

С учётом уравнения \eqref{q_vs_w} граничное условие \eqref{gu1} перепишется в следующем виде:
\beq\label{C-4}
-\left[\frac{\partial}{\partial x}\left(W^4\right)\right]_{x=0}=\frac{4q_i}{\pi Kh}
\tag{C-4}
\eeq

Нелинейное дифференциальное уравнение в частных производных \eqref{C-2} вместе с граничными условиями \eqref{GU_1}, \eqref{C-4} и начальным условием \eqref{Initial_cond} может быть решено численным методом, представленным в приложении A.

Чтобы свести к минимуму количество вычислений и дать представление о влиянии различных параметров, будет получено автомодельное решение (решения такого вида обсуждаются Ames-ом в работе [12]).
Ищем решение вида
\beq\label{C-5}
W=\alpha\,\psi(\xi)\,t^m\,\,\,\,\,,\,\,\,\,\,\xi=\beta xt^n,
\tag{C-5}
\eeq
где константы $\alpha$, $\beta$, $m$, $n$ и функцию $\psi(\xi)$ нужно найти.
Подстановка выражения \eqref{C-5} в \eqref{C-2}, \eqref{C-4}, \eqref{Initial_cond} и \eqref{GU_1} приводит к уравнениям:
\beq\label{C-6}
\begin{split}
\frac{d^2}{d\xi^2}\left(\psi^4\right)=\frac{1}{5}\left(\psi-4\xi\frac{d\psi}{d\xi}\right),\,\,\xi>0;\\[4pt]
-\left[\frac{d}{d\xi}\left(\psi^4\right)\right]_{\xi=0}=\frac{2}{\pi};\,\,\,\,\,\,\,\,\,\,\lim_{\xi\to\infty}{\psi(\xi)}=0
\end{split}
\tag{C-6}
\eeq
с
\beq\label{C-7}
\begin{split}
\alpha=\left(\frac{4q_i^2}{h^2K}\right)^{\!\!\frac{1}{5}};\,\,\,\,\,\,\,\,\,\,\beta\left(\frac{h^3}{8Kq_i^3}\right)^{\!\!\frac{1}{5}};\,\,\,\,\,\,\,\,\,\,m=\frac{1}{5};\,\,\,\,\,\,\,\,\,\,n=-\frac{4}{5}.
\end{split}
\tag{C-7}
\eeq
Функция $\psi(\xi)$ может быть получена численно из системы уравнений \eqref{C-6}.
Альтернативно, исходная задача \eqref{C-2}, \eqref{C-4}, \eqref{Initial_cond} и \eqref{GU_1} может быть решена численно относительно $W$ и затем $\psi(\xi)$ может быть получена из \eqref{C-5} и \eqref{C-7}.
Используя последний подход, с помощью метода приложения A находим $\psi(\xi)$ и
\beq\label{C-8}
\begin{split}
\psi(\xi^*)=0\text{ при }\xi^*=1.01\\
\psi(0)=0.83\,\,\,\,\,\,\,\,\,\,\,\,\,\,\,\,\,\,\,
\end{split}
\tag{C-8}
\eeq

Таким образом, из уравнений \eqref{C-3}, \eqref{C-5} и \eqref{C-7} длина трещины
\beq\label{C-9}
L=0.68\left[\frac{Gq_i^3}{(1-\nu)\mu h^4}\right]^{\frac{1}{5}}t^{\frac{4}{5}}
\tag{C-9}
\eeq
с
$$
\frac{x}{L}=\frac{\xi}{\xi^*}
$$

Функция $\psi(\beta xt^{-4/5})/\psi(0)=W(x,t)/W(0,t)$ изображена на рис. \ref{fig:width-vs-distance-from-well}.
Из \eqref{Max_W}, \eqref{C-3}, \eqref{C-5} и \eqref{C-7} максимальное раскрытие и давление в центре трещины:
\beq\label{C-10}
W(0,t)=2.5\left[\frac{(1-\nu)\mu q_i^2}{Gh}\right]^{\!\frac{1}{5}}t^{\frac{1}{5}}\,\,\,\,\,\,\,;\,\,\,\,\,\,\,\Delta p(0,t)=2.5\left[\frac{G^4\mu q_i^2}{(1-\nu)^4h^6}\right]^{\frac{1}{5}}t^{\frac{1}{5}}
\tag{C-10}
\eeq

Из \eqref{C-10} имеем:
\beq\label{C-11}
\frac{\partial W}{\partial t}\sim t^{-\frac{4}{5}}
\tag{C-11}
\eeq

Таким образом, при $t\to0$ слагаемое $\partial W/\partial t$ доминирует над слагаемым $8C$ в \eqref{Main_eq} за исключением области рядом с $x=L$.
Следовательно, полученное решение с $C=0$ будет хорошим приближением к решению при $C\neq 0$ на достаточно малых временах.

\subsection{Приложение D. Приближения Перкинса и Керна}

Для простоты сравнения мы записываем здесь приближенное решение Перкинса и Керна [6].
При пренебрежении утечками и скоростью изменения объёма трещины в уравнении неразрывности \eqref{Continuity} правая часть уравнения \eqref{Main_eq} равна нулю и интегрирование с использованием \eqref{15} и \eqref{GU_2} даёт
\beq\label{D-1}
W=4\left[\frac{(1-\nu)\mu q_1L}{\pi G}\left(1-\frac{x}{L}\right)\right]^{1/4}
\tag{D-1}
\eeq

В этом выражении неизвестна длина $L$.
Более того, скорость потока $q(x)$ равна $q_i$ везде, включая $x=L$, что неправдоподобно.
Для сравнения, наше решение на больших временах при исключении $C$ из \eqref{B-5}-\eqref{B-7} (путём подстановки одного в другое) даёт
\beq\label{D-2}
W=4\left[\frac{2(1-\nu)\mu q_iL}{\pi^2G}\right]^{1/4}\phi\left(\frac{x}{L}\right),
\tag{D-2}
\eeq
где $\phi(\zeta)$ определяется выражением \eqref{B-8}.
Таким образом, \eqref{D-1} отличается от \eqref{D-2} при $x=0$ на множитель $0.89$.

На малых временах приближённое решение для $L$ может быть получено из общего уравнения неразрывности, полученного путём интегрирования \eqref{Main_eq} с $C=0$ и $\partial W/\partial t$ из \eqref{D-1}.
Таким образом, мы получаем \eqref{C-9} с заменой числового коэффициента на $0.60$.

Кроме того, поправка к \eqref{D-1} может быть получена из \eqref{Main_eq} с $\partial W/\partial t$, рассчитанном по \eqref{D-1}.

При $x=0$ скорректированное значение $W$ отличается от \eqref{D-1} в $0.94$ раза.

Таким образом, результат Перкинса и Керна \eqref{D-1} для ширины трещины в терминах её длины является хорошим приближением на ранних временах.
Однако длина должна быть определена из \eqref{C-9}, а не по формуле Картера [2].

\subsection{Приложение E. Влияние ступенчатых утечек на длину трещины}

Заменим выражение \eqref{leak-off} выражением
\beq\label{E-1}
q_l=\frac{C}{\sqrt{t-\tau}}+V_s\delta(t-\tau),
\tag{E-1}
\eeq
где $V_s$ -- внезапная утечка (объём/площадь), происходящая мгновенно в момент времени $t=\tau$ и $\delta(t-\tau)$ -- дельта-функция Дирака.
Мы следуем методу приложения B и получаем вместо \eqref{B-4}:
\beq\label{E-2}
C\int\limits_{0}^{t}{\frac{dL}{d\tau}}\frac{d\tau}{\sqrt{t-\tau}}+V_s\frac{dL}{dt}=\frac{q_i}{2h}.
\tag{E-2}
\eeq

Уравнение \eqref{E-2} может быть решено с помощью преобразования Лапласа.
Результат:
\beq\label{E-3}
L=\frac{q_i}{\pi Ch}\left[t^{\frac{1}{2}}+\frac{V_s}{2C}\left(L^{b^2t}\,\textrm{erfc}\!\left(b\sqrt{t}\right)-1\right)\right],
\tag{E-3}
\eeq
где
$$
b=\frac{\pi^{\frac{1}{2}}C}{V_s}.
$$

Так как
\beq\label{E-4}
e^{b^2t}\textrm{erfc}\!\left(b\sqrt{t}\right)=\frac{1}{\pi^{\frac{1}{2}}b}\left(1-\frac{1}{2b^2t}+...\right)t^{-\frac{1}{2}}
\tag{E-4}
\eeq
слагаемое с $V_s$ в \eqref{E-3} пренебрежимо мало при
\beq\label{E-5}
t\gg(V_s/C)^2,
\tag{E-5}
\eeq
что является небольшим временем для многих жидкостей гидроразрыва.

\newpage

\section{Перевод статьи Перкинса и Керна (Ширина трещин гидроразрыва)}

\subsection{Реферат}

Механика жидкости, механика разрушения хрупких материалов и теория упругой деформации горных пород показывают, что для данного пласта ширина трещины в основном определяется перепадом давления жидкости в трещине.
Режимы работы, вызывающие большой перепад давления вдоль трещины (например, высокая скорость закачки и вязкие жидкости), приведут к относительно широким трещинам.
И наоборот, режимы работы, которые вызывают низкий перепад давления (низкие скорости закачки и слабовязкие жидкости), приведут к относительно узким трещинам.

В данной работе были получены графики и уравнения, которые позволяют оценить ширину как горизонтальных, так и вертикальных трещин при различных условиях потока.

\subsection{Введение}

Есть много предположений относительно геометрии гидравлически созданных трещин в земной коре.
Один из практически значимых вопросов –- это ширина трещины в динамических условиях, т. е. при создании и распространении (расширении) трещины.
Эту информацию о ширине можно использовать, например, для оценки площади трещины, образовавшейся в различных условиях [10].
Кроме того, в последнее время наблюдается тенденция к использованию крупных расклинивающих частиц [13, 15].
Поэтому желательно знать, какие параметры (факторы) можно варьировать, чтобы обеспечить попадание крупных частиц в трещину.

В литературе сообщается о некоторых работах по исследованию ширины трещин.
В частности, было несколько российских публикаций, посвященных этой теме [1, 31, 32].
Эти работы касались в основном теории упругости и применения этой теории к трещинам гидроразрыва.
Эти исследования не привели к инженерному методу для оценки ширины трещин в динамических условиях.
В недавней статье [3] были рассмотрены и обобщены российские концепции.

В более ранней статье [9] наших лабораторий также обсуждалось применение теории упругости к трещинам ГРП.
Этот первый подход, основанный в основном на фотоупругих исследованиях, оказался слишком упрощенным для точного описания процесса разрушения.
Однако эти ранние мысли послужили руководством при разработке более точных концепций.

Мы хотели бы представить в этой статье наши текущие представления о ширине трещин и некоторые оценки ширины трещины гидроразрыва для нескольких условий.
Мы считаем, что теперь можно с достаточной точностью прогнозировать факторы, влияющие на ширину трещин.
Кроме того, метод прогнозирования сводится к простому и удобному графическому или численному расчету.

\subsection{Трещины в хрупком упругом материале}

Многие исследователи [2, 4, 30] показали, что прочные горные породы ведут себя упруго в некотором диапазоне напряжений.
Конечно, если растягивающее напряжение, воздействующее на горную породу, превышает некоторое предельное значение, то горная порода разрушится при растяжении.
Аналогичным образом существуют некоторые предельные касательные напряжения, которые могут быть приложены к горным породам.
Хабберт и Уиллис [11] обсудили условия сдвига, которые приведут к разрушению.

В условиях умеренного напряжения (таких, которые могут возникнуть при гидравлическом разрыве пласта) и при относительно быстром воздействии напряжения большинство горных пород разрушается хрупким образом.
Следовательно, для этого обсуждения трещин ГРП в земной коре мы предполагаем, что горные породы ведут себя как хрупкие упругие материалы.

Построим обсуждение следующим образом (следующие мысли применимы только к хрупким материалам).

1. Сначала мы рассмотрим хрупкую эластичную систему.
Энергетический баланс покажет минимальное давление, необходимое для разрушения горной породы, и по этому давлению мы вычислим минимальную ширину трещины в результате распространения трещины гидроразрыва.

2. Затем мы покажем, что в обычных условиях трещиноватости ширина трещин заметно больше, чем минимальная ширина распространяющихся трещин.
Фактически мы обнаружим, что ширина трещины контролируется перепадом давления жидкости в трещине.

3. Мы обсудим перепады давления в трещинах и результирующую ширину трещины для различных условий эксплуатации как для вертикальных, так и для горизонтальных трещин.

4. Наконец, мы обсудим значение этих представлений, их связь с давлением гидроразрыва и т. д.

Во-первых, рассмотрим минимальные давления расширения трещины.
Мы можем пролить некоторый свет на этот вопрос, рассмотрев теорию, предложенную Гриффитом [7,8] для объяснения разрыва хрупких эластичных материалов.
Гриффит признал, что твердые материалы обладают поверхностной энергией (подобной поверхностному натяжению в жидкости).
Фундаментальная концепция теории Гриффита состоит в том, что при распространении трещины без приложения внешней работы (внутри упругой среды, находящейся под внешним напряжением) уменьшение энергии деформации в результате упругой деформации вблизи трещины уравновешивается увеличением поверхностной энергии.

Аналогичный подход при приложении внешней работы закачкой жидкости можно использовать для оценки ширины трещины гидроразрыва в горных породах в статических условиях (когда движение жидкости прекратилось).
Рассмотрим бесконечную упругую среду, содержащую плоскую трещину, ограниченную окружностью -- трещина в форме монетки.
Если бы жидкость была закачана в эту ранее существовавшую трещину, но при давлении меньшем, чем необходимо для распространения трещины в длину, тогда трещина была бы "<раздута">.
Для идеально упругой среды связь между формой трещины и давлением внутри трещины была рассчитана Снеддоном [22].
На рис. \ref{fig:Perkins1} показан концептуальный набросок с некоторыми геометрическими соотношениями, предсказанными теорией упругости.

\setcounter{figure}{0}
\begin{figure}[H]
\center
\includegraphics[width=0.57\textwidth]{Perkins_1}
\caption{Набросок неограниченной трещины с геометрией, предсказанной теорией упругости}
\label{fig:Perkins1}
\end{figure}

Теперь давайте предположим, что давление в трещине увеличивается до тех пор, пока трещина не будет готова распространяться по радиусу.
Пусть объем $dV$ закачивается при давлении распространения трещины $P$, и предположим, что это приводит к увеличению радиуса трещины $dC$.
Объем работы, выполненной для закачки жидкости в трещину, явно равен $PdV$.
Количество энергии, запасенной в системе трещин, возникает из-за (1) увеличения потенциальной энергии упругой среды и (2) увеличения поверхностной энергии, вызванного увеличением радиуса трещины.
Для получения минимального давления распространения трещины совершаемая работа приравнивается к энергии, запасенной в системе трещины.
Используя этот подход, Сэк [21] вывел уравнение, которое дает минимальное давление, необходимое для распространения трещины в горной породе:
\beq\label{Perkins_1}
P_m-\sigma=\sqrt{\frac{\pi\alpha E}{2\left(1-\nu^2\right)C}}\,\,\,,
\tag{1}
\eeq
где $P_m$ -- минимальное давление распространения трещины, psi (фунт на квадратный дюйм),\newline
$\sigma$ -- общее напряжение в породе, перпендикулярное плоскости трещины, psi (общее напряжение грунта определяется как сумма напряжения в скелете горной породы и порового давления, см. [11]),\newline
$\alpha$ -- удельная поверхностная энергия породы, фут-фунт/кв. дюйм,\newline
$E$ -- модуль Юнга породы, psi,\newline
$\nu$ -- коэффициент Пуассона для породы,\newline
$C$ -- радиус трещины, футы.

Это уравнение показывает, что для данной породы (т. е. фиксированных значений поверхностной энергии, модуля Юнга и коэффициента Пуассона) минимальное давление распространения трещины (превышающее напряжение грунта) изменяется обратно пропорционально квадратному корню из радиуса трещины.

\subsection{Экспериментальная проверка подхода Гриффита-Снеддона}

Подход Гриффита-Снеддона к минимальному давлению распространения трещины можно проверить экспериментально.
Снеддон показал, что объем радиально-симметричной трещины при постоянном давлении $P$, действующем в трещине, определяется выражением \eqref{Perkins_2}.
\beq\label{Perkins_2}
V=\frac{16\left(1-\nu^2\right)C^3\left(P-\sigma\right)}{3E},
\tag{2}
\eeq
где $V$ -- объём трещины.

Комбинируя выражение \eqref{Perkins_2} с выражением \eqref{Perkins_1}, чтобы исключить $C$, получаем выражение \eqref{Perkins_3}.
\beq\label{Perkins_3}
\left(P-\sigma\right)=\left[\frac{2\pi^3\alpha^3E^2}{3\left(1-\nu^2\right)^2V}\right]^{1/5}
\tag{3}
\eeq

Следовательно, давление распространения трещины должно изменяться обратно пропорционально корню пятой степени из объема трещины.

Мы измерили давление распространения трещины и объем трещины для модели из портландцемента.
Полностью исключить утечку жидкости в модели не удалось.
Однако она была минимальна за счет использования жидкости с низкими фильтрующими свойствами и цемента с низкой проницаемостью.
Объем просачивающейся жидкости был оценен и вычтен из общего объема закачки для оценки объема жидкости, оставшейся в трещине.
Модель была устроена таким образом, чтобы можно было измерить давление, действующее внутри трещины.

\begin{figure}[H]
\center
\includegraphics[width=0.57\textwidth]{Perkins_2}
\caption{Набросок модели из портландцемента и зависимость давления распространения от объёма трещины}
\label{fig:Perkins2}
\end{figure}

На рис. \ref{fig:Perkins2} показан эскиз модели и график зависимости $\log P$ от $\log V$.
Через данные проведена линия с наклоном "<$-1/5$">, что подтверждает взаимосвязь, показанную выражением \eqref{Perkins_3}.

\subsection{Минимальное давление раскрытия трещины}

Чтобы предсказать минимальные давления распространения трещины для полевых случаев, мы должны определить типичные физические свойства горных пород.

Значения коэффициентов Пуассона для горных пород, о которых сообщается в литературе [2, 4, 30], колеблются примерно от 0.05 до 0.25.
Поскольку минимальное давление распространения трещины не очень чувствительно к изменению коэффициента Пуассона, использование среднего значения 0.15 представляется оправданным.

Модули Юнга для горных пород изменяются в гораздо более широком диапазоне.
На модуль Юнга влияет тип породы, пористость и среднее напряжение.
В таблице \ref{tab:Perkins_table_1} приведены диапазоны модулей, которые согласуются с большинством литературных данных [2, 4, 30].

\begin{table}[htbp]
	\centering\small
	\caption{Оценки модулей Юнга для различных пород}
	\label{tab:Perkins_table_1}
		\begin{tabular}{|c|c|}
			\hline
			Тип породы&Вероятное значение модуля Юнга, $10^6$ psi\\
			\hline
			Пористые, рыхлые и слабосцементированные пески&от 0.5 до 1.5\\ \hline
			Пески средней твёрдости&от 2 до 4\\ \hline
			Твёрдый и плотный песчаник&от 5 до 7.5\\ \hline
			Известняк и доломит&от 8 до 13\\ \hline
		\end{tabular}
	\normalsize
\end{table}

Данных о поверхностной энергии горных пород недостаточно.
Горное бюро США [26] сообщило одни из лучших данных.
Они сообщают значение 0.0265 фут-фунт/кв. дюйм для поверхностной энергии кристаллов кварца и 0.00613 фут-фунт/кв. дюйм для поверхностной энергии кристаллов кальцита ($\text{CaCO}_3$).

Кажущаяся поверхностная энергия при разрушении, вероятно, соответствует поверхностной энергии цементированного материала и фактически разрушенных зерен кварца.
Однако из-за пористой природы породы новая поверхность, фактически созданная во время расщепления, составляет менее 1 кв. дюйма/кв. дюйм поверхности трещины.
Следовательно, кажущаяся поверхностная энергия во время типичного гидроразрыва оценивается не более чем в 0.01 фут-фунт/кв. дюйм и, возможно, намного меньше.

Теперь можно оценить минимальные давления, необходимые для распространения средней трещины.
Возьмем упрощенный случай осесимметричного разрушения (трещина в виде копейки).
Предположим, что следующие свойства являются типичными: $E=4\cdot10^6\text{ psi}$, $\alpha=0.01\text{ ft-lb/sq in.}$ (или меньше) и $\nu=0.15$.

Минимальное давление распространения трещины можно рассчитать по формуле \eqref{Perkins_1}, а соответствующую максимальную ширину трещины в стволе скважины можно рассчитать по формуле \eqref{Perkins_4}.
Выражение \eqref{Perkins_4} получено Снеддоном [22] для случая трещины в форме копейки.
\beq\label{Perkins_4}
W_m=\frac{8\left(P-\sigma\right)\left(1-\nu^2\right)C}{\pi E},
\tag{4}
\eeq
где $W_m$ -- максимальное раскрытие трещины на скважине.

Рассчитанные значения представлены в зависимости от радиуса трещины на рис. \ref{fig:Perkins3}.
Эти кривые показывают, что для рассматриваемого состояния статической жидкости в трещине гидроразрыва ширина трещины очень мала.

\begin{figure}[H]
\center
\includegraphics[width=0.57\textwidth]{Perkins_3}
\caption{Минимальное давление распространения трещины и максимальная ширина для типичной неограниченной трещины}
\label{fig:Perkins3}
\end{figure}

\subsection{Ширина трещины в динамических условиях}

Из предыдущего обсуждения ясно, что в статических условиях (и при условии отсутствия утечки флюида) трещины будут очень узкими.
Если бы жидкость закачивалась с разумной скоростью в эти узкие трещины, давление закачки должно было бы быть чрезвычайно большим.
Однако результирующие высокие давления флюидов в трещине заставят стенки трещины раздвинуться дальше друг от друга.
По мере увеличения ширины трещины давление, необходимое для нагнетания жидкости, будет уменьшаться.
Фактически достигается состояние равновесия.
Результирующая ширина трещины контролируется, в основном, перепадом давления жидкости в трещине.
Давление жидкости на передней кромке трещины почти равно противодействующему напряжению грунта (давление, превышающее напряжение грунта на передней кромке трещины, обычно пренебрежимо мало, как мы только что видели в предыдущем разделе, поэтому в оставшейся части этой статьи оно принимается равным нулю).
Это концептуально показано на рис. \ref{fig:Perkins4} (аналогичный эскиз можно нарисовать для горизонтальной трещины).
Условия эксплуатации, которые приводят к высокому перепаду давления вдоль трещины (например, высокие скорости закачки и вязкие жидкости), приведут к относительно широким трещинам.
И наоборот, рабочие условия, которые приводят к низкому перепаду давления (низкие скорости закачки и слабовязкие жидкости), приведут к относительно узким трещинам.

\begin{figure}[H]
\center
\includegraphics[width=0.57\textwidth]{Perkins_4}
\caption{Набросок давлений при гидроразрыве пласте}
\label{fig:Perkins4}
\end{figure}

Поскольку ширина трещины определяется перепадом давления в трещине, мы должны рассмотреть несколько управляющих ситуаций.
В данной статье мы обсудим следующие случаи.

1. Вертикально ориентированные, вертикально ограниченные трещины, возникающие при:
(a) течении ньютоновских жидкостей при ламинарном режиме течения;
(b) течении ньютоновской жидкости при турбулентном режиме течения; 
(c) течении неньютоновских жидкостей при ламинарном режиме течения;
и (d) когда большое количество песка в трещине ограничивает поток флюида и тем самым увеличивает перепад давления вдоль трещины.

2. Горизонтально ориентированные осесимметричные трещины при ламинарном течении ньютоновских жидкостей -- включая: (a) трещины, расположенные глубоко в земле; (b) очень мелкие трещины.

Конечно, можно рассмотреть и другие условия.
Однако мы считаем, что эти случаи охватывают большинство ситуаций и в целом демонстрируют ожидаемое поведение
(например, ширина неограниченных вертикальных трещин будет примерно такой же, как и у горизонтальных трещин того же размера).

\subsection{Вертикальные трещины}

В породе часто существуют условия, которые ограничивают рост вертикальных трещин в вертикальном направлении.
Зоны с более высоким горизонтальным напряжением, чем в продуктивной зоне, иногда находятся выше и ниже продуктивной зоны и, например, вызывают эффект ограничения по вертикали
(иногда в сланцах обнаруживаются высокие горизонтальные напряжения; кроме того, в известняковых коллекторах горизонтальные напряжения в непроницаемых участках могут быть выше, чем в проницаемых после снижения пластового давления).
Трещина будет расти до тех пор, пока не достигнет граничной зоны, а затем будет ограничена в вертикальном росте.
Тем не менее, она будет продолжать расширяться в боковом направлении от ствола скважины, как показано на рис. \ref{fig:Perkins5}.
Трещина будет проникать вверх и вниз в граничные зоны до тех пор, пока не будет достигнуто состояние равновесия.
Это состояние равновесия аналогично состоянию рычага в статическом равновесии.
Жидкость для гидроразрыва оказывает давление на поверхность трещины, стремясь открыть и расширить трещину.
Высокие напряжения вверху и внизу имеют тенденцию к закрытию трещины.
Таким образом, трещина будет распространяться в ограничивающие зоны до тех пор, пока противодействующие силы не уравновесят друг друга.
Если эти ограничивающие зоны недостаточно толстые или если перепад давления в трещине становится достаточно высоким, то трещина может распространиться в следующие (другие) зоны.

\begin{figure}[H]
\center
\includegraphics[width=0.57\textwidth]{Perkins_5}
\caption{Набросок ограниченной вертикальной трещины (нарисовано без сохранения масштабов)}
\label{fig:Perkins5}
\end{figure}

\subsubsection{Ширина трещин, образующихся в результате течения ньютоновских жидкостей при ламинарном течении}

К счастью, довольно легко предсказать, будет ли ньютоновская жидкость течь ламинарно или турбулентно.
В классической работе [19] Рейнольдс обнаружил, что турбулентность начинается тогда, когда число Рейнольдса (отношение сил вязкости к силам инерции в жидкости) превышает определенное значение.
В условиях гидроразрыва число Рейнольдса может быть уменьшено до (см. Приложение A):
$$
N_{Re}=7.81\cdot 10^3\,\frac{(Q)(\text{Sp. Gr.})}{(H)\,(\mu)},
$$
где
$N_{Re}$ -- число Рейнольдса,\newline
$Q$ -- закачиваемый расход (баррель в минуту),\newline
$\text{Sp. Gr.}$ -- удельный вес жидкости гидроразрыва,\newline
$H$ -- высота трещины (футов),\newline
$\mu$ -- вязкость жидкости гидроразрыва (сантиПуаз).

Таким образом, когда $(Q)(\text{Sp. Gr.})/(H)(\mu)$ меньше, чем примерно 0.32, поток жидкости в трещине будет ламинарным.

Выражение для ширины трещины примет следующий вид (вывод в приложении A):
\beq\label{Perkins_5}
W=0.38\left[\frac{(Q)(\mu)(L)}{E}\right]^{1/4},
\tag{5}
\eeq
где
$W$ -- максимальная ширина трещины около скважины (дюймов),\newline
$Q$ -- закачиваемый расход (баррель в минуту),\newline
$\mu$ -- эффективная вязкость жидкости гидроразрыва (сантиПуаз),\newline
$L$ -- длина вертикальной трещины, измеренная от ствола скважины (футов),\newline
$E$ -- модуль Юнга породы пласта (psi).

Используя диапазон значений модулей Юнга из таблицы \ref{tab:Perkins_table_1}, выражение \eqref{Perkins_5} представлено на рис. \ref{fig:Perkins6} в удобном графическом виде.

\begin{figure}[H]
\center
\includegraphics[width=0.87\textwidth]{Perkins_6}
\caption{Ширина ограниченных вертикальных трещин при течении ньютоновских жидкостей и ламинарном режиме потока}
\label{fig:Perkins6}
\end{figure}

В качестве примера предположим, что условия при проведении работ по гидроразрыву пласта были следующими:
$Q=30\text{ баррелей в минуту}$, $\mu=4\text{сП}$, $L=500\text{ футов}$, $\text{Sp. Gr.}=0.9$ и $H=40\text{ футов}$.

$$
\frac{Q(\text{Sp. Gr.})}{H\mu}=\frac{(30)(0.9)}{(40)(4)}=0.17
$$
(таким образом, поток ламинарный и можно использовать рис. \ref{fig:Perkins6}).

$$
Q\mu L=(30)(4)(500)=6\cdot 10^4.
$$
Таким образом, рис. \ref{fig:Perkins6} показывает, что максимальная ширина трещины около скважины будет около $0.13-0.15$ дюйма.

\subsubsection{Ширина трещин, образующихся в результате течения ньютоновских жидкостей при турбулентном течении}

Если $(Q)(\text{Sp. Gr.})/(H)(\mu)$ больше, чем $0.32$, то поток жидкости в трещине будет турбулентным.
В этом случае ширина трещины равна (вывод в приложении B):
\beq\label{Perkins_6}
W=0.6\left[\frac{(Q)^2(\text{Sp. Gr.})(L)}{(E)(H)}\right]^{1/4},
\tag{6}
\eeq
где $\text{Sp. Gr.}$ -- удельный вес жидкости гидроразрыва;\newline
$H$ -- высота вертикальной трещины (футов).

Снова используя значения модулей Юнга из таблицы \ref{tab:Perkins_table_1}, выражение \eqref{Perkins_6} можно изобразить графически, как на рис. \ref{fig:Perkins7}.

\begin{figure}[H]
\center
\includegraphics[width=0.87\textwidth]{Perkins_7}
\caption{Ширина ограниченных вертикальных трещин при течении ньютоновских жидкостей и турбулентном режиме потока}
\label{fig:Perkins7}
\end{figure}

Рассмотрим тот же пример, что и раньше, за исключением того, что вязкость жидкости для гидроразрыва составляет 1 сП: $Q = 30$ баррелей в минуту, $\mu =$ 1 сП, $L=500$ футов, $\text{Sp. Gr.}=0.9$ и $H = 40$ футов.
Тогда:
$$
\frac{Q\,(\text{Sp. Gr.})}{H\mu}=\frac{30\cdot 0.9}{40\cdot 1}=0.675.
$$
Таким образом, в этом случае поток жидкости турбулентный и из рис.7:
$$
\frac{Q^2\,(\text{Sp. Gr.})\,L}{H}=\frac{(30)^2(0.9)(500)}{40}=1.01\cdot 10^4.
$$

Рис.7 показывает, что максимальная ширина трещины вблизи ствола скважины будет примерно $0.14-0.16$ дюйма.

\subsubsection{Ширина трещин, образующихся в результате течения неньютоновских жидкостей при ламинарном течении}

Если используются неньютоновские жидкости, такие как загущенные масла или эмульсии, то перед оценкой ширины трещины необходимо определить характеристики текучести жидкости.
Данные вискозиметра Fann должны быть нанесены на график аналогично логарифмическому графику, представленному на рис. 13 в Приложении C.
Уравнение прямолинейного участка сводится к двум константам: $k'$, которая пропорциональна точке пересечения при определённой скорости ротора, и $n'$, которая пропорциональна наклону прямой
(детали фактического определения $k'$ и $n'$ по данным вискозиметра Fann приведены в Приложении C).
Затем эти две константы используются вместо <<вязкости>>.

К сожалению, нам не удалось построить графики ширины трещин для неньютоновских жидкостей также просто, как на рис. \ref{fig:Perkins6} и \ref{fig:Perkins7}.
Необходимо было включить в расчетную диаграмму модули Юнга пород пласта.
Значения модулей Юнга снова можно оценить по таблице \ref{tab:Perkins_table_1}.
После определения констант потока $k'$ и $n'$ для жидкости и оценки модуля Юнга породы можно оценить ширину трещины по рис. \ref{fig:Perkins8} (вывод в приложении C).

\begin{figure}[H]
\center
\includegraphics[width=0.5\textwidth]{Perkins_8}
\caption{Ширина ограниченных вертикальных трещин при течении неньютоновских жидкостей и ламинарном режиме потока}
\label{fig:Perkins8}
\end{figure}

Предположим, что для данного случая гидроразрыва $k'=4.69\cdot 10^{-2}\text{ фунт}\cdot\text{с}^{n'}/\text{фут}^2$, $n'=0.654$ (эти конкретные значения для $k'$ и $n'$ для очень густых гелеобразных жидкостей гидроразрыва), $Q=20\text{ баррелей в минуту}$, $L=100\text{ футов}$, $H=30\text{ футов}$ и $E=6\cdot10^6\text{ psi}$ (для твёрдого песчаника из таблицы \ref{tab:Perkins_table_1}).

$$
\frac{k'Q^{n'}LH^{1-n'}}{E}=\frac{(4.69\cdot10^{-2})(20)^{0.654}(100)(30)^{0.346}}{6\cdot10^6}=1.8\cdot10^{-5}.
$$
Из рис. \ref{fig:Perkins8} максимальная ширина трещины около скважины в рассматриваемом кейсе оценивается в 0.2 дюйма.

\subsubsection{Ширина трещин, образующихся, когда песок в трещине ограничивает поток жидкости}

Оценки для ширины трещин из рис. \ref{fig:Perkins6}, \ref{fig:Perkins7} и \ref{fig:Perkins8} применяются, когда в трещину закачивается чистый флюид.
Эти оценочные значения ширины действительны и при редком распределении расклинивающего агента, взвешенного в жидкости (поскольку разбавленная суспензия твердого материала не окажет заметного влияния на перепад давления вдоль трещины).
Однако, если в качестве расклинивающего агента закачивается большое количество песка, то его присутствие в трещине будет влиять на перепад давления и тем самым на ширину трещины.

При полевых работах песок обычно помещают в трещину таким образом, чтобы обеспечить одно из двух следующих условий: (1) большое количество песка оседает на дно трещины; или (2) песок почти равномерно взвешен в трещине и сконцентрирован за счет просачивания жидкости гидроразрыва.

Условия, приводящие к оседанию песка в трещине, обсуждались в предыдущей статье [14].
Ширину трещины для этих условий рассчитать непросто.
Мы изучили несколько случаев и обнаружили, что ширина трещины была в несколько раз больше, чем предсказанная ширина трещины без присутствия песка.
В настоящее время мы можем только предположить, что для всех случаев, когда присутствует песок, ширина трещины должна оцениваться описанным ниже способом верным для случая, когда песок концентрируется из-за просачивания флюида, но не оседает.

Если песок находится во взвешенном состоянии в очень вязкой жидкости, он будет оседать так медленно, что осевшая песчаная пачка не успеет образоваться во время гидроразрыва.
Однако во время обработки жидкость для гидроразрыва может просачиваться в пористую формацию, оставляя высокую концентрацию песка во взвешенном состоянии в трещине.
Вязкость концентрированной песчаной суспензии будет значительно выше, чем вязкость чистой жидкости для гидроразрыва.
На рис. \ref{fig:Perkins9} показано отношение вязкости раствора к вязкости чистой жидкости гидроразрыва в зависимости от объемной доли твердого материала в растворе.

\begin{figure}[H]
\center
\includegraphics[width=0.5\textwidth]{Perkins_9}
\caption{Вязкость суспензии, содержащей взвешенные твёрдые частицы, по сравнению с вязкостью чистой жидкости}
\label{fig:Perkins9}
\end{figure}

В случае, если песок концентрируется за счет просачивания флюида, ширина трещины легко оценивается следующим образом.

1. Количество жидкости, оставшейся в трещине, оценивается по методу, предложенному Ховардом и Фастом [10].
Зная количество жидкости в трещине и количество песка в трещине, можно рассчитать среднюю концентрацию шлама.

2. Средняя вязкость суспензии может быть оценена по рис. \ref{fig:Perkins9}.

3. Ширина трещины затем оценивается по рис. \ref{fig:Perkins6} и \ref{fig:Perkins7}, используя среднюю вязкость и плотность суспензии, а не вязкость и плотность чистой жидкости для гидроразрыва.

В действительности свойства суспензии изменяются вдоль трещины.
Следовательно, ширина, рассчитанная только что показанным образом, должна интерпретироваться только как приблизительная ширина.

\subsection{Горизонтальные трещины}

Если трещина ориентирована горизонтально, ширина трещины может быть результатом двух типов движения породы.
Если трещина находится глубоко в земле, ширина трещины в основном определяется сжатием породы вблизи трещины.
Однако, если трещина очень неглубокая, ширина трещины также может быть результатом изгиба и подъема покрывающей породы.
Концептуально это показано на рис. 10.

\begin{figure}[H]
\center
\includegraphics[width=0.5\textwidth]{Perkins_10}
\caption{Глубокие и неглубокие трещины гидроразрыва}
\label{fig:Perkins10}
\end{figure}

В Приложении D показано, что сжатие окружающей породы является основным механизмом, приводящим к расширению трещины, если ее глубина превышает примерно три четверти радиуса трещины (следовательно, это механизм, который контролирует большинство процедур гидроразрыва.)
Для этого условия ширина приблизительно определяется выражением 7.
\beq\label{Perkins_7}
W(\text{in.})=0.22\left[\frac{Q(\text{bbl/min})\,\mu(\text{cp})\,C(\text{ft})}{E(\text{\text{psi}})}\right]^{1/4},
\tag{7}
\eeq
где $C$ -- радиус трещины (футов).\newline
При выводе этого выражения мы предполагали, что среда однородна.
На самом деле, в случае горизонтальной трещины значение модуля Юнга следует считать равным эффективному среднему значению для продуктивной зоны и пластов выше и ниже нее, поскольку они также сжимается при открытии трещины.
Однако для простоты мы использовали значения модуля Юнга для продуктивных зон, как показано в таблице \ref{tab:Perkins_table_1}, при применении этого уравнения.

\begin{figure}[H]
\center
\includegraphics[width=0.87\textwidth]{Perkins_11}
\caption{Приблизительная ширина горизонтальных трещин при течении ньютоновских жидкостей и ламинарном режиме потока}
\label{fig:Perkins11}
\end{figure}

На рис. \ref{fig:Perkins11} графически показан порядок величины ожидаемой ширины горизонтальных трещин (глубина больше трех четвертей радиуса трещины), если поток жидкости ламинарный.
Ламинарное течение жидкости в каждой точке горизонтальной трещины, вероятно, редко встречается при полевых работах.
Следовательно, турбулентный поток также необходимо учитывать, прежде чем можно будет вывести общеприменимое уравнение.
Однако турбулентная зона обычно не простирается далеко от ствола скважины; поэтому рис. \ref{fig:Perkins11} примерно верен во всех случаях.

В приложение D также есть выражение для оценки ширины неглубоких горизонтальных трещин (глубина составляет менее трех четвертей радиуса трещины).

Если в трещину закачивается большое количество взвешенного песка, то ширину трещины можно оценить, используя вязкость песчаной суспензии, а не вязкость основной жидкости гидроразрыва.

\subsection{Следствия представленных концепций для ширины трещины}

Теперь давайте кратко обсудим некоторые следствия и значение рассмотренных представлений о ширине трещин гидроразрыва.
Во-первых, рассмотрим взаимосвязь между шириной трещины и давлением гидроразрыва.
Падение давления вдоль трещины можно оценить по формуле A-10 или D-4 (в Приложениях A и D, соответственно), а ширина трещины определяется из выражений \ref{Perkins_5}, \ref{Perkins_6} или \ref{Perkins_7}.
Для вертикально ориентированных трещин падение давления вдоль трещины, вероятно, будет варьироваться от нескольких десятков фунтов на квадратный дюйм (для очень высоких трещин) до примерно 1000 фунтов на квадратный дюйм (для тонких зон).
Более высокие давления в тонких зонах увеличивают вероятность прорыва из продуктивной зоны в вышележащие и нижележащие зоны.

Для горизонтальных трещин или неограниченных вертикальных трещин перепады давления вначале будут очень высокими, но довольно низкими для крупных трещин (т.е. существенно уменьшаются по мере развития трещины).

Забойные давления гидроразрыва будут равны сумме полного напряжения породы, перпендикулярного плоскости трещины, и перепада давления в трещине.
Хабберт и Уиллис [11] и Клири [4] обсудили факторы, влияющие на напряжения породы.

На рис. \ref{fig:Perkins12} показаны наши оценки диапазонов напряжений породы (аналогичные оценкам Хабберта и Уиллиса или Клири) плюс перепады давления для вертикальных и горизонтальных трещин (как обсуждается в этой статье).

\begin{figure}[H]
\center
\includegraphics[width=0.55\textwidth]{Perkins_12}
\caption{Теоретически предсказанные забойные давления и полевые данные}
\label{fig:Perkins12}
\end{figure}

На этом рисунке также показаны некоторые забойные давления гидроразрыва, рассчитанные на основе фактических полевых режимов.
Очевидно, что забойное давление для большинства этих ГРП упало в диапазоне, прогнозируемом для вертикальных трещин.
Интересно отметить, что прогнозируемые давления для вертикальных и горизонтальных трещин перекрываются на глубинах менее примерно 3000 футов.
Следовательно, на таких малых глубинах ориентация трещины не может быть определена путем беглого изучения градиента давления гидроразрыва на забое.

Еще одним интересным следствием этих концепций является ожидаемое поведение после прекращения закачки.
Поскольку давление распространения трещины почти равно напряжению грунта, трещина будет продолжать распространяться после прекращения закачки.
Трещина будет становиться уже и длиннее до тех пор, пока не будет достигнута минимальная ширина (обсуждавшаяся ранее) или пока стенки пласта не захватят распорный материал.
Конечно, флюид будет продолжать просачиваться по всей площади трещины и особенно вблизи выступающего края трещины по мере смыкания стенок.
Это ограничивает дополнительную длину, полученную после прекращения закачки.

Наконец, давайте рассмотрим факторы, влияющие на ширину трещины (см. выражения \ref{Perkins_5}, \ref{Perkins_6} и \ref{Perkins_7}).

1. Толщина продуктивной зоны не должна влиять на давление или ширину горизонтальных трещин.
Для ограниченных вертикальных трещин перепад давления вдоль трещины будет большим для тонких зон и небольшим для толстых зон.
С другой стороны, ширина трещины в стволе скважины почти не зависит от высоты вертикальной трещины (за исключением турбулентного течения).

2. Глубина продуктивной зоны обычно мало влияет на ширину трещины (за исключением того, что свойства породы могут меняться в зависимости от глубины).
Для очень неглубоких горизонтальных трещин ширина может быть несколько больше нормальной из-за подъема вскрышных пород.
Однако это не должно иметь значения в большинстве обычных операций по гидроразрыву пласта.

3. Ширина трещины не особенно чувствительна к свойствам породы.
Модули Юнга горных пород имеют диапазон примерно в десять-двадцать раз.
Однако ширина трещины обратно пропорциональна четвертому корню из модуля Юнга;
тем не менее, из этого диапазона модулей следует ожидать примерно двукратное изменение ширины трещины.

4. Поскольку вязкость жидкости для гидроразрыва (или эффективная вязкость, если речь идет о суспензиях) может варьироваться в очень широком диапазоне, этот фактор будет оказывать заметное влияние на ширину трещины.

5. Скорость закачки насоса также влияет на ширину, но обычно диапазон скоростей насоса ограничен доступной мощностью.

6. Ширина трещины в стволе скважины также зависит от длины или радиуса трещины.
По мере увеличения объема жидкости в трещине ширина трещины будет увеличиваться.

7. На ширину трещины сильно влияет большое количество твердого расклинивающего материала в трещине.
Твердый материал увеличивает сопротивление потоку жидкости и приводит к расширению трещины.

\subsection{Список обозначений}

\subsection{Список цитированной литературы}

\subsection{Приложение A. Вывод выражения для ширины вертикальной трещины, образующейся при течении ньютоновской жидкости при ламинарном потоке}

Ширину ограниченных вертикальных трещин, образовавшихся в результате нагнетания ньютоновских жидкостей, можно оценить, сделав следующие предположения.

1. Предположим, что трещина вертикальна и имеет фиксированную высоту $H$ (высота не зависит от расстояния от ствола скважины).

2. Предположим, что форма поперечного сечения трещины в любой точке в основном эллиптическая и что максимальная ширина в этой точке пропорциональна разнице между давлением и напряжением в этой точке.
Уравнение эллипса принимается за уравнение Снеддона для двумерной системы (уравнение А-10).

3. Предположим, что падение давления может быть оценено с помощью уравнения Фаннинга, в котором гидравлический радиус заменен на эллипс.

4. Предположим, что давление жидкости на расширяющемся краю трещины практически равно общему напряжению грунта, перпендикулярному плоскости трещины.

5. Предположим, что утечки закачиваемой жидкости нет (предполагая предельное условие общей скорости просачивания жидкости достаточно большой, чтобы трещина больше не расширялась, можно легко показать, что утечка мало влияет на ширину трещины).

Уравнение Фаннинга может быть записано как уравнение А-1.
\beq\label{Perkins_A1}
\frac{dP}{dx}=\frac{2fv^2\rho}{D_e},
\tag{A-1}
\eeq
где, для ламинарного потока,
\beq\label{Perkins_A2}
f=\frac{16\mu}{D_e\nu\rho}
\tag{A-2}
\eeq

Перри показал, что для эллипса практически с нулевым эксцентриситетом (эксцентриситет связан с отношением малого диаметра к большему):
\beq\label{Perkins_A3}
D_e=\gamma R_H=\gamma\left(\frac{W}{2.546}\right)
\tag{A-3}
\eeq


Мы предполагаем, что значение $\gamma=3.6$ следует использовать для всех рассматриваемых здесь случаев, как ламинарных, так и турбулентных.
Теперь можно записать число Рейнольдса для жидкости, текущей в трещине:
\beq\label{Perkins_A6}
N_{Re}=\frac{D_e\nu\rho}{\mu}=\frac{\gamma R_H}{}
\tag{A-6}
\eeq


\subsection{Приложение B. Вывод выражения для ширины вертикальной трещины, образующейся при течении ньютоновской жидкости при турбулентном потоке}

Основные предположения для этого случая такие же, как перечисленные в Приложении A.
Если число Рейнольдса больше 2500, то мы предполагаем, что
\beq\label{Perkins_B1}
f=0.0125
\tag{B-1}
\eeq

На коэффициент трения влияет гладкость стенок трещины; значение $f=0.0125$ соответствует относительной шероховатости [17] около $0.02$.
Влияние относительной шероховатости ясно продемонстрированы в опытах Хюита [12].

Подстановка выражений \eqref{Perkins_A3}, \eqref{Perkins_A5}, \eqref{Perkins_A8}, \eqref{Perkins_A10} и \eqref{Perkins_B1} в уравнение \eqref{Perkins_A1} даёт уравнение \eqref{Perkins_B2}.

\beq\label{Perkins_B2}
\left(P-\sigma\right)^3dP=\left[\frac{(0.0125)(2.546)}{\pi^2(3.6)}\right]\left[\frac{\rho Q^2E^3}{(1-\nu^2)^3H^5}\right]dx
\tag{B-2}
\eeq

Интегрируя с использованием условия $\left(P-\sigma\right)=0$ на передней кромке трещины и подставляя в выражение \eqref{Perkins_A10}, получаем выражение \eqref{Perkins_B3} для максимальной ширины трещины вблизи ствола скважины.

\beq\label{Perkins_B3}
W=0.6\left[\frac{(Q^2)(\text{Sp. Gr.})(L)}{(E)(H)}\right]^{1/4}
\tag{B-3}
\eeq

\subsection{Приложение C. Вывод выражения для ширины вертикальной трещины, образующейся при течении неньютоновской жидкости при ламинарном потоке}

\subsection{Приложение D. Вывод выражения для примерной ширины горизонтальной трещины}


\end{document}
